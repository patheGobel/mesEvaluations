\documentclass[12pt]{article}
\usepackage{stmaryrd}
\usepackage{graphicx}
\usepackage[utf8]{inputenc}

\usepackage[french]{babel}
\usepackage[T1]{fontenc}
\usepackage{hyperref}
\usepackage{verbatim}

\usepackage{color, soul}

\usepackage{pgfplots}
\pgfplotsset{compat=1.15}
\usepackage{mathrsfs}

\usepackage{amsmath}
\usepackage{amsfonts}
\usepackage{amssymb}
\usepackage{tkz-tab}

\usepackage{tikz}
\usetikzlibrary{arrows, shapes.geometric, fit}


\usepackage[margin=2cm]{geometry}
\begin{document}

\begin{minipage}{0.5\textwidth}
	Ministère de l'éducation nationale  \\
	Inspection académique de Kédougou   \\
	Lycée de Dindéfélo            \\
	Cellule de mathématiques            \\
	M. BA                          \\
	Classe : TS  \\
\end{minipage}
\begin{minipage}{0.5\textwidth}
	Année scolaire 2023-2024 \\
	Date : 14-05-2024 \\
	Durée : 4h 00 \\
\end{minipage}

\begin{center}
	\section*{\textcolor{red}{\underline{Correction du Devoir N1 Du Second Semestre}}}
\end{center}
\section*{\textcolor{red}{\underline{Exercice 1} (2 points) :}}
Calcucler les limtes suivantes
\[\lim_{x \to +\infty}\frac{\ln(x-1)}{2x-3}\quad\quad\lim_{x \to +\infty}\frac{\ln(x)}{\sqrt{x}}\]
\section*{\textcolor{red}{\underline{Correction de l'exercice 1} (4 points) :}}
\section*{\textcolor{red}{\underline{Exercice 2} (4 points) :}}
1)Resoudre dans $\mathbb{C}$ l'équation $z^{4}=1$. \textbf{2pt}

2)En déduire les solutions dans $\mathbb{C}$ de l'équation $\left[\left(\sqrt{3}-i\right)z+i  \right]^{4}$=1. \textbf{2pt}
\section*{\textcolor{red}{\underline{Correction de l'exercice 2} (4 points) :}}
\section*{\textcolor{red}{\underline{Exercice 3} (4 points) :}}
Le plan complexe est muni d’un repère orthonormé $(O,\vec{u},\vec{v})$

On considère les points A et B d’affixes respectives a et b tels que

$a=-(\frac{1+\sqrt{3}}{2})(1-i)$ et $b=-(\frac{1+\sqrt{3}}{2})(1+i)$ 

1) Ecrire a et b sous forme trigonométrique.\textbf{1pt}

2) Quelle est la nature du triangle OAB ?\textbf{1pt}

3) On désigne par C le point d’affixe 1.

a) Ecrire le nombre complexe $\frac{1-a}{b-a}$  sous forme trigonométrique.\textbf{1pt}

b) En déduire la nature du triangle ABC et  utiliser ce résultat pour placer les points A, B, C sur la figure.\textbf{1pt}
\section*{\textcolor{red}{\underline{Correction de l'exercice 3} (4 points) :}}
\section*{\textcolor{red}{\underline{Problème} (10,5 points):}}
\subsection*{Partie A:}
Soit $f$ la fonction numérique définie par $f(x)=-x+e^{x-1}$  et $C_{f}$ sa courbe dans un repère orthonormé $(O,\vec{i},\vec{j})$ unité 1cm

1) Calculer les limites de $f$ aux bornes de son ensemble de définition puis étudier la nature des branches infinies.\textbf{2pts}

Préciser la position de la courbe par rapport à son asymptote.\textbf{1pt}

2) Dresser le tableau de variation de $f$ puis en déduire le signe de $f(x)$.\textbf{1,5pts}

3) Construire $C_{f}$.\textbf{0,5pts}

4) Soit $g$ la fonction définie sur $\mathbb{R}$ par $g(x)=x+e^{-x-1}$

  a) Exprimer $g(x)$ en fonction de $f(x)$ \textbf{0,5pt}
  
  b) Construire alors $C_{g}$ dans le même repère.  \textbf{0,5pt}
\section*{Partie B:}
Soit $h$ la fonction définie par $h(x)=-x+\ln(-x+e^{x-1})$  et $\Gamma$ sa courbe dans autre repère orthonormal

1)

	a)Déterminer l’ensemble de définition de $h$ puis calculer les limites de $h$ aux bornes de son ensemble de définition.Puis déduire les asyptotes enventuelles \textbf{1,5pt}\\
Etudier $\Gamma$ au voisinage de $-\infty$

	b)Montrer que $\forall x>1, h(x)=\ln(-xe^{-x}+e^{-1}) $ . \textbf{.0,5pt}
	
2) Dresser le tableau de variation de  $h$.   \textbf{1pt}
    
3) Soit $\varphi$ la restriction de $h$ à l’intervalle $I=\left]-\infty, 1\right[ $
    
    a) Montrer que $\varphi$  réalise une bijection de $I$ vers $J$  à préciser.
    \textbf{0,5pt}
        
    b) Montrer que l’équation $\varphi(x)=0$ admet une solution unique $\alpha$ dont on précisera une valeur approchée à $10^{-1}$ prés   \textbf{1pt}
    
    c) Construire $\Gamma$ et la courbe de $\varphi^{-1}$  dans le même repère.  \textbf{1pt}
\section*{\textcolor{red}{\underline{Correction du Problème} (10,5 points):}}
\subsection*{Partie A:}
1)Le domaine de définition de $f$:\\
$f$ est la somme d'une fonction polynomiale et d'une fonction exponentielle définie sur $\mathbb{R}$ donc $Df=\mathbb{R}$\\
Les limites aux bornes de $Df$.\\
Les bornes de $Df$ sont $-\infty$ et $+\infty$\\
\underline{En $-\infty$}:
\[\lim_{x \to -\infty}f(x)=\lim_{x \to -\infty}-x+e^{x-1}=+\infty\]
Donc \textcolor{green}{\[\lim_{x \to -\infty}f(x)=+\infty\]}
\underline{En $+\infty$}:
\begin{equation*}
\lim_{x \to +\infty}f(x)=\lim_{x \to +\infty}-x+e^{x-1}=+\infty
\begin{cases}
\lim_{x \to +\infty} -x=-\infty\\
\lim_{x \to +\infty}e^{x-1}=+\infty
\end{cases}
\textbf{Par somme, FI}
\end{equation*}
Levons l'indétermination\\
\begin{equation*}
\lim_{x \to +\infty}f(x)=\lim_{x \to +\infty}x\left(-\frac{1}{x}+\frac{e^{x-1}}{x}\right)=+\infty
\end{equation*}
Donc \textcolor{green}{\[\lim_{x \to -\infty}f(x)=+\infty\]}
\textbf{Branche parabolique}
 
\underline{En $-\infty$}:
\begin{equation*}
\lim_{x \to -\infty}\frac{f(x)}{x}=\lim_{x \to -\infty}\frac{-x+e^{x-1}}{x}=\lim_{x \to -\infty}\frac{-x}{x}+\frac{e^{x-1}}{x}=\lim_{x \to -\infty}-1+e^{-1}\frac{e^{x}}{x}=-1
\quad\text{donc} \lim_{x \to -\infty}\frac{f(x)}{x}=-1
\end{equation*}
Cherchons $\lim_{x \to -\infty}\left[ f(x)+x\right] $
\[\lim_{x \to -\infty}\left[ f(x)+x\right]=\lim_{x \to -\infty}-x+e^{x-1}+x=\lim_{x \to -\infty}e^{x-1}=0\]
Donc $y=-x$ est asymptote oblique à $(C_{f})$ au voisinage de $+\infty$

\underline{En $+\infty$}:
\[\lim_{x \to +\infty}\frac{f(x)}{x}=\lim_{x \to +\infty}\frac{-x+e^{x-1}}{x}=\lim_{x \to +\infty}\frac{-x}{x}+\frac{e^{x-1}}{x}=+\infty\]
Donc \[\lim_{x \to +\infty}\frac{f(x)}{x}=+\infty\] Donc $(Cf)$ admet une branche parabolique de direction (Oy)

La position de de $Cf$ par rapport à $y=-x$

Pour ce faire, chercons le signe de $\left[ f(x)-y\right] $

$\left[ f(x)-y\right]=-x+e^{x-1}-(-x)=e^{x-1}$ donc $\left[ f(x)-y\right]=e^{x-1}$

Ainsi, $\forall x \in \mathbb{R}$, $e^{x-1}>0$ donc $\forall x \in \mathbb{R}$, $\left[ f(x)-y\right]>0$

donc $\forall x \in \mathbb{R}$, $Cf$ est au dessus de (D):$y=-x$

2)Dressons le tableau de variation puis donons le signe de $f(x)$

Calculons la dérivé $f'$ de $f$

$f'(x)=-1+e^{x-1}$

Cherchons le signe de f':

Supposons que $\forall x \in \mathbb{R}$, $f'(x)>0$

$f'(x)>0 \Longrightarrow -1+e^{x-1}>0 \Longrightarrow e^{x-1}>1 \Longrightarrow x-1>0 \Longrightarrow x>1$ 

donc $f'(x)>0$ si $x\in\left]1; +\infty\right[ $ et $f'(x)<0$ si $x\in\left]-\infty; 1\right[ $

Variation de $f$:

\begin{itemize}
\item[•] Si $x\in\left]1; +\infty\right[, f'(x)>0$ donc $f$ est croissante
\item[•] Si $x\in\left]-\infty; 1\right[, f'(x)<0$ donc $f$ est décroissante
\end{itemize}
\textcolor{red}{Tableau de variation}\\
%Tableau de Variation
\definecolor{cqcqcq}{rgb}{0.7529411764705882,0.7529411764705882,0.7529411764705882}
\begin{tikzpicture}[line cap=round,line join=round,>=triangle 45,x=1cm,y=1cm]
%\draw [color=cqcqcq,, xstep=1cm,ystep=1cm] (-7,-10) grid (-22,17);
\clip(-22,-5) rectangle (12,10);
\draw [line width=2pt] (-23,8)-- (-7,8); %première ligne A(-22,8)---B(-7,8)
\draw [line width=2pt] (-22,6)-- (-7,6); %deuxième ligne
\draw [line width=2pt] (-22,4)-- (-7,4); %troisième ligne
\draw [line width=2pt] (-22,-2)-- (-7,-2);%dernière ligne
\draw [line width=2pt] (-22,-2)-- (-22,8); %première colonne
\draw [line width=2pt] (-19,8)-- (-19,-2); %deuxième colone
\draw [line width=2pt] (-13,6)-- (-13,-2); %troisième colonne
\draw [line width=2pt] (-7,8)-- (-7,-2); %quatrième colonne
\draw (-21,1.5) node[anchor=north west] {$f(x)$};
\draw (-21,5.5) node[anchor=north west] {$f'(x)$};
\draw (-21,7) node[anchor=north west] {$x$};
\draw (-19,7) node[anchor=north west] {$-\infty$};
\draw (-13.3,6.5) node[anchor=north west] {$1$};
\draw (-8,7) node[anchor=north west] {$+\infty$};
%signe de la dérivé
\draw (-17,5.3) node[anchor=north west] {$-$};
\draw (-13.3,5.3) node[anchor=north west] {$O$};
\draw (-10,5.3) node[anchor=north west] {$+$};

\draw [->,line width=2pt] (-18,3) -- (-13.5,-1.3);
\draw (-18.8,3.9) node[anchor=north west] {$+\infty$};
\draw (-13.25,-1) node[anchor=north west] {\textbf{\textcolor{blue}{0}}};

\draw [->,line width=2pt] (-12.5,-1.3) -- (-8,3.5);
\draw (-8,3.9) node[anchor=north west] {$+\infty$};
\end{tikzpicture}

$f(1)=0$

d'après le tableau de variation, $\forall x\in \mathbb{R}\setminus\left\lbrace 1\right\rbrace $, $f(x)>0$

3) Construisons la courbe (Cf)

\begin{tikzpicture}
\begin{axis}[
    axis lines = middle,
    xlabel = \(x\),
    ylabel = \(f(x)\),
    xmin = -4,
    xmax = 4,
    ymin = -4,
    ymax = 4,
    domain = -4:4,
    samples = 100,
    grid = both,
    legend pos = north west,
]

\addplot[blue, thick] {-x + exp(x-1)};
\addplot[green, thick] {x + exp(-x-1)};
%\addlegendentry{\(-x + e^{x-1}\)}
\addplot[red, thick] {-x};
%\addlegendentry{\(-x\)}
\end{axis}
\end{tikzpicture}

4)

a) Exprimons $g(x)$ en fonction de $f(x)$

On a : $g(x)=x+e^{-x-1}$

$g(-x)=-x+e^{x-1}=f(x)$

donc $g(-x)=f(x)$

 b) Construisons $C_{g}$ [Voir figure]
 
 \section*{Partie B:}
 1)

	a)Ensemble de définition de $h$

Nous remarquons que 	$h(x)=-x+\ln(f(x))$

Or $\forall x\in \mathbb{R}\setminus\left\lbrace 1\right\rbrace $, $f(x)>0$ et $x$ existe

donc $Dh=\mathbb{R}\setminus\left\lbrace 1\right\rbrace $

Limites aux bornes de $Dh$

les bornes de $Dh$ sont $-\infty$, 1 et $+\infty$

\underline{En $-\infty$}:
\begin{equation*}
\lim_{x \to -\infty}h(x)=\lim_{x \to -\infty}\left[ -x+\ln(-x+e^{x-1})\right] 
\begin{cases}
\lim_{x \to -\infty} -x+e^{x-1}=+\infty\\
\lim_{y \to +\infty}\ln(y)=+\infty\\
\lim_{x \to -\infty}-x=+\infty
\end{cases}
\end{equation*}
\textcolor{red}{\[\text{Par composition,}\lim_{x \to -\infty}\ln(-x+e^{x-1})=+\infty\quad\text{et par somme,}\lim_{x \to -\infty}h(x)=+\infty\]} 
\textcolor{red}{\[\text{donc,}\quad h(x)=+\infty\]}

\underline{En $+\infty$}:
\begin{equation*}
\lim_{x \to +\infty}h(x)=\lim_{x \to +\infty}\left[-x+\ln(-x+e^{x-1})\right] 
\begin{cases}
\lim_{x \to +\infty} -x+e^{x-1}=+\infty\\
\lim_{y \to +\infty}\ln(y)=+\infty\\
\lim_{x \to +\infty}-x=-\infty
\end{cases}
\end{equation*}
\textcolor{red}{\[\text{Par composition,}\lim_{x \to +\infty}\ln(-x+e^{x-1})=+\infty\quad\text{et par somme, forme indéterminée.}\]} 
%\textcolor{red}{\[\text{donc,}\quad h(x)=+\infty\]}
Levons l'indétermination
\begin{equation*}
\lim_{x \to +\infty}h(x)=\lim_{x \to +\infty}\left[-x+\ln(-x+e^{x-1})\right]=\lim_{x \to +\infty}\left[-x+\ln\left[e^{x}(-xe^{-x}+e^{-1})\right] \right]=\lim_{x \to +\infty}\ln\left[(-xe^{-x}+e^{-1})\right]
\end{equation*}
\begin{equation*}
\lim_{x \to +\infty}h(x)=\lim_{x \to +\infty}\ln\left[(-xe^{-x}+e^{-1})\right]
\begin{cases}
\lim_{x \to +\infty} -xe^{-x}+e^{-1}=\lim_{x \to +\infty} -\frac{x}{e^{x}}+e^{-1}=e^{-1}\\
\lim_{y \to e^{-1}}\ln(y)=-1
\end{cases}
\end{equation*}
\textcolor{red}{\[\text{Par composition,}\lim_{x \to +\infty}\ln(-x+e^{x-1})=-1\]} 
\textcolor{red}{\[\text{donc,}\lim_{x \to +\infty}h(x)=-1\]} 
\underline{En $1^{-}$}:
\begin{equation*}
\lim_{x \to 1^{-}}h(x)=\lim_{x \to 1^{-}}-x+\ln(-x+e^{x-1})
\begin{cases}
\lim_{x \to 1^{-}} -x+e^{x-1}=\lim_{x \to 1^{-}}f(x)=0^{+}\\
\lim_{y \to 0^{+}}\ln(y)=-\infty\\
\lim_{x \to 1^{-}} -x=-1
\end{cases}
\end{equation*}
%\textcolor{red}{\[\lim_{x \to 1^{-}}h(x)=-\infty\]}
\textcolor{red}{\[\text{Par composition,}\lim_{x \to 1^{-}}\ln(-x+e^{x-1})=-\infty\quad\text{et par somme,}\lim_{x \to 1^{-}}-x+\ln(-x+e^{x-1})=-\infty\]} 

\textcolor{red}{\[\text{Donc}\lim_{x \to 1^{-}}h(x)=-\infty\]}

\underline{En $1^{+}$}:
\begin{equation*}
\lim_{x \to 1^{+}}h(x)=\lim_{x \to 1^{+}}-x+\ln(-x+e^{x-1})
\begin{cases}
\lim_{x \to 1^{+}} -x+e^{x-1}=\lim_{x \to 1^{+}}f(x)=0^{+}\\
\lim_{y \to 0^{+}}\ln(y)=-\infty\\
\lim_{x \to 1^{-}}-x=-1
\end{cases}
\end{equation*}
\textcolor{red}{\[\text{Par composition,}\lim_{x \to +\infty}\ln(-x+e^{x-1})=-\infty\quad\text{et par somme,}\lim_{x \to 1^{-}}-x+\ln(-x+e^{x-1})=-\infty\]} 

\textcolor{red}{\[\text{Donc}\lim_{x \to 1^{+}}h(x)=-\infty\]}

b)Montrons que $\forall x>1, h(x)=\ln(-xe^{-x}+e^{-1}) $ 

On a: $h(x)=-x+\ln(-x+e^{x-1})=-x+\ln\left[e^{x}(-xe^{-x}+e^{-1})\right]=-x+x+\ln\left[(-xe^{-x}+e^{-1})\right]$

Donc $h(x)=\ln\left[(-xe^{-x}+e^{-1})\right]$

%\subsection*{Montrons que la droite $y=x-1$ d’équation est une asymptote à $\Gamma$.}

%\[h\quad \text{est asymptote à\quad} \Gamma\quad \text{si} \lim_{x \to +\infty}\left[ h(x)-y\right] =0\]

%$h(x)=\ln\left[(-xe^{-x}+e^{-1})\right]=\left[ -x+\ln(-x+e^{x-1})\right]$
%\[\lim_{x \to +\infty}\left[ h(x)-y\right]=\lim_{x \to +\infty}\left[ -x+\ln(-x+e^{x-1})-(-x-1)\right]=\lim_{x %%\to +\infty}\left[ \ln(-x+e^{x-1})+1\right]\]
%\[\text{Or}\quad\lim_{x \to +\infty}\left[ \ln(-x+e^{x-1})\right]=-1\]
%\[\text{donc}\quad\lim_{x \to +\infty}\left[ h(x)-y\right] =0\]
%Donc $y=x-1$ d’équation est une asymptote à $\Gamma$ au voisinage de $+\infty$.

2) Dressons le tableau de variation de  $h$. 

Dérivons $h$:

$h'(x)=-1+\left[\frac{f'(x)}{f(x)}\right]=-1+\left[\frac{-1+e^{x-1}}{-x+e^{x-1}}\right]=\frac{x-e^{x-1}-1+e^{x-1}}{-x+e^{x-1}}=\frac{x-1}{-x+e^{x-1}}$

Donc $h'(x)=\frac{x-1}{f(x)}$

D'après la réponse à la question 2), $\forall x\in \mathbb{R}\setminus\left\lbrace 1\right\rbrace $, $f(x)>0$

Donc le signe de $h(x)$ dépend de celui du numérateur $x-1$
\begin{itemize}
\item[•] Si $x\in \left]-\infty; 1\right[$, f'(x)<0 donc f est décroissante
\item[•] Si $x\in \left]1; +\infty\right[$, f'(x)>0 donc f est croissante
\end{itemize}

\textcolor{red}{Tableau de variation}\\
\definecolor{cqcqcq}{rgb}{0.7529411764705882,0.7529411764705882,0.7529411764705882}
\begin{tikzpicture}[line cap=round,line join=round,>=triangle 45,x=1cm,y=1cm]
%\draw [color=cqcqcq,, xstep=1cm,ystep=1cm] (-7,-10) grid (-22,17);
\clip(-22,-5) rectangle (12,10);
\draw [line width=2pt] (-23,8)-- (-7,8); %première ligne A(-22,8)---B(-7,8)
\draw [line width=2pt] (-22,6)-- (-7,6); %deuxième ligne
\draw [line width=2pt] (-22,4)-- (-7,4); %troisième ligne
\draw [line width=2pt] (-22,-2)-- (-7,-2);%dernière ligne
\draw [line width=2pt] (-22,-2)-- (-22,8); %première colonne(-21,-2)<-- (-22,8);
\draw [line width=2pt] (-19,8)-- (-19,-2); %deuxième colone
\draw [line width=2pt] (-13,6)-- (-13,-2); %troisième colonne
\draw [line width=2pt] (-13.2,6)-- (-13.2,-2); %troisième colonne
\draw [line width=2pt] (-7,8)-- (-7,-2); %quatrième colonne
\draw (-21,1.5) node[anchor=north west] {$h(x)$};
\draw (-21,5.5) node[anchor=north west] {$h'(x)$};
\draw (-21,7) node[anchor=north west] {$x$};
\draw (-19,7) node[anchor=north west] {$-\infty$};
\draw (-13.3,6.5) node[anchor=north west] {$1$};
\draw (-13.1,-1) node[anchor=north west] {$-\infty$};
\draw (-14.3,-1) node[anchor=north west] {$-\infty$};
\draw (-8,7) node[anchor=north west] {$+\infty$};
%signe de la dérivé
\draw (-17,5.3) node[anchor=north west] {$-$};
\draw (-13.39,5.3) node[anchor=north west] {\textbf{\textcolor{blue}{O}}};
\draw (-10,5.3) node[anchor=north west] {$+$};

\draw [->,line width=2pt] (-18,3) -- (-13.5,-1);
\draw (-18.8,4) node[anchor=north west] {$+\infty$};

\draw [->,line width=2pt] (-12.5,-1) -- (-8,3);
\draw (-8,4) node[anchor=north west] {$-1$};
\end{tikzpicture}

%Cherchons le point où la courbe $\Gamma$ coupent l'axe des abscisses .\\
%Pour ce faire résolvons l'équation $h(x)=0 \Rightarrow \ln(-xe^{-x}+e^{-1})=\ln(1)$\\
%$\ln(-xe^{-x}+e^{-1})=\ln(1) \Rightarrow -xe^{-x}+e^{-1}=1 \Rightarrow -xe^{-x}=1-e^{-1}$

\begin{tikzpicture}
\begin{axis}[
    axis lines = middle,
    xlabel = \(x\),
    ylabel = \(f(x)\),
    xmin = -4,
    xmax = 4,
    ymin = -4,
    ymax = 4,
    domain = -4:4,
    samples = 100,
    grid = both,
    legend pos = north west,
]

\addplot[green, thick] {-x+ln(-x+exp(x-1))};
\addplot[blue, thick] {-1};
\addplot[yellow, thick] {-x};    
 % Asymptote verticale x = 1
    \addplot[red, thick] coordinates {(1,\pgfkeysvalueof{/pgfplots/ymin}) (1,\pgfkeysvalueof{/pgfplots/ymax})};
    \node[above right] at (axis cs:1,3) {\(x=1\)};  
 \addplot[blue, thick] {x};
\end{axis}
\end{tikzpicture}

3)$\varphi$ la restriction de $h$ à l’intervalle $I=\left]-\infty; 1 \right[ $
   
    a) Montrons que $\varphi$  réalise une bijection de $I$ vers $J$  à préciser.
    
$\varphi$ est continue et stricement décroissante sur $I=\left]-\infty; 1 \right[  $ donc c'est une bijection de \[\text{I vers } \varphi(I)\]

\[\varphi(I)=\left] \lim_{x \to 1^{-}}\varphi(x); \lim_{x \to -\infty}\varphi(x) \right[=\left]-\infty; +\infty\right[\]
    Donc $J=\left]-\infty; +\infty\right[$
    
b) Montrer que l’équation $\varphi(x)=0$ admet une solution unique $\alpha$ dont on précisera une valeur approchée à $10^{-1}$ prés \textbf{.1pt}
\subsection*{Existence}
\[\varphi(x)=0 \text{ admet une solution sur $I$ si } \lim_{x \to -\infty}\varphi(x)\times\lim_{x \to 1^{-}}\varphi(x)<0\]
Or
\[\lim_{x \to -\infty}\varphi(x)=+\infty \text{ et} \lim_{x \to 1^{-}}\varphi(x)=-\infty\]

\[\text{Donc } \lim_{x \to -\infty}\varphi(x)\times\lim_{x \to 1^{-}}\varphi(x)<0\]
Donc l'équation $\varphi(x)=0$ admet une solution sur $I$.
\subsection*{Unicité}
Comme $\varphi(x)$ est bijectif et strictement croissant et que $0\in J$ donc la solution est unique. 

Une valeur approchée de $\alpha$:

Je n'ai pas trop d'idée pour le moment
\end{document}