\documentclass{article}
\usepackage{amsmath}
\usepackage{amssymb}
\usepackage[margin=2cm]{geometry}

\begin{document}

\begin{minipage}{0.5\textwidth}
	Ministère de l'éducation nationale  \\
	Inspection académique de Kédougou   \\
	Lycée de Dindéfelo            \\
	Cellule de mathématiques            \\
	M. BA                          \\
	Classe : Tle  \\
\end{minipage}
\begin{minipage}{0.5\textwidth}
	Année scolaire 2023-2024 \\
	Date : 14-05-2024 \\
	Durée : 3h 00 \\
\end{minipage}

\begin{center}
	\textbf{{\underline{Devoir N1 Du Second Semestre}}}
\end{center}

\section*{Exercice 1 (4 points) :}
1) Résoudre dans $\mathbb{R}$ les équations suivantes  (1 pt $\times$ 2)\\
a)$\ln(x) = \ln(5 –x)$ ;\quad\quad  b)$\ln(5 –x)=-5$ \\
2)Résoudre dans $\mathbb{R}$ les inéquations suivantes (1 pt $\times$ 2)\\
d)$\ln(x-2)<0$ ;\quad\quad e)$1<\ln(5 –x)$
\section*{Exercice 2 (11 points) :}
Soit  $f$  la fonction définie par : $f(x) =\frac{x^{2}-2x+1}{x}$ .\\
    1- a - Donner  l’ensemble de définition  Df .\textbf{(0,5pt)}\\
b - Montrer que pour tout x $\in$ Df ; $f(x) =x-2+\frac{1}{x} $.\textbf{(1pt)}\\
       2  - Etudier les limites de f aux bornes des intervalles de définition.\textbf{(2pts)}\\ 
              En déduire que f admet une asymptote verticale.\textbf{(0,5pt)}\\
       3 - Calculer  f'(x) puis étudier son signe. Dresser le tableau de variation de f \textbf{(2pt)}.\\
             En déduire les extrema  de f.\textbf{(0,5pt)}\\
        4 - Vérifier que  I (0 ; -2) est centre de symétrie de Cf.\textbf{(1pt)} \\
        5 -  Montrer la droite (D) d’équation  y = x - 2  est une asymptote à (Cf).\textbf{(1pt)} \\
             Etudier la position de (Cf) par rapport à (D).\textbf{(1pt)}\\
         6 - Construire (Cf) dans un repère orthonormal.\textbf{(1,5pt)}\\
\section*{Exercice 3 (6 points) :}
Le tableau ci-dessous  donne la quantité de matière première X en tonnes (t) et le chiffre d’affaire Y en millions francs (f) d’une entreprise. On considère  la série double(x ; y) :

\begin{tabular}{|c|c|c|c|c|c|c|c|}
\hline
X & 2 & 2,5 & 3 & 3,5 & 4 & 4,5 & 5 \\
\hline
Y & 21 & 25 & 29 & 30 & 40 & 46 & 53\\
\hline
\end{tabular}\\
\begin{itemize}
\item[1)] Représenter le nuage de points.\\
\item[2)] Calculer la variance de X et la variance de Y.\\
\item[3)]  Calculer la covariance de X et Y.\\
\item[4)] Calculer le coefficient de corrélation linéaire r.\\
\item[5)] a)  Déterminer l équation de la droite de régression de Y en X puis la représenter\\
\item[b)]  En déduire une estimation du chiffre d’affaire pour 8 tonnes de matière première.\\
\item[c)]  Pour un chiffre d’affaire 100000000 calculer la quantité de matière première.

\end{itemize}
\end{document}