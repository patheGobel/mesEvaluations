\documentclass[12pt]{article}
\usepackage{stmaryrd}
\usepackage{graphicx}
\usepackage[utf8]{inputenc}

\usepackage[french]{babel}
\usepackage[T1]{fontenc}
\usepackage{hyperref}
\usepackage{verbatim}

\usepackage{color, soul}

\usepackage{pgfplots}
\pgfplotsset{compat=1.15}
\usepackage{mathrsfs}

\usepackage{amsmath}
\usepackage{amsfonts}
\usepackage{amssymb}
\usepackage{tkz-tab}

\usepackage{tikz}
\usetikzlibrary{arrows, shapes.geometric, fit}


\usepackage[margin=2cm]{geometry}
\begin{document}

\begin{minipage}{0.5\textwidth}
	Ministère de l'éducation nationale  \\
	Inspection académique de Kédougou   \\
	Cellule de mathématiques            \\
	M. BA\\
	Classe : Tle  \\
\end{minipage}
\begin{minipage}{0.5\textwidth}
	Année scolaire 2023-2024 \\
	Date : 10-06-2024 \\
	Durée : 3h 00 \\
\end{minipage}

\begin{center}
	\textbf{{\underline{Composition N2 Du Second Semestre}}}
\end{center}
\section*{\underline{Exercice 1: }\textbf{6 pts}}
\subsection*{ Resoudre dans $\mathbb{R}$ 1pt+1pt+1,5pts+1pt+1,5pts}
\begin{itemize}
\item[a)] $\ln(2x-1)=\ln(x+1)$

\item[b)] $\ln(x-1)+\ln(x+1)=\ln(x+2)$

\item[c)] $\ln(2x-1)+2\ln(x+1)=\ln(x-1)$

\item[d)] $\ln(x-1)\leq\ln(3-x)$

\item[e)] $\ln(1-x)-\ln(2x+3)\geq\ln(x-1)$
\end{itemize}
\section*{\underline{Exercice 2: }\textbf{6 pts}}
\subsection*{ 1) Développer } $(x+1)(x-3)(x+2)$
\subsection*{ 2) Résoudre } $e^{3x}-7e^{x}-6=0$
\subsection*{ 3) Résoudre } $x^{4}-5x^{2}+6=0$ puis $e^{4x}-5e^{2x}+6=0$
\subsection*{ 4) Développer } $(3+x)(2x-1)$ et $(x-2)(3+x)(2x-1)$
\subsection*{ 5) Résoudre } $2e^{-2x}+5e^{-x}-3=0$ et $2e^{3x+1}+e^{2x+1}-13e^{x+1}+6e=0$
\subsection*{ 6) Résoudre dans $\mathbb{R}^{2}$} 
\( \begin{cases}
x + y = 2 \\
\ln x + \ln y = 0
\end{cases}\)

\section*{\underline{Problème: }\textbf{8 pts}}
Soit $f(x)=\ln(x^{2}-6x+9)$
\begin{itemize}
\item[1)a-] Montrer que l'esemble de définition de $f$ est $Df=\mathbb{R}\setminus\left\lbrace 3 \right\rbrace $ et détermine les limites aux bornes de $Df$.$\textbf{0,5pt+1pt}$

\item[b-] Etuider les variations de $f$.$\textbf{1,5pt}$
\end{itemize}

2)Soit la courbe (Cf) représentative de $f$ dans un repère orthonormé (unité 1 cm).
\begin{itemize}
\item[a-]Déterminer les points d'intersections de $Cf$ avec les axes du repère.$\textbf{1pt}$

\item[b-]Ecrire une équation de la tangente (T) à (Cf) au point d'abscisse 0.$\textbf{0,5pt}$

\item[c-]Montrer que la droite d'équation $x=3$ est axe de  symétrie de (Cf). $\textbf{1pt}$

\item[d-]Tracer (Cf) et la tangente (T). $\textbf{1,5pt}$

\end{itemize}
3) Montrer que $f(x)=2\ln(x-3)$ sur $ \left]3 +\infty \right[ $. $\textbf{1pt}$

\end{document}
