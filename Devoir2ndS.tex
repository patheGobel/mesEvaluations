\documentclass{article}
\usepackage{amsmath}
\usepackage{amssymb}
\usepackage[margin=2cm]{geometry}

\begin{document}

\begin{minipage}{0.5\textwidth}
	Ministère de l'éducation nationale  \\
	Inspection académique de Kédougou   \\
	Cellule de mathématiques            \\
	Classe : 2nS  \\
\end{minipage}
\begin{minipage}{0.5\textwidth}
	Année scolaire 2023-2024 \\
	Date : 11-05-2024 \\
	Durée : 4h 00 \\
\end{minipage}

\begin{center}
	\textbf{{\underline{Composition Du Second Semestre}}}
\end{center}
\section*{EXERCICE 1}
Le plan est muni d’ un repère ( O , i ,j) .

1°) Trouver le réel a pour que les deux relations suivantes soient des équations cartésiennes de la même droite
\\ ( D) : 

$ax + 3y-4 =0$   ;  $5x+\frac{15}{2}y-10 = 0$

2°) Déterminer le vecteur directeur de ( D)  dont la première coordonnée est -6.

3°) Déterminer un système d’équations paramétriques de ( D) .

4°) Quel est le paramètre du point A dont les coordonnées (x, y ) vérifient x  = y.

\section*{EXERCICE 1}
Soient ABC un triangle ;  E  et F   les milieux des cotés [ BC ] et [ AC].                                                   

Soit  G  le point tel que $\overrightarrow{AG}=\frac{1}{3}\overrightarrow{AB}$  . Les droites (AE) et (FG) se coupent en M.

Le but de l’exercice est de calculer le réel t tel que $\overrightarrow{AM}=t\overrightarrow{AE}$

On muni le plan du repère $( A ;\overrightarrow{AB};\overrightarrow{AC} )$

1°) Calculer les coordonnées des points E , F  et  G.

2°) Déterminer des équations cartésiennes des droites (FG ) et  (AE).

3°) En déduire les coordonnées du point M.

4°) Déterminer alors t .
\end{document}