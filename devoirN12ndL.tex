\documentclass[12pt,a4paper]{article}
\usepackage{amsmath,amssymb,mathrsfs,tikz,times,pifont}
\usepackage{enumitem}
\newcommand\circitem[1]{%
\tikz[baseline=(char.base)]{
\node[circle,draw=gray, fill=red!55,
minimum size=1.2em,inner sep=0] (char) {#1};}}
\newcommand\boxitem[1]{%
\tikz[baseline=(char.base)]{
\node[fill=cyan,
minimum size=1.2em,inner sep=0] (char) {#1};}}
\setlist[enumerate,1]{label=\protect\circitem{\arabic*}}
\setlist[enumerate,2]{label=\protect\boxitem{\alph*}}
%%%::::::by chnini ameur :::::::%%%
\everymath{\displaystyle}
\usepackage[left=1cm,right=1cm,top=1cm,bottom=1.7cm]{geometry}
\usepackage{array,multirow}
\usepackage[most]{tcolorbox}
\usepackage{varwidth}
\tcbuselibrary{skins,hooks}
\usetikzlibrary{patterns}
%%%::::::by chnini ameur :::::::%%%
\newtcolorbox{exa}[2][]{enhanced,breakable,before skip=2mm,after skip=5mm,
colback=yellow!20!white,colframe=black!20!blue,boxrule=0.5mm,
attach boxed title to top left ={xshift=0.6cm,yshift*=1mm-\tcboxedtitleheight},
fonttitle=\bfseries,
title={#2},#1,
% varwidth boxed title*=-3cm,
boxed title style={frame code={
\path[fill=tcbcolback!30!black]
([yshift=-1mm,xshift=-1mm]frame.north west)
arc[start angle=0,end angle=180,radius=1mm]
([yshift=-1mm,xshift=1mm]frame.north east)
arc[start angle=180,end angle=0,radius=1mm];
\path[left color=tcbcolback!60!black,right color = tcbcolback!60!black,
middle color = tcbcolback!80!black]
([xshift=-2mm]frame.north west) -- ([xshift=2mm]frame.north east)
[rounded corners=1mm]-- ([xshift=1mm,yshift=-1mm]frame.north east)
-- (frame.south east) -- (frame.south west)
-- ([xshift=-1mm,yshift=-1mm]frame.north west)
[sharp corners]-- cycle;
},interior engine=empty,
},interior style={top color=yellow!5}}
%%%%%%%%%%%%%%%%%%%%%%%

\usepackage{fancyhdr}
\usepackage{eso-pic}         % Pour ajouter des éléments en arrière-plan
% Commande pour ajouter du texte en arrière-plan
\AddToShipoutPicture{
    \AtTextCenter{%
        \makebox[0pt]{\rotatebox{80}{\textcolor[gray]{0.7}{\fontsize{5cm}{5cm}\selectfont PGB}}}
    }
}
\usepackage{lastpage}
\fancyhf{}
\pagestyle{fancy}
\renewcommand{\footrulewidth}{1pt}
\renewcommand{\headrulewidth}{0pt}
\renewcommand{\footruleskip}{10pt}
\fancyfoot[R]{
\color{blue}\ding{45}\ \textbf{2024}
}
\fancyfoot[L]{
\color{blue}\ding{45}\ \textbf{Prof:M. BA}
}
\cfoot{\bf
\thepage /
\pageref{LastPage}}
\begin{document}
\renewcommand{\arraystretch}{1.5}
\renewcommand{\arrayrulewidth}{1.2pt}
\begin{tikzpicture}[overlay,remember picture]
\node[draw=blue,line width=1.2pt,fill=purple,text=blue,inner sep=3mm,rounded corners,pattern=dots]at ([yshift=-2.5cm]current page.north) {\begingroup\setlength{\fboxsep}{0pt}\colorbox{white}{\begin{tabular}{|*1{>{\centering \arraybackslash}p{0.28\textwidth}} |*2{>{\centering \arraybackslash}p{0.2\textwidth}|} *1{>{\centering \arraybackslash}p{0.19\textwidth}|} }
\hline
\multicolumn{3}{|c|}{$\diamond$$\diamond$$\diamond$\ \textbf{Lycée de Dindéfélo}\ $\diamond$$\diamond$$\diamond$ }& \textbf{A.S. : 2024/2025} \\ \hline
\textbf{Matière: Mathématiques}& \textbf{Niveau : 2nd}\textbf{L} &\textbf{Date: 13/12/2024} & \textbf{Durée : 2 heures} \\ \hline
\multicolumn{4}{|c|}{\parbox[c]{10cm}{\begin{center}
\textbf{{\Large\sffamily Devoir n$ ^{\circ} $ 1 Du 1$ ^\text{\bf er} $ Semestre}}
\end{center}}} \\ \hline
\end{tabular}}\endgroup};
\end{tikzpicture}
\vspace{3cm}

\section*{\underline{Exercice 1 :} 9 points}
Ecrire les réels suivants sous formes de fractions irréductibles :
\[
\textbf{A}=\frac{5}{2}-\frac{1}{\frac{2}{3}} \quad\quad;\quad\quad \textbf{B}=\frac{2}{3}-\frac{4}{5}+\frac{3}{7} \quad\quad;\quad\quad \textbf{C}=\left(3+\frac{4}{5} \right) \left(\frac{1}{5}-4 \right) 
\]

\[
\textbf{D}=\left(3-\frac{2}{3} \right) + \left(-\frac{3}{2}+3 \right) \quad\quad;\quad\quad \textbf{E}=\frac{\frac{1}{4}+3}{\frac{3}{5}} \times \frac{7}{3} \quad\quad;\quad\quad
\textbf{F}= \frac{\frac{1}{2}+\frac{2}{3}-1}{\frac{3}{4}}
\]

\[
 \textbf{G}= \left(\frac{5}{3}+1 \right) \div \left(\frac{4}{3}-\frac{1}{4} \right)\quad\quad;\quad\quad
 \textbf{H}=\frac{\frac{3}{2}-\frac{5}{3}}{-\frac{7}{2}+\frac{1}{6}} \times \frac{1}{-3+\frac{2}{5}}
 \quad\quad;\quad\quad \textbf{I}=\left(\frac{2}{3}-1 \right)\times \frac{2+\frac{1}{3}}{\frac{3}{5}+\frac{1}{5}}
\]
\section*{\underline{Exercice 2 :} 4 points}
Compléter dans chaque cas
\begin{itemize}
    \item \textbf{Cube d'une somme :} \quad $(a + b)^3 = ...$
    \item \textbf{Cube d'une différence :} \quad $(a - b)^3 = ...$
    \item \textbf{Somme des cubes :} \quad $a^3 + b^3 = ...$
    \item \textbf{Différence des cubes :} \quad $a^3 - b^3 = ...$
\end{itemize}
\section*{\underline{Exercice 3 :} 6 points}
1) Développer et réduire les expressions données :
\[
C = (2x - 3)(4x^2 + 6x + 9)
\]
\[
E = (-2x + 3)^3 - 2x(2x^2 - 5x + 3)
\]
\[
F = (x - 2)^2(-2x^2 + x + 1)
\]

2) Factoriser au mieux :

\[
F = 4x^2 + 20x + 25 - 9(4x^2 - 25)
\]
\[
G = 27 - 8x^3
\]
\[
I = x^3 - 8 + (x - 2)^2
\]
\end{document}
