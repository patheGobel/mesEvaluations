\documentclass{article}
\usepackage{amsmath}
\usepackage{amssymb}
\usepackage[margin=2cm]{geometry}

\begin{document}

\begin{minipage}{0.5\textwidth}
	Ministère de l'éducation nationale  \\
	Inspection académique de Kédougou   \\
	Lycée de Dindéferlo            \\
	Cellule de mathématiques            \\
	M. BA                          \\
	Classe : $2^{nd}S$  \\
\end{minipage}
\begin{minipage}{0.5\textwidth}
	Année scolaire 2023-2024 \\
	Date : 25-04-2024 \\
	Durée : 2h 00 \\
\end{minipage}

\begin{center}
	\textbf{{\underline{Devoir N1 Du Second Semestre}}}
\end{center}

\section*{Exercice 1 (10 points) :}
\subsection*{1) Résoudre dans $\mathbb{R}$ les équations suivantes  : (1 pt $\times$ 5)}
$(E_{1}):x^{2}-2|x|-3 = 0$\\
$(E_{2}):x^{4} + x^{2} = 6$\\
$(E_{3}):3x-5\sqrt{x}-2 = 0$\\
$(E_{4}):(x+\frac{1}{x})^{2}-3(x+\frac{1}{x})+2=0$\\
$(E_{5}):4x^{2}-4\sqrt{2+3\sqrt{2}}x+2+3\sqrt{2}=0$
\subsection*{1) Somme et Produit : (1 pt $\times$ 5)}
Soit le trinôme du second degré $p(x)=x^{2}-5x+2$

Sans calculer les racines $x_{1}$ et $x_{2}$, calculer $x_{1}+x_{2}$ ; $x_{1}\times x_{2}$ ; 
$x_{1}^{2}+x_{2}^{2}$ ; $\frac{1}{x_{1}}+\frac{1}{x_{2}}$ ; $x_{1}^{3}+x_{2}^{3}$
\section*{Exercice 2 (5 points) :}
Soit $ABCD$ un rectangle tel que $AB = 5$ et $BC = 9$

1. Construire $G$ barycentre des points pondérés
$(A;1), (B;1), (C;1) et (D;2)$
Faire un dessin précis 

2. Soit $I$ le milieu de $[BC]$ et $K$ le point vérifiant $\overrightarrow{AK}=\frac{2}{3}\overrightarrow{AD}$

Montrer que les point s $I$, $J$ et $K$ sont alignés.
\section*{Exercice 3 (5 points) :}
1) Soit $G=\mathrm{Bar}\lbrace (A,2),(B,5)\rbrace$\\
Montrer que $\overrightarrow{AG}=\frac{5}{7}\overrightarrow{AB}$ puis construire le point $G$

2) Déterminer dans chaque cas les réels $\alpha$ et $\beta$ pour que $G$ soit le \\barycentre des points pondérés $(A,\alpha)$ et $(B,\beta)$\\
a) $\overrightarrow{AB}=-\frac{2}{5}\overrightarrow{GB}$.\\
b) $3\overrightarrow{AG}=2\overrightarrow{BA}$.\\
\end{document}