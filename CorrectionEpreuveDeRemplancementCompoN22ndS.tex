\documentclass[12pt]{article}
\usepackage{stmaryrd}
\usepackage{graphicx}
\usepackage[utf8]{inputenc}

\usepackage[french]{babel}
\usepackage[T1]{fontenc}
\usepackage{hyperref}
\usepackage{verbatim}

\usepackage{color, soul}

\usepackage{pgfplots}
\pgfplotsset{compat=1.15}
\usepackage{mathrsfs}

\usepackage{amsmath}
\usepackage{amsfonts}
\usepackage{amssymb}
\usepackage{tkz-tab}

\usepackage{tikz}
\usetikzlibrary{arrows, shapes.geometric, fit}


\usepackage[margin=2cm]{geometry}
\begin{document}

\begin{minipage}{0.5\textwidth}
	Ministère de l'éducation nationale  \\
	Inspection académique de Kédougou   \\
	Lycée de Dindéferlo            \\
	Cellule de mathématiques            \\
	M. BA                          \\
	Classe : $2^{nd}$S  \\
\end{minipage}
\begin{minipage}{0.5\textwidth}
	Année scolaire 2023-2024 \\
	Date : 24-06-2024 \\
	Durée : 4h 00 \\
\end{minipage}

\begin{center}
	\textbf{{\underline{\textcolor{green}{Correction du Devoir N2 Du Second Semestre}}}}
\end{center}

\section*{Exercice 1 (7 points) :}
\begin{itemize}
\item[1.] Dans chacun des cas suivants, vérifier que $\alpha$ est une racine de $P$, puis déterminer $Q(x)$ tel que $P(x)=(x-\alpha)Q(x)$ :
\begin{itemize}
\item[a.] $P(x)=2x^{2}-(1+2\sqrt{3})x-1-\sqrt{3},\quad \alpha=-\frac{1}{2}$ \textbf{1 pt}

\item[b.] $P(x)=4x^{3}+x^{2}-11x+6,\quad \alpha=1\quad;\quad \alpha=-2$ \textbf{1,5 pts}
\end{itemize}
\item[2.] Soit $P(x)=2x^{3}-7x^{2}-17x+10$
\begin{itemize}
\item[a.] Vérifier que $\alpha=-2$ est une racine de $P$, puis déterminer $Q(x)$ tel que $P(x)=(x-\alpha)Q(x)$ \textbf{1 pt}

\item[b.] Donner une factorisation complète de $P(x)$ \textbf{1 pt}
\item[c.] Résoudre l'équation $P(x)=0$ \textbf{1 pt}
\item[d.] Résoudre l'inéquation $P(x)>0$ \textbf{1 pt}
\end{itemize}
\end{itemize}
\section*{\underline{\textcolor{green}{Correction Exercice 2: \textbf{7 pts}}}}
\begin{itemize}
\item[1.] Dans chacun des cas suivants, vérifions que $\alpha$ est une racine de $P$, puis déterminons $Q(x)$ tel que $P(x)=(x-\alpha)Q(x)$ :
\begin{itemize}
\item[a.] \textcolor{green}{$P(x)=2x^{2}-(1+2\sqrt{3})x-1-\sqrt{3},\quad \alpha=-\frac{1}{2}$} \textbf{1 pt}

$\alpha=-\frac{1}{2}$ est une racine de $P(x)$ ssi $P(\alpha)=0$

Ef effet, 
\begin{align*}
P(x)&=2(-\frac{1}{2})^{2}-(1+2\sqrt{3})(-\frac{1}{2})-1-\sqrt{3}\\
	&=\frac{1}{2}+\frac{1}{2}+\sqrt{3}-1-\sqrt{3}\\
	&=0
\end{align*}
Donc $P(\alpha)=0$ d'où $\alpha$ est racine de $P(x)$

Déterminons $Q(x)$

Par Hörner,

\definecolor{ududff}{rgb}{0.30196078431372547,0.30196078431372547,1}
\definecolor{cqcqcq}{rgb}{0.7529411764705882,0.7529411764705882,0.7529411764705882}
\begin{tikzpicture}[line cap=round,line join=round,>=triangle 45,x=1cm,y=1cm]
%\draw [color=cqcqcq,, xstep=1cm,ystep=1cm] (-9.37,-3.73) grid (14.73,8.29);\clip(-9.37,-3.73) rectangle (14.73,8.29);
\draw [line width=2pt] (-8,4)-- (1,4);
\draw [line width=2pt] (-9,3)-- (1,3);
\draw [line width=2pt] (-9,2)-- (1,2);
\draw [line width=2pt] (-8,1)-- (1,1);
\draw [line width=2pt] (-8,4)-- (-8,1);
\draw [line width=2pt] (-5,4)-- (-5,1);
\draw [line width=2pt] (-2,4)-- (-2,1);
\draw [line width=2pt] (1,4)-- (1,1);
\draw [line width=2pt] (-9,3)-- (-9,2);
\begin{scriptsize}
\draw [fill=ududff] (-6.59,3.57) node {$2$};
\draw [fill=ududff] (-3.59,3.59) node {$-1-2\sqrt{3}$};
\draw [fill=ududff] (-0.63,3.61) node {$-1-\sqrt{3}$};
\draw [fill=ududff] (-8.63,2.41) node {$-\frac{1}{2}$};
\draw [fill=ududff] (-3.61,2.51) node {$-1$};
\draw [fill=ududff] (-0.69,2.59) node {$1+\sqrt{3}$};
\draw [fill=ududff] (-6.71,1.57) node {$2$};
\draw [fill=ududff] (-3.63,1.53) node {$-2-2\sqrt{3}$};
\draw [fill=ududff] (-0.77,1.53) node {$0$};
\end{scriptsize}
\end{tikzpicture}

Donc \underline{\textcolor{green}{$Q(x)=(2x-2-2\sqrt{3})$}}

Donc la factorisation de $\textcolor{green}{\boxed{P(x)=2(x+\frac{1}{2})(x-1-\sqrt{3})}}$
\item[b.] $P(x)=4x^{3}+x^{2}-11x+6,\quad \alpha=1\quad;\quad \alpha=-2$ \textbf{1,5 pts}
\end{itemize}
\item[2.] Soit $P(x)=2x^{3}-7x^{2}-17x+10$
\begin{itemize}
\item[a.] Vérifier que $\alpha=-2$ est une racine de $P$, puis déterminer $Q(x)$ tel que $P(x)=(x-\alpha)Q(x)$ \textbf{1 pt}

\item[b.] Donner une factorisation complète de $P(x)$ \textbf{1 pt}
\item[c.] Résoudre l'équation $P(x)=0$ \textbf{1 pt}
\item[d.] Résoudre l'inéquation $P(x)>0$ \textbf{1 pt}
\end{itemize}
\end{itemize}
\section*{Exercice 2 (6 points) :}
1) Déterminer les valeurs du paramètre réel \( m \) pour lesquelles l'équation 

(E1):\( (m - 1)x^2 + 2mx + m - 2 = 0 \) 

admet deux solutions distinctes non nulles de signes contraires. \textbf{2pts}

2) Pour quelles valeurs du paramètre réel \( m \), l'équation

(E2):\( x^2 - 2(m + 1)x - 2m - 3 = 0 \) 
 
admet-elle deux racines distinctes strictement positives ? \textbf{2pts}

3) Pour quelles valeurs du paramètre \( m \), l'équation
 
(E3):\( m^{2}x^2 - 2m(m + 1)x + 2m + 1 = 0 \) 

admet-elle deux solutions distinctes strictement négatives ? \textbf{2pts}
\section*{\underline{\textcolor{green}{Correction Exercice 2: \textbf{6 pts}}}}
\section*{Exercice 3 (7 points) :}
Soit $A\begin{pmatrix} 1 \\ 2\end{pmatrix}\;,\quad B\begin{pmatrix} 3 \\ 4\end{pmatrix}$ deux points du plan et $(D_{1})\ :\ 2x-3y+5=0.$

1) Déterminer un système d'équations paramétriques des droites $(AB)$ et $(D_{1}).$\textbf{2pts}

2) Soit $(D_{2})\ :\ \left\lbrace\begin{array}{rcl} x&=&2-t \\ y&=&3+4t\end{array}\right.$

a) $A\begin{pmatrix} 1 \\ 2\end{pmatrix}$ et $D\begin{pmatrix} 0 \\ 11\end{pmatrix}$ appartiennent-ils à $(D_{2})\;\ ?$ \textbf{1pts=0,5pt+0,5pt}

b) Déterminer l'équation cartésienne de $(D_{2})$. \textbf{2pts}

3) Déterminer les coordonnées de point d'intersection de $(D_{1})$ et $(D_{2})$. \textbf{2pts}
\section*{\underline{\textcolor{green}{Correction Exercice 3: \textbf{7 pts}}}}
\end{document}
