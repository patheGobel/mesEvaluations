\documentclass[12pt]{article}
\usepackage{stmaryrd}
\usepackage{graphicx}
\usepackage[utf8]{inputenc}

\usepackage[french]{babel}
\usepackage[T1]{fontenc}
\usepackage{hyperref}
\usepackage{verbatim}

\usepackage{color, soul}

\usepackage{pgfplots}
\pgfplotsset{compat=1.15}
\usepackage{mathrsfs}

\usepackage{amsmath}
\usepackage{amsfonts}
\usepackage{amssymb}
\usepackage{tkz-tab}

\usepackage{tikz}
\usetikzlibrary{arrows, shapes.geometric, fit}


\usepackage[margin=2cm]{geometry}
\begin{document}

\begin{minipage}{0.5\textwidth}
	Ministère de l'éducation nationale  \\
	Inspection académique de Kédougou   \\
	Lycée de Dindéferlo            \\
	Cellule de mathématiques            \\
	M. BA                          \\
	Classe : $2^{nd}$S  \\
\end{minipage}
\begin{minipage}{0.5\textwidth}
	Année scolaire 2023-2024 \\
	Date : 24-06-2024 \\
	Durée : 4h 00 \\
\end{minipage}

\begin{center}
	\textbf{{\underline{\textcolor{green}{Correction du Devoir N2 Du Second Semestre}}}}
\end{center}

\section*{Exercice 1 (7 points) :}
\begin{itemize}
\item[1.] Dans chacun des cas suivants, vérifier que $\alpha$ est une racine de $P$, puis déterminer $Q(x)$ tel que $P(x)=(x-\alpha)Q(x)$ :
\begin{itemize}
\item[a.] $P(x)=2x^{2}-(1+2\sqrt{3})x-1-\sqrt{3},\quad \alpha=-\frac{1}{2}$ \textbf{1 pt}

\item[b.] $P(x)=4x^{3}+x^{2}-11x+6,\quad \alpha=1\quad;\quad \alpha=-2$ \textbf{1,5 pts}
\end{itemize}
\item[2.] Soit $P(x)=2x^{3}-7x^{2}-17x+10$
\begin{itemize}
\item[a.] Vérifier que $\alpha=-2$ est une racine de $P$, puis déterminer $Q(x)$ tel que $P(x)=(x-\alpha)Q(x)$ \textbf{1 pt}

\item[b.] Donner une factorisation complète de $P(x)$ \textbf{1 pt}
\item[c.] Résoudre l'équation $P(x)=0$ \textbf{1 pt}
\item[d.] Résoudre l'inéquation $P(x)>0$ \textbf{1 pt}
\end{itemize}
\end{itemize}
\section*{\underline{\textcolor{green}{Correction Exercice 2: \textbf{7 pts}}}}
\begin{itemize}
\item[1.] Dans chacun des cas suivants, vérifions que $\alpha$ est une racine de $P$, puis déterminons $Q(x)$ tel que $P(x)=(x-\alpha)Q(x)$ :
\begin{itemize}
\item[a.] \textcolor{green}{$P(x)=2x^{2}-(1+2\sqrt{3})x-1-\sqrt{3},\quad \alpha=-\frac{1}{2}$} \textbf{1 pt}

$\alpha=-\frac{1}{2}$ est une racine de $P(x)$ ssi $P(\alpha)=0$

Ef effet, 
\begin{align*}
P(-\frac{1}{2})&=2(-\frac{1}{2})^{2}-(1+2\sqrt{3})(-\frac{1}{2})-1-\sqrt{3}\\
	&=\frac{1}{2}+\frac{1}{2}+\sqrt{3}-1-\sqrt{3}\\
	&=0
\end{align*}
Donc $P(\alpha)=0$ d'où $\alpha$ est racine de $P(x)$

Déterminons $Q(x)$

Par Hörner,

\definecolor{ududff}{rgb}{0.30196078431372547,0.30196078431372547,1}
\definecolor{cqcqcq}{rgb}{0.7529411764705882,0.7529411764705882,0.7529411764705882}
\begin{tikzpicture}[line cap=round,line join=round,>=triangle 45,x=1cm,y=1cm]
%\draw [color=cqcqcq,, xstep=1cm,ystep=1cm] (-9.37,-3.73) grid (14.73,8.29);\clip(-9.37,-3.73) rectangle (14.73,8.29);
\draw [line width=2pt] (-8,4)-- (1,4);
\draw [line width=2pt] (-9,3)-- (1,3);
\draw [line width=2pt] (-9,2)-- (1,2);
\draw [line width=2pt] (-8,1)-- (1,1);
\draw [line width=2pt] (-8,4)-- (-8,1);
\draw [line width=2pt] (-5,4)-- (-5,1);
\draw [line width=2pt] (-2,4)-- (-2,1);
\draw [line width=2pt] (1,4)-- (1,1);
\draw [line width=2pt] (-9,3)-- (-9,2);
\begin{scriptsize}
\draw [fill=ududff] (-6.59,3.57) node {$2$};
\draw [fill=ududff] (-3.59,3.59) node {$-1-2\sqrt{3}$};
\draw [fill=ududff] (-0.63,3.61) node {$-1-\sqrt{3}$};
\draw [fill=ududff] (-8.63,2.41) node {$-\frac{1}{2}$};
\draw [fill=ududff] (-3.61,2.51) node {$-1$};
\draw [fill=ududff] (-0.69,2.59) node {$1+\sqrt{3}$};
\draw [fill=ududff] (-6.71,1.57) node {$2$};
\draw [fill=ududff] (-3.63,1.53) node {$-2-2\sqrt{3}$};
\draw [fill=ududff] (-0.77,1.53) node {$0$};
\end{scriptsize}
\end{tikzpicture}

Donc \underline{\textcolor{green}{$Q(x)=(2x-2-2\sqrt{3})$}}

Donc la factorisation de $\textcolor{green}{\boxed{P(x)=2(x+\frac{1}{2})(x-1-\sqrt{3})}}$

\item[b.] $P(x)=4x^{3}+x^{2}-11x+6,\quad \alpha=1\quad;\quad \alpha=-2$ \textbf{1,5 pts}

$\alpha=1$ est une racine de $P(x)$ ssi $P(1)=0$

Ef effet, 
\begin{align*}
P(1)&=4(1)^{3}+(1)^{2}-11(1)+6\\
	&=4+1-11+6\\
	&=0
\end{align*}
\textcolor{green}{Donc $P(1)=0$ d'où 1 est racine de $P(x)$}

$\alpha=-2$ est une racine de $P(x)$ ssi $P(\alpha)=0$

Ef effet, 
\begin{align*}
P(-2)&=4(-2)^{3}+(-2)^{2}-11(-2)+6\\
	&=-4\times 8+4+22+6\\
	&=0
\end{align*}
\textcolor{green}{Donc $P(-2)=0$ d'où -2 est racine de $P(x)$}

Déterminons $Q(x)$

Par double Hörner,

\definecolor{ududff}{rgb}{0.30196078431372547,0.30196078431372547,1}
\definecolor{cqcqcq}{rgb}{0.7529411764705882,0.7529411764705882,0.7529411764705882}
\begin{tikzpicture}[line cap=round,line join=round,>=triangle 45,x=1cm,y=1cm]
%\draw [color=cqcqcq,, xstep=1cm,ystep=1cm] (-9.78,-6.01) grid (11.5,6.01);
%\clip(-9.78,-6.01) rectangle (11.5,6.01);
\draw [line width=2pt] (-8,3)-- (7,3);
\draw [line width=2pt] (-9,2)-- (7,2);
\draw [line width=2pt] (-9,1)-- (7,1);
\draw [line width=2pt] (-9,0)-- (7,0);
\draw [line width=2pt] (-9,-1)-- (3,-1);
\draw [line width=2pt] (-8,3)-- (-8,-2);
\draw [line width=2pt] (-8,-2)-- (3,-2);
\draw [line width=2pt] (7,3)-- (7,0);
\draw [line width=2pt] (-9,0)-- (-9,-1);
\draw [line width=2pt] (-9,2)-- (-9,1);
\draw [line width=2pt] (-5,3)-- (-5,-2);
\draw [line width=2pt] (-1,3)-- (-1,-2);
\draw [line width=2pt] (3,3)-- (3,-2);
\begin{scriptsize}
\draw [fill=ududff] (-6.54,2.53) node {$4$};
\draw [fill=ududff] (-3.02,2.55) node {$1$}; 
\draw [fill=ududff] (0.98,2.53) node {$-11$};
\draw [fill=ududff] (5,2.59) node {$6$}; 
\draw [fill=ududff] (-8.58,1.49) node {$1$};
\draw [fill=ududff] (-3.06,1.51) node {$4$};
\draw [fill=ududff] (0.96,1.51) node {$5$};
\draw [fill=ududff] (4.94,1.55) node {$6$};
\draw [fill=ududff] (-6.72,0.47) node {$4$};
\draw [fill=ududff] (-3.06,0.47) node {$5$}; 
\draw [fill=ududff] (0.96,0.51) node {$-6$};
\draw [fill=ududff] (4.98,0.57) node {$0$}; 
\draw [fill=ududff] (-8.56,-0.55) node {$-2$};
\draw [fill=ududff] (-3.02,-0.49) node {$-8$}; 
\draw [fill=ududff] (0.92,-0.47) node {$6$};
\draw [fill=ududff] (-6.76,-1.51) node {$4$}; 
\draw [fill=ududff] (-3.02,-1.55) node {$-3$};
\draw [fill=ududff] (0.96,-1.41) node {$0$};
\end{scriptsize}
\end{tikzpicture}

Donc \underline{\textcolor{green}{$Q(x)=4x-3$}}

Donc la factorisation de $\textcolor{green}{\boxed{P(x)=(x-1)(x+2)(4x-3)}}$

\end{itemize}
\item[2.] Soit $P(x)=2x^{3}-7x^{2}-17x+10$
\begin{itemize}
\item[a.] Vérifions que $\alpha=-2$ est une racine de $P$, puis déterminons $Q(x)$ tel que

 $P(x)=(x-\alpha)Q(x)$ \textbf{1 pt}

$\alpha=-2$ est une racine de $P(x)$ ssi $P(-2)=0$

Ef effet, 
\begin{align*}
P(-2)&=2(-2)^{3}-7(-2)^{2}-17(-2)+10\\
	&=-16-28+34+10\\
	&=0
\end{align*}
Donc $P(-2)=0$ d'où -2 est racine de $P(x)$

Déterminons $Q(x)$

Par Hörner,

\definecolor{ududff}{rgb}{0.30196078431372547,0.30196078431372547,1}
\definecolor{cqcqcq}{rgb}{0.7529411764705882,0.7529411764705882,0.7529411764705882}
\begin{tikzpicture}[line cap=round,line join=round,>=triangle 45,x=1cm,y=1cm]
%\draw [color=cqcqcq,, xstep=1cm,ystep=1cm] (-9.37,-3.73) grid (14.73,8.29);
%\clip(-9.37,-3.73) rectangle (14.73,8.29);
\draw [line width=2pt] (-8,4)-- (5,4);
\draw [line width=2pt] (-9,3)-- (5,3);
\draw [line width=2pt] (-9,2)-- (5,2);
\draw [line width=2pt] (-8,1)-- (5,1);
\draw [line width=2pt] (-8,4)-- (-8,1);
\draw [line width=2pt] (5,4)-- (5,1);
\draw [line width=2pt] (-5,4)-- (-5,1);
\draw [line width=2pt] (-2,4)-- (-2,1);
\draw [line width=2pt] (1,4)-- (1,1);
\draw [line width=2pt] (-9,3)-- (-9,2);
\begin{scriptsize}
\draw [fill=ududff] (-6.59,3.57) node {$2$};
\draw [fill=ududff] (-3.59,3.59) node {$-7$};
\draw [fill=ududff] (-0.63,3.61) node {$-17$};
\draw [fill=ududff] (2.95,3.61) node {$10$};
\draw [fill=ududff] (-8.63,2.41) node {$-2$};
\draw [fill=ududff] (-3.61,2.51) node {$-4$};
\draw [fill=ududff] (-0.69,2.59) node {$22$};
\draw [fill=ududff] (2.97,2.53) node {$-10$};
\draw [fill=ududff] (-6.71,1.57) node {$2$};
\draw [fill=ududff] (-3.63,1.53) node {$-11$};
\draw [fill=ududff] (-0.77,1.53) node {$5$};
\draw [fill=ududff] (2.99,1.49) node {$0$};
\end{scriptsize}
\end{tikzpicture}

Donc la factorisation de $\textcolor{green}{\boxed{Q(x)=2x^{2}-11x+5}}$

\item[b.] Donnons une factorisation complète de $P(x)$ \textbf{1 pt}

Pour ce faire, posons $Q(x)=0$

$Q(x)=0\implies 2x^{2}-11x+5=0$

$\Delta = 11^{2}-4\times 2\times 5=81$

$x_{1}=\frac{11-9}{4}$ ; $x_{2}=\frac{11+9}{4}$

$x_{1}=\frac{1}{2}$ ; $x_{2}=5$

$P(x)=2(x+2)(x-\frac{1}{2})(x-5)$

Donc la factorisation de $\textcolor{green}{\boxed{P(x)=2(x+2)(x-\frac{1}{2})(x-5)}}$
\item[c.] Résolvons l'équation $P(x)=0$ \textbf{1 pt}

$P(x)=0\implies 2(x+2)(x-\frac{1}{2})(x-5)=0\implies x=-2$ ou $x=\frac{1}{2}$ ou $x=5$

$\textcolor{green}{\boxed{S=\left\lbrace  -2, \frac{1}{2}, 5 \right\rbrace }}$

\item[d.] Résoudre l'inéquation $P(x)>0$ \textbf{1 pt}

\definecolor{xdxdff}{rgb}{0.49019607843137253,0.49019607843137253,1}
\definecolor{ududff}{rgb}{0.30196078431372547,0.30196078431372547,1}
\definecolor{cqcqcq}{rgb}{0.7529411764705882,0.7529411764705882,0.7529411764705882}
\begin{tikzpicture}[line cap=round,line join=round,>=triangle 45,x=1cm,y=1cm]
%\draw [color=cqcqcq,, xstep=1cm,ystep=1cm] (-9.78,-6.01) grid (11.5,6.01);
%\clip(-9.78,-6.01) rectangle (11.5,6.01);
\draw [line width=2pt] (-8,3)-- (4,3);
\draw [line width=2pt] (-8,2)-- (4,2);
\draw [line width=2pt] (-8,1)-- (4,1);
\draw [line width=2pt] (-8,0)-- (4,0);
\draw [line width=2pt] (-8,-1)-- (4,-1);
\draw [line width=2pt] (-8,-2)-- (4,-2);
\draw [line width=2pt] (-8,3)-- (-8,-2);
\draw [line width=2pt] (-5,3)-- (-5,-2);
\draw [line width=2pt] (-3,2)-- (-3,-2);
\draw [line width=2pt] (0,2)-- (0,-2);
\draw [line width=2pt] (2,2)-- (2,-2);
\draw [line width=2pt] (4,3)-- (4,-2);
\begin{scriptsize}
\draw [fill=ududff] (-6.54,2.51) node {$x$};
\draw [fill=ududff] (-4.82,2.41) node {$-\infty$};
\draw [fill=ududff] (-3.02,2.41) node {$-2$};
\draw [fill=ududff] (-0.04,2.45) node {$\frac{1}{2}$};
\draw [fill=ududff] (1.98,2.41) node {$5$};
\draw [fill=ududff] (3.4,2.41) node {$+\infty$};
\draw [fill=ududff] (-6.7,1.45) node {$x+2$};
\draw [fill=ududff] (-4,1.47) node {$-$};
\draw [fill=ududff] (-1.52,1.49) node {$+$};
\draw [fill=ududff] (0.94,1.51) node {$+$};
\draw [fill=ududff] (2.98,1.49) node {$+$};
\draw [fill=ududff] (-6.76,0.51) node {$x-\frac{1}{2}$};
\draw [fill=ududff] (-4.1,0.57) node {$-$};
\draw [fill=ududff] (-1.54,0.57) node {$-$};
\draw [fill=ududff] (0.98,0.59) node {$+$};
\draw [fill=ududff] (2.98,0.55) node {$+$};
\draw [fill=ududff] (-6.82,-0.51) node {$x-5$};
\draw [fill=ududff] (-3.98,-0.55) node {$-$};
\draw [fill=ududff] (-1.54,-0.45) node {$-$};
\draw [fill=ududff] (1.04,-0.39) node {$-$};
\draw [fill=ududff] (2.94,-0.39) node {$+$};
\draw [fill=ududff] (-6.84,-1.49) node {$P(x)$};
\draw [fill=ududff] (-4.04,-1.53) node {$-$};
\draw [fill=ududff] (-1.52,-1.55) node {$+$};
\draw [fill=ududff] (1.02,-1.55) node {$-$};
\draw [fill=ududff] (2.98,-1.55) node {$+$};
\draw [fill=xdxdff] (-3,1.51) node {\textbf{O}};
\draw [fill=xdxdff] (0,0.47) node {\textbf{O}};
\draw [fill=xdxdff] (2,-0.59) node {\textbf{O}};
\draw [fill=xdxdff] (-3,-1.49) node {\textbf{O}};
\draw [fill=xdxdff] (0,-1.49) node {\textbf{O}};
\draw [fill=xdxdff] (2,-1.49) node {\textbf{O}};
\end{scriptsize}
\end{tikzpicture}

$\textcolor{green}{\boxed{P(x)=\left] -2; \frac{1}{2}\right[\cup\left]5 +\infty\right[  }}$

\end{itemize}
\end{itemize}

\section*{Exercice 2 (6 points) :}
1) Déterminer les valeurs du paramètre réel \( m \) pour lesquelles l'équation 

(E1):\( (m - 1)x^2 + 2mx + m - 2 = 0 \) 

admet deux solutions distinctes non nulles de signes contraires. \textbf{2pts}

2) Pour quelles valeurs du paramètre réel \( m \), l'équation

(E2):\( x^2 - 2(m + 1)x - 2m - 3 = 0 \) 
 
admet-elle deux racines distinctes strictement positives ? \textbf{2pts}

3) Pour quelles valeurs du paramètre \( m \), l'équation
 
(E3):\( m^{2}x^2 - 2m(m + 1)x + 2m + 1 = 0 \) 

admet-elle deux solutions distinctes strictement négatives ? \textbf{2pts}
\section*{\underline{\textcolor{green}{Correction Exercice 2: \textbf{6 pts}}}}
1) Déterminons les valeurs du paramètre réel \( m \) pour lesquelles l'équation 

(E1):\( (m - 1)x^2 + 2mx + m - 2 = 0 \) 

admet deux solutions distinctes non nulles de signes contraires. \textbf{2pts}

\textcolor{green}{Cherchons $\Delta_{m}^{'}$}

$\Delta_{m}^{'}=(m)^{2}-(m - 1)(m - 2)$

$\Delta_{m}^{'}=3m-2$

\textcolor{green}{Etudions le signe $\Delta_{m}^{'}$}

Posons $\Delta_{m}^{'}=0\implies 3m-2=0\implies \textcolor{green}{m=\frac{2}{3}}$

\begin{itemize}
\item Si $m\in \left]-\infty; \frac{2}{3}\right[ $ alors $\Delta_{m}^{'}<0$
\item Si $m\in \left]\frac{2}{3}; +\infty\right[ $ alors $\Delta_{m}^{'}>0$
\end{itemize}

\textcolor{green}{Deux solutions distinctes non nulles de signes contraires}

Pour ce faire, Cherchons S et P

$S=-\frac{2m}{m-1}$

$S=0\implies \textcolor{green}{m=0}$ et $\textcolor{green}{m\neq 1}$

$P=\frac{m-2}{m-1}$

$P=0\implies \textcolor{green}{m\neq 2}$ et $\textcolor{green}{m\neq 1}$

\definecolor{cqcqcq}{rgb}{0.7529411764705882,0.7529411764705882,0.7529411764705882}
\begin{tikzpicture}[line cap=round,line join=round,>=triangle 45,x=1cm,y=1cm]
%\draw [color=cqcqcq,, xstep=1cm,ystep=1cm] (-7,-10) grid (-22,17);
\clip(-22,0) rectangle (12,10);
\draw [line width=2pt] (-23,8)-- (-7,8); %première ligne A(-22,8)---B(-7,8)
\draw [line width=2pt] (-22,6)-- (-7,6); %deuxième ligne
\draw [line width=2pt] (-22,4)-- (-7,4); %troisième ligne
\draw [line width=2pt] (-22,2)-- (-7,2); %quatième ligne
\draw [line width=2pt] (-22,0)-- (-7,0); %cinquième ligne
\draw [line width=2pt] (-22,0)-- (-22,8); %première colonne (-22,4)<----(-22,8);
\draw [line width=2pt] (-18,8)-- (-18,0); %deuxième colone  (-18,8)--->(-18,4);
\draw [line width=2pt] (-15,6)-- (-15,0); %troisième colonne (-13,6)--> (-13,4);

\draw [line width=2pt] (-13,6)-- (-13,0); %quatrième colonne (-13,6)-- (-13,4);

\draw [line width=2pt] (-11,6)-- (-11,0); %cinquième colonne (-13,6)-- (-13,4);
\draw [line width=2pt] (-11.1,6)-- (-11.1,0); %cinquième colonne (-13,6)-- (-13,4);
\draw [line width=2pt] (-9,6)-- (-9,0); %sixième colonne (-13,6)-- (-13,4);
\draw [line width=2pt] (-7,8)-- (-7,0); %Sepième colonne (-7,8)-->(-7,4);
\draw (-22,5.5) node[anchor=north west] {$\Delta_{m}=3m+2$};
\draw (-22,3) node[anchor=north west] {$S=\frac{-2m}{m-1}$}; 
\draw (-22,1.5) node[anchor=north west] {$P=\frac{m-2}{m-1}$};
\draw (-21,7) node[anchor=north west] {$m$};
\draw (-18,7) node[anchor=north west] {$-\infty$};
\draw (-8,7) node[anchor=north west] {$+\infty$};
%Signe de Delta
\draw (-16.8,5.3) node[anchor=north west] {$-$};
\draw (-14.5,5.3) node[anchor=north west] {$-$};
\draw (-12.5,5.3) node[anchor=north west] {$+$};
\draw (-10.5,5.3) node[anchor=north west] {$+$};
\draw (-8.5,5.3) node[anchor=north west] {$+$};
%Signe de S
\draw (-16.8,3.3) node[anchor=north west] {$-$};
\draw (-14.5,3.3) node[anchor=north west] {$+$};
\draw (-12.5,3.3) node[anchor=north west] {$+$};
\draw (-10.5,3.3) node[anchor=north west] {$-$};
\draw (-8.5,3.3) node[anchor=north west] {$-$};
%Signe de P
\draw (-16.8,1.3) node[anchor=north west] {$+$};
\draw (-14.5,1.3) node[anchor=north west] {$+$};
\draw (-12.5,1.3) node[anchor=north west] {$+$};
\draw (-10.5,1.3) node[anchor=north west] {$-$};
\draw (-8.5,1.3) node[anchor=north west] {$+$};
\draw (-13.2,7) node[anchor=north west] {$\frac{2}{3}$};
\draw (-11.5,7) node[anchor=north west] {$1$};
\draw (-15.2,7) node[anchor=north west] {$0$};
\draw (-9.5,7) node[anchor=north west] {$2$};
%++++++++++++++++++++++++++++++++++++++
\draw (-9.3,1.5) node[anchor=north west] {$O$};
\draw (-13.3,5.3) node[anchor=north west] {$O$};
\draw (-15.3,3.5) node[anchor=north west] {$O$};
\end{tikzpicture}

Pour que les deux solutions soient de signes contraires, il faut et il suffit que P<0

\textcolor{green}{Donc si $m\in\left]1 ; 2\right[ $ alors (E) admet deux solutions de signes contraires}

%2) Pour quelles valeurs du paramètre réel \( m \), l'équation

%(E2):\( x^2 - 2(m + 1)x - 2m - 3 = 0 \) 
 
%admet-elle deux racines distinctes strictement positives ? \textbf{2pts}

%3) Pour quelles valeurs du paramètre \( m \), l'équation
 
%(E3):\( m^{2}x^2 - 2m(m + 1)x + 2m + 1 = 0 \) 

%admet-elle deux solutions distinctes strictement négatives ? \textbf{2pts}
\section*{Exercice 3 (7 points) :}
Soit $A\begin{pmatrix} 1 \\ 2\end{pmatrix}\;,\quad B\begin{pmatrix} 3 \\ 4\end{pmatrix}$ deux points du plan et $(D_{1})\ :\ 2x-3y+5=0.$

1) Déterminer un système d'équations paramétriques des droites $(AB)$ et $(D_{1}).$\textbf{2pts}

2) Soit $(D_{2})\ :\ \left\lbrace\begin{array}{rcl} x&=&2-t \\ y&=&3+4t\end{array}\right.$

a) $A\begin{pmatrix} 1 \\ 2\end{pmatrix}$ et $D\begin{pmatrix} 0 \\ 11\end{pmatrix}$ appartiennent-ils à $(D_{2})\;\ ?$ \textbf{1pts=0,5pt+0,5pt}

b) Déterminer l'équation cartésienne de $(D_{2})$. \textbf{2pts}

3) Déterminer les coordonnées de point d'intersection de $(D_{1})$ et $(D_{2})$. \textbf{2pts}
\section*{\underline{\textcolor{green}{Correction Exercice 3: \textbf{7 pts}}}}
\end{document}
