\documentclass{article}
\usepackage{amsmath}
\usepackage{amssymb}
\usepackage[margin=2cm]{geometry}

\begin{document}

\begin{minipage}{0.7\textwidth}
	Ministère de l'éducation nationale  \\
	Inspection académique de Kédougou   \\
	Lycée de Dindéfélo            \\
	Cellule de mathématiques            \\
	M. BA                          \\
	Classe : 1erS  \\
\end{minipage}
\begin{minipage}{0.5\textwidth}
	Année scolaire 2023-2024 \\
	Date : 08-05-2024 \\
	Durée : 3h 00 \\
\end{minipage}

\begin{center}
	\textbf{{\underline{Devoir N2 Du Second Semestre}}}
\end{center}

\section*{\underline{Exercice 1} (9 points) :}
\subsection*{a) Calculer les limites suivantes (0,5 pt $\times$ 4)}
\[ \lim_{x \to 2}\frac{x^{2}-5x-14}{x^{2}+4x+4}\quad\quad 
\lim_{x \to 2}\frac{4x^{2}-3x-1}{x^{3}-4x^{2}+2x+1}\quad\quad 
\lim_{x \to 1}\frac{2x-\sqrt{x+1}-4}{x^{2}-2x-3}\quad\quad 
\lim_{x \to 0}x\mid x+\frac{1}{x}\mid\]
\subsection*{1) Dans chacun des cas suivants, calculer la limites en $+\infty$ et en $-\infty$ (1 pt $\times$ 4)}
\[ f(x)=\frac{\sqrt{3x^{2}+1}}{3x-1}\quad\quad 
g(x)=x+2+\sqrt{x^{2}-3x+1}\quad\quad 
h(x)=\sqrt{x^{2}-2x-1}-\sqrt{x^{2}-7x+3} \]
\[I(x)=\frac{x-\sqrt{x^{2}-3x+1}}{2x+\sqrt{4x^{2}+x}} \]
\subsection*{2) Etudier de continuité: (0,5pt +1pt)}
\[ f(x) = \begin{cases} 
  -x+1-\frac{2x}{x^{2}+1}, & \text{si } x \leq 1 \\
  \frac{x-1}{\sqrt{x^{2}-2x+6}}, & \text{si } x > 1
\end{cases} \]
a)Déterminer le domaine de définition $D_{f}$\\
b)Etudier la continuité de f en $x_{0}=1$
\subsection*{3) Prolongement par continuité: (0,5pt +1pt)}
Soit $h$ la fonction définie par:
\[ h(x) = \begin{cases} 
  \sqrt{x^{2}-4}, & \text{si } x > 2 \\
  \sqrt{x^{2}-9x+14}, & \text{si } x < 2
\end{cases} \]
a)Déterminer $D_{h}$.\\
b)La fonction h est-elle prolongeable par continuité en $2$ ?
\section*{\underline{Exercice 2}(3 points) :}
\subsection*{a) Dans chaque cas, justifier que f est dérivable sur I. (1,5 pts)}
\[f(x)=(2x^{2}-1)(x-3)^{4};\quad I=\mathbb{R}\quad\quad g(x)=\frac{-3}{x^{3}-1};\quad I=\mathbb{R}\setminus\left\lbrace 1 \right\rbrace \quad\quad I(x)=\sqrt{3x^{2}+x+7};\quad I=\mathbb{R}\]
\subsection*{b) Soit $f(x)=|x+1|$ }
Montrer que $f$ est continue en -1. Qu'en est-il de sa dérivable en -1 ? \textbf{(1,5 pts)}
\section*{\underline{Exercice 3}(5 points) :}
\subsection*{ Dans chacun des cas suivants, calculer la dérivé de $f$}
\[
f(x)=\sqrt{\frac{x+1}{x-1}}\quad\quad g(x)=(x-1)\sqrt{2-3x}\quad\quad h(x)=(\frac{x-1}{x-2})^{3}\quad I(x)=\frac{\sqrt{x}-1}{\sqrt{x}+1}\quad\quad J(x)=\frac{-1}{(2x+1^{2})(x+2)}
\]
\section*{\underline{Exercice 4} (3 points) :}
\subsection*{a) Cercle trigonométrique (0,5 pt $\times$ 3 )}
Contruire le cerlce trigonométrique en graduant l'axe des \textbf{cos} et des 
\textbf{sin} et en plaçant les angles remarquables.
%\[\cos(\frac{\pi}{6})=\cdots\quad\quad \cos(\frac{\pi}{4})=\cdots \quad\quad \cos(\frac{\pi}{3})=\cdots\quad\quad \cos(\frac{\pi}{2})=\cdots \]
\subsection*{a) Compléter les formules (0,5 pt $\times$ 3 )}
\[\cos(\pi-x)=\cdots\quad\quad \sin(\pi-x)=\cdots\quad\quad \cos(\frac{\pi}{2}+x)=\cdots\]
\end{document}