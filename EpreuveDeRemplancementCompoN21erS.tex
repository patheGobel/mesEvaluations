\documentclass[12pt]{article}
\usepackage{stmaryrd}
\usepackage{graphicx}
\usepackage[utf8]{inputenc}

\usepackage[french]{babel}
\usepackage[T1]{fontenc}
\usepackage{hyperref}
\usepackage{verbatim}

\usepackage{color, soul}

\usepackage{pgfplots}
\pgfplotsset{compat=1.15}
\usepackage{mathrsfs}

\usepackage{amsmath}
\usepackage{amsfonts}
\usepackage{amssymb}
\usepackage{tkz-tab}

\usepackage{tikz}
\usetikzlibrary{arrows, shapes.geometric, fit}


\usepackage[margin=2cm]{geometry}
\begin{document}

\begin{minipage}{0.5\textwidth}
	Ministère de l'éducation nationale  \\
	Inspection académique de Kédougou   \\
	Cellule de mathématiques            \\
	M. BA\\
	Classe : 1erS
\end{minipage}
\begin{minipage}{0.5\textwidth}
	Année scolaire 2023-2024 \\
	Date : 10-06-2024 \\
	Durée : 4h 00 \\
\end{minipage}

\begin{center}
	\textbf{{\underline{Composition N2 Du Second Semestre}}}
\end{center}
\section*{\underline{Exercice 1: }\textbf{10 pts}}
Dans chacun des cas suivants étudier les fonctions suivantes:

\[a)f(x)=\frac{7x-9}{4x-5}\quad\quad\textbf{5 pts}\]
\[b)f(x)=\frac{1-x}{2+3x}\quad\quad\textbf{5 pts}\]
\section*{\underline{Exercice 2: }\textbf{5 pts}}
Une urne contient 4 boules rouges, 3 boules vertes et une boule noire.

On tire au hasard 3 boules de cette urne.

\begin{itemize}
\item[1)] Calculer le nombre de tirages possibles. \textbf{1 pt}
\item[2)] Quel est le nombre de tirages contenant :
\begin{itemize}
\item[a)] 2 boules rouges. \textbf{1 pt}
\item[b)] 3 boules unicolores. \textbf{1 pt}
\item[c)] 3 boules de couleurs différentes. \textbf{1 pt}
\item[d)] une boule noire. \textbf{1 pt}
\end{itemize}
\end{itemize}

\section*{\underline{Exercice 3: }\textbf{5 pts}}
Soit $(u_n)$ une suite numérique définie par:

\[ u_n = 3n + 1 \]

1. Démontrer que $(u_n)$ est une suite arithmétique et déterminer sa raison et son premier terme. \textbf{1,5 pts}

2. Étudier les variations de la suite $(u_n)$. \textbf{1,5 pts}

3. Calculer la somme des 1000 premiers termes de la suite $(u_n)$. \textbf{2 pts}

\end{document}