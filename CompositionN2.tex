\documentclass{article}
\usepackage{amsmath}
\usepackage{amssymb}
\usepackage[margin=2cm]{geometry}

\begin{document}

\begin{minipage}{0.5\textwidth}
	Ministère de l'éducation nationale  \\
	Inspection académique de Kédougou   \\
	Cellule de mathématiques            \\
	Classe : 2nS  \\
\end{minipage}
\begin{minipage}{0.5\textwidth}
	Année scolaire 2023-2024 \\
	Date : 11-05-2024 \\
	Durée : 3h 00 \\
\end{minipage}

\begin{center}
	\textbf{{\underline{Devoir N1 Du Second Semestre}}}
\end{center}
\section*{EXERCICE 1 (7 points)}   
Soit l’équation (E): \((m -1)x^2 + 2(m + 2)x - m + 3 = 0\).

\begin{enumerate}
    \item 
    \begin{enumerate}
        \item Déterminer \(m\) pour que \(-1\) soit racine de (E).\textbf{(0,5 pt)}
        
        \item Déterminer alors l’autre racine. \textbf{(0,5 pt)}
    \end{enumerate}

    \item Déterminer l’ensemble des réels \(m\) tels que l’équation admette deux racines \(x'\) et \(x''\) vérifiant la condition donnée dans chacun des cas suivants :
    \begin{enumerate}
        \item \(x'\) et \(x''\) sont inverses ;\textbf{(1 pt)}
        
        \item \(x'\) et \(x''\) sont opposés ;\textbf{(1 pt)}
        
        \item \(x'\) et \(x''\) sont de signes opposés ;\textbf{(1 pt)}
        
        \item \(x'\) et \(x''\) sont de signe positif ;\textbf{(1 pt)}
        
        \item \(\frac{1}{2x'-1} + \frac{1}{2x''-1} = 6\).
    \end{enumerate}

    \item Former une relation entre les racines \(x'\) et \(x''\) indépendante de \(m\).\textbf{(1 pt)}

    \item Former une équation du second degré dont les racines sont \(X_1 = \frac{1}{x' - 2}\) et \(X_2 = \frac{1}{x'' - 2}\).\textbf{(1 pt)}
\end{enumerate}
%\section*{Exercice 1 (8 points) :}
%On donne l'équation (E) du second degré : \(mx^2 - (2m + 3)x + (m - 1) = 0\) \\
%\(x\) étant l'inconnue et \(m\) le paramètre réel.

%\begin{enumerate}
%    \item Pour quelle valeur de \(m\) l'équation est-elle du second degré?

%    \item On suppose \(m \neq 0\)
%    \begin{enumerate}
%        \item Calculer le discriminant \(\Delta\) en fonction de \(m\).

%        \item Déterminer les valeurs de \(m\) pour lesquelles (E) admet deux solutions distinctes.
%
%        \item Déterminer les valeurs de \(m\) pour lesquelles (E) admet deux solutions de signes contraires.

%        \item Déterminer les valeurs de \(m\) pour lesquelles (E) admet deux solutions positives.

%       \item Déterminer les valeurs de \(m\) pour lesquelles (E) admet deux solutions négatives.

%        \item Déterminer les valeurs de \(m\) pour lesquelles (E) admet deux solutions opposées.

%        \item Déterminer les valeurs de \(m\) pour lesquelles (E) admet deux solutions inverses.
%    \end{enumerate}
%\end{enumerate}
\section*{EXERCICE 2 (4,5 points)}

\begin{enumerate}
    \item Résoudre par la méthode du déterminant de Cramer le système (A) suivant :\textbf{(1,5 pt)}
    \[
    \begin{cases}
    2x + y = 4 \\
    3x + 2y = 7
    \end{cases}
    \]

    \item Résoudre ce système par une méthode autre que le déterminant de Cramer.\textbf{(1,5 pt)}

    \item On suppose que l’unique solution du système (A) est \((1, 2)\). En déduire l’ensemble des solutions des systèmes suivants :\textbf{(0,5 $\times$ 3 pt)}
    \begin{enumerate}
        \item 
        \[
        \begin{cases}
        2x^2 + \frac{1}{y} = 4 \\
        -3x^2 - \frac{2}{y} = -7
        \end{cases}
        \]
        
        \item 
        \[
        \begin{cases}
        2|x - 1| + \sqrt{y} = 4 \\
        2|x - 1| + 2\sqrt{y} = 7
        \end{cases}
        \]
        
        \item 
        \[
        \begin{cases}
        -\frac{2}{x} - \frac{1}{y} = 4 \\
        \frac{3}{x} + \frac{2}{y} = -7
        \end{cases}
        \]
    \end{enumerate}
\end{enumerate}
\section*{Exercice 3 : (5 points)}

Dans le plan muni d’un repère orthonormal, on considère la droite d’équation \(-4x + 3y + 1 = 0\) et les points \(A(2, -3)\), \(B(1, 1)\) et \(C(-2, -3)\).\\
On fera une figure que l’on complétera.

\begin{enumerate}
    \item Montrer que la droite (L) passe par les points B et C.\textbf{(0,5 pt)}

    \item Trouver un système d’équations paramétriques de la droite (L).\textbf{(1,5 pt)}

    \item Trouver une équation cartésienne de la droite (D) qui passe par le point \(E(-1, -7)\) et de vecteur directeur \(\vec{u}(3, 5)\).\textbf{(1,5 pt)}

    \item Étudier la position relative des droites (L) et (D).\textbf{(1,5 pt)}
\end{enumerate}
%\section*{EXERCICE 3 (3 points)}

%Soit \((\vec{i}, \vec{j})\) une base. \( \vec{U} \) et \( \vec{V} \), deux vecteurs définis par leurs coordonnées respectives \( (1, 1) \) et \( (1, -1) \).

%\begin{enumerate}
%    \item Montrer que \((\vec{U}, \vec{V})\) est une base.

%    \item Exprimer \(\vec{i}\) et \(\vec{j}\) en fonction de \(\vec{U}\) et \(\vec{V}\).

%    \item Soit \(\vec{W}(3, 1)\) vecteur dans la base \((\vec{i}, \vec{j})\). Déterminer les coordonnées de \(\vec{W}\) dans la base \((\vec{U}, \vec{V})\).

%    \item Soit \(\vec{T}\) le vecteur de coordonnées \((x, y)\) dans la base \((\vec{i}, \vec{j})\).
%    \begin{enumerate}
%        \item Exprimer le vecteur \(\vec{T}\) en fonction des nombres \(x\) et \(y\) et des vecteurs \(\vec{U}\) et \(\vec{V}\).
        
%        \item Soit \((X, Y)\) les coordonnées du vecteur \(\vec{T}\) dans la base \((\vec{U}, \vec{V})\). Exprimer \(X\) et \(Y\) en fonction de \(x\) et \(y\).
%    \end{enumerate}
%\end{enumerate}

\section*{EXERCICE 4 (4,5 points)}
On donne les droites (D1), (D2) et (D3) d’équations respectives :

\( x – y + 1 =0\) ,  \(x + y + 9 = 0\) et  	 \(4x – y – 14 = 0\)

    Ces droites forment un triangle ABC. 
    
    1) Calculer les coordonnées des points A , B et C. \textbf{(0,5 pt $\times$ 2)}
    
    2) Trouver l’équation de la droite (L) parallèle à (D1) et passant par C.\textbf{(1,5 pt)}
    
    3) Trouver l’équation de la droite (L’) perpendiculaire à (D2) passant par B.\textbf{(1,5 pt)}
    
\end{document}
