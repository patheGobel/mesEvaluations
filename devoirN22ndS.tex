\documentclass{article}
\usepackage{amsmath}
\usepackage{amssymb}
\usepackage[margin=2cm]{geometry}

\begin{document}

\begin{minipage}{0.5\textwidth}
	Ministère de l'éducation nationale  \\
	Inspection académique de Kédougou   \\
	Lycée de Dindéferlo            \\
	Cellule de mathématiques            \\
	M. BA                          \\
	Classe : $2^{nd}S$  \\
\end{minipage}
\begin{minipage}{0.5\textwidth}
	Année scolaire 2023-2024 \\
	Date : 25-04-2024 \\
	Durée : 2h 00 \\
\end{minipage}

\begin{center}
	\textbf{{\underline{Devoir N1 Du Premier Semestre}}}
\end{center}

\section*{Exercice 1 (10 points) :}
\subsection*{1) Mettre sous forme canonique les trinômes suivants : (1 pt $\times$ 5)}
$A(x) = 7x^{2}-11x + 13$\\
$B(x) =\sqrt{2}x^{2}+3\sqrt{3}x-1$\\
$C(x) =x^{2}-\sqrt{5}$\\
$D =x^{2} + 2mx-3m^{2}$\\
$E(x) =(m-2)x^{2}+(5m-14)x-20$
\subsection*{1) Factoriser, si possible, chacun des trinômes suivants : (1 pt $\times$ 5)}
$F(x) =x^{2}-x\sqrt{2}+2$\\
$G(x)=(m-3)x^{2}-9x$\\ 
$H(x)=x^{2}-mx +\frac{1}{4}m^{2}+m+\frac{3}{4}$\\ 
$I(x)= x^{2} + 2(\sqrt{2}-2)x+5-4\sqrt{2}$\\ 
$J(x)=4x^{2}-4\sqrt{2+3\sqrt{2}}x+2+3\sqrt{2}$\\
\section*{Exercice 2 (5 points) :}
\subsection*{ Résoudre dans $\mathbb{R}$ les équations suivantes : (1 pt $\times$ 5)}
a°)$x^{2}-12x+36=0$;\\ 
b°) $-x^{2}-6x+16=0$ ;\\
c°) $\frac{x^{2}}{2}-3x+\frac{5}{2}=0$\\
\section*{Exercice 3 (5 points) :}
1) Soit $G=\mathrm{Bar}\lbrace (A,2),(B,5)\rbrace$\\
Montrer que $\overrightarrow{AG}=\frac{5}{7}\overrightarrow{AB}$ puis construire le point $G$

2) Déterminer dans chaque cas les réels $\alpha$ et $\beta$ pour que $G$ soit le \\barycentre des points pondérés $(A,\alpha)$ et $(B,\beta)$\\
a) $\overrightarrow{AB}=-\frac{2}{5}\overrightarrow{GB}$.\\
b) $3\overrightarrow{AG}=2\overrightarrow{BA}$.\\
\end{document}