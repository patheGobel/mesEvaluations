\documentclass[12pt]{article}
\usepackage{stmaryrd}
\usepackage{graphicx}
\usepackage[utf8]{inputenc}

\usepackage[french]{babel}
\usepackage[T1]{fontenc}
\usepackage{hyperref}
\usepackage{verbatim}

\usepackage{color, soul}

\usepackage{pgfplots}
\pgfplotsset{compat=1.15}
\usepackage{mathrsfs}

\usepackage{amsmath}
\usepackage{amsfonts}
\usepackage{amssymb}
\usepackage{tkz-tab}

\usepackage{tikz}
\usetikzlibrary{arrows, shapes.geometric, fit}


\usepackage[margin=2cm]{geometry}
\begin{document}

\begin{minipage}{0.5\textwidth}
	Ministère de l'éducation nationale  \\
	Inspection académique de Kédougou   \\
	Lycée de Dindéfélo            \\
	Cellule de mathématiques            \\
	M. BA                          \\
	Classe : TS  \\
\end{minipage}
\begin{minipage}{0.5\textwidth}
	Année scolaire 2023-2024 \\
	Date : 08-05-2024 \\
	Durée : 4h 00 \\
\end{minipage}

\begin{center}
	\textbf{{\underline{Devoir N1 Du Second Semestre}}}
\end{center}
\section*{\underline{Exercice 1} (2 points) :}
Calcucler les limtes suivantes
\[\lim_{x \to +\infty}\frac{\ln(x-1)}{2x-3}\quad\quad\lim_{x \to +\infty}\frac{\ln(x)}{\sqrt{x}}\]
\section*{\underline{Exercice 2} (3 points) :}
1)Resoudre dans $\mathbb{C}$ l'équation $z^{4}=1$. \textbf{1,5pt}

2)En déduire les solutions dans $\mathbb{C}$ de l'équation $\left[\left(\sqrt{3}-i\right)z+i  \right]^{4}$=1. \textbf{1,5pt}
\section*{\underline{Exercice 3} (4 points) :}
Le plan complexe est muni d’un repère orthonormé $(O,\vec{u},\vec{v})$

On considère les points A et B d’affixes respectives a et b tels que

$a=-(\frac{1+\sqrt{3}}{2})(1-i)$ et $b=-(\frac{1+\sqrt{3}}{2})(1+i)$ 

1) Ecrire a et b sous forme trigonométrique.\textbf{1pt}

2) Quelle est la nature du triangle OAB ?\textbf{1pt}

3) On désigne par C le point d’affixe 1.

a) Ecrire le nombre complexe $\frac{1-a}{b-a}$  sous forme trigonométrique.\textbf{1pt}

b) En déduire la nature du triangle ABC et  utiliser ce résultat pour placer les points A, B, C sur la figure.\textbf{1pt}
\section*{\underline{Problème} (11 points) :}
\subsection*{Partie A:}
Soit $f$ la fonction numérique définie par $f(x)=-x+e^{x-1}$  et $C_{f}$ sa courbe dans un repère orthonormé $(O,\vec{i},\vec{j})$ unité 1cm

1) Calculer les limites de $f$ aux bornes de son ensemble de définition puis étudier la nature des branches infinies.\textbf{2pts}

Préciser la position de la courbe par rapport à son asymptote.\textbf{1pt}

2) Dresser le tableau de variation de $f$ puis en déduire le signe de $f(x)$.\textbf{1,5pts}

3) Construire $C_{f}$.\textbf{0,5pts}

4) Soit $g$ la fonction définie sur $\mathbb{R}$ par $g(x)=x+e^{-x-1}$

  a) Exprimer $g(x)$ en fonction de $f(x)$ \textbf{0,5pt}
  
  b) Construire alors $C_{g}$ dans le même repère.  \textbf{0,5pt}
\section*{Partie B:}
Soit $h$ la fonction définie par $h(x)=\ln(-x+e^{x-1})$  et $\Gamma$ sa courbe dans autre repère orthonormal

1)

	a)Déterminer l’ensemble de définition de $h$ puis calculer les limites de $h$ aux bornes de son ensemble de définition. \textbf{1pt}
	
	b)Montrer que $\forall x>1, h(x)=\ln(-xe^{-x}+e^{-1}) $ puis montrer que la droite $y=x-1$ d’équation est une asymptote à $\Gamma$. \textbf{1pt}
	
2) Dresser le tableau de variation de  $h$.   \textbf{1pt}
    
3) Soit $\varphi$ la restriction de $h$ à l’intervalle $I=\left] 1, +\infty \right[ $
    
    a) Montrer que $\varphi$  réalise une bijection de $I$ vers $J$  à préciser.
    \textbf{0,5pt}
        
    b) Montrer que l’équation $\varphi(x)=0$ admet une solution unique $\alpha$ dont on précisera une valeur approchée à $10^{-1}$ prés   \textbf{1pt}
    
    c) Construire $\Gamma$ et la courbe de $\varphi^{-1}$  dans le même repère.  \textbf{0,5pt}
\end{document}