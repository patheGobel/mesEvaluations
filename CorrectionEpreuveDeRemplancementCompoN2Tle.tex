\documentclass[12pt]{article}
\usepackage{stmaryrd}
\usepackage{graphicx}
\usepackage[utf8]{inputenc}

\usepackage[french]{babel}
\usepackage[T1]{fontenc}
\usepackage{hyperref}
\usepackage{verbatim}

\usepackage{color, soul}

\usepackage{pgfplots}
\pgfplotsset{compat=1.15}
\usepackage{mathrsfs}

\usepackage{amsmath}
\usepackage{amsfonts}
\usepackage{amssymb}
\usepackage{tkz-tab}

\usepackage{tikz}
\usetikzlibrary{arrows, shapes.geometric, fit}


\usepackage[margin=2cm]{geometry}
\begin{document}

\begin{minipage}{0.5\textwidth}
	Ministère de l'éducation nationale  \\
	Inspection académique de Kédougou   \\
	Cellule de Mathématiques\\
	M.BA\\
	Classe : Tle  \\
\end{minipage}
\begin{minipage}{0.5\textwidth}
	Année scolaire 2023-2024 \\
	Date : 14-05-2024 \\
	Durée : 3h 00 \\
\end{minipage}

\begin{center}
	\textbf{{\underline{\textcolor{green}{Correction du Devoir N2 Du Second Semestre}}}}
\end{center}
\section*{\underline{Exercice 1: }\textbf{6 pts}}
\subsection*{ Resoudre dans $\mathbb{R}$ 1pt+1pt+1,5pts+1pt+1,5pts}
\begin{itemize}
\item[a)] $\ln(2x-1)=\ln(x+1)$

\item[b)] $\ln(x-1)+\ln(x+1)=\ln(x+2)$

\item[c)] $\ln(2x-1)+2\ln(x+1)=\ln(x-1)$

\item[d)] $\ln(x-1)\leq\ln(3-x)$

\item[e)] $\ln(1-x)-\ln(2x+3)\geq\ln(x-1)$
\end{itemize}
\section*{\underline{\textcolor{green}{Correction Exercice 1: \textbf{6 pts}}}}

\subsection*{a) \(\ln(2x-1)=\ln(x+1)\)}

\textbf{\underline{\textcolor{green}{Domaine de Validité: D}}}

L'équation n'a de sens que si $2x-1>0$ et $x+1>0$

Posons $2x-1=0$ et $x+1=0$

C'est-à-dire $x=\frac{1}{2}$ et $x=-1$

\definecolor{cqcqcq}{rgb}{0.7529411764705882,0.7529411764705882,0.7529411764705882}
\begin{tikzpicture}[line cap=round,line join=round,>=triangle 45,x=1cm,y=1cm]
%\draw [color=cqcqcq,, xstep=1cm,ystep=1cm] (-7,-10) grid (-22,17);
\clip(-22,3) rectangle (12,10);
\draw [line width=2pt] (-23,8)-- (-7,8); %première ligne A(-22,8)---B(-7,8)
\draw [line width=2pt] (-22,6)-- (-7,6); %deuxième ligne
\draw [line width=2pt] (-22,5)-- (-7,5); %troisième  ligne
\draw [line width=2pt] (-22,4)-- (-7,4); %quatrième ligne
\draw [line width=2pt] (-22,4)-- (-22,8); %première colonne (-22,4)<----(-22,8);
\draw [line width=2pt] (-18,8)-- (-18,4); %deuxième colone  (-18,8)--->(-18,4);
\draw [line width=2pt] (-7,8)-- (-7,4); %quatrième colonne (-7,8)-->(-7,4);
\draw (-21,7) node[anchor=north west] {$x$};
\draw (-18,7) node[anchor=north west] {$-\infty$};
\draw (-8,7) node[anchor=north west] {$+\infty$};
\draw (-21,5.7) node[anchor=north west] {$2x-1$};
\draw (-15.8,5.7) node[anchor=north west] {$-$};
\draw (-15.3,4.8) node[anchor=north west] {$O$};
\draw (-10.5,5.7) node[anchor=north west] {$+$};
\draw (-21,4.7) node[anchor=north west] {$x+1$};
\draw (-15,5.7) node[anchor=north west] {$-$};
\draw (-15.8,4.7) node[anchor=north west] {$-$};
\draw (-11.3,5.8) node[anchor=north west] {$O$};
\draw (-10.5,4.7) node[anchor=north west] {$+$};
\draw (-15,4.7) node[anchor=north west] {$+$};
\draw [line width=2pt] (-15,6)-- (-15,4); %(-13,6)-- (-13,4);
\draw [line width=2pt] (-11,6)-- (-11,4); %(-13,6)-- (-13,4);
\draw (-15.5,7) node[anchor=north west] {$-1$};
\draw (-11.3,7) node[anchor=north west] {$\frac{1}{2}$};
\end{tikzpicture}

Donc D=$\left]\frac{1}{2} , +\infty\right[ $

\textbf{\underline{\textcolor{green}{Résolution}}}

$\ln(2x-1)=\ln(x+1)\Longrightarrow 2x-1=x+1\Longrightarrow x=-2$

Comme $-2\notin$D Donc \textcolor{green}{\boxed{S=\emptyset}}

--------------------------------------------------------------------------------------------------------------

\subsection*{b) $\ln(x-1)+\ln(x+1)=\ln(x+2)$}

\textbf{\underline{\textcolor{green}{Domaine de Validité: D}}}

L'équation n'a de sens que si $x-1>0$, $x+1>0$ et $x+2>0$.

Posons $x-1=0$, $x+1=0$ et $x+2=0$.

C'est-à-dire $x=1$, $x=-1$ et $x=-2$.

\definecolor{cqcqcq}{rgb}{0.7529411764705882,0.7529411764705882,0.7529411764705882}
\begin{tikzpicture}[line cap=round,line join=round,>=triangle 45,x=1cm,y=1cm]
%\draw [color=cqcqcq,, xstep=1cm,ystep=1cm] (-7,-10) grid (-22,17);
\clip(-22,3) rectangle (12,10);
\draw [line width=2pt] (-23,8)-- (-7,8); %première ligne A(-22,8)---B(-7,8)
\draw [line width=2pt] (-22,6)-- (-7,6); %deuxième ligne
\draw [line width=2pt] (-22,5)-- (-7,5); %troisième  ligne
\draw [line width=2pt] (-22,4)-- (-7,4); %quatrième ligne
\draw [line width=2pt] (-22,3)-- (-7,3); %cinquième ligne
\draw [line width=2pt] (-22,3)-- (-22,8); %première colonne (-22,4)<----(-22,8);
\draw [line width=2pt] (-18,8)-- (-18,3); %deuxième colone  (-18,8)--->(-18,4);
\draw [line width=2pt] (-15,6)-- (-15,3); %troisième colone (-13,6)-- (-13,4);
\draw [line width=2pt] (-13,6)-- (-13,3); %quatrième colonne (-13,6)-- (-13,4);
\draw [line width=2pt] (-11,6)-- (-11,3); %cinquième colonne (-13,6)-- (-13,4);
\draw [line width=2pt] (-7,8)-- (-7,3); % colonne (-7,8)-->(-7,4);
\draw (-21,7) node[anchor=north west] {$x$};
\draw (-18,7) node[anchor=north west] {$-\infty$};
\draw (-8,7) node[anchor=north west] {$+\infty$};
\draw (-21,5.7) node[anchor=north west] {$x-2$};
\draw (-15.8,5.7) node[anchor=north west] {$-$};
\draw (-15.3,5.8) node[anchor=north west] {$O$};
\draw (-15,5.7) node[anchor=north west] {$+$};
\draw (-13,5.7) node[anchor=north west] {$+$};
\draw (-10.5,5.7) node[anchor=north west] {$+$};

\draw (-21,4.7) node[anchor=north west] {$x+1$};
\draw (-15.8,4.7) node[anchor=north west] {$-$};
\draw (-13,4.7) node[anchor=north west] {$+$};
\draw (-15,4.7) node[anchor=north west] {$-$};
\draw (-13.3,4.8) node[anchor=north west] {$O$};
\draw (-10.5,4.7) node[anchor=north west] {$+$};

\draw (-21,4) node[anchor=north west] {$x-1$};
\draw (-15.8,4) node[anchor=north west] {$-$};
\draw (-15,4) node[anchor=north west] {$-$};
\draw (-13,4) node[anchor=north west] {$-$};
\draw (-11.3,4) node[anchor=north west] {$O$};
\draw (-10.5,4) node[anchor=north west] {$+$};

\draw (-15.4,7) node[anchor=north west] {$-2$};
\draw (-13.3,7) node[anchor=north west] {$-1$};
\draw (-11.3,7) node[anchor=north west] {$1$};
\end{tikzpicture}

Donc $D=\left]1, +\infty\right[$.

\textbf{\underline{\textcolor{green}{Résolution}}}

$\ln(x-1) + \ln(x+1) = \ln(x+2) \Longrightarrow \ln((x-1)(x+1)) = \ln(x+2)$

$\Longrightarrow (x-1)(x+1) = x+2 \Longrightarrow x^2 - 1 = x+2 \Longrightarrow x^2 - x - 3 = 0$

$\Longrightarrow (x-3)(x+1) = 0 \Longrightarrow x=3$ ou $x=-1$.

Comme $x \in D$, donc $S=\{3\}$.

--------------------------------------------------------------------------------------------------------------

\subsection*{c)$\ln(2x-1)+2\ln(x+1)=\ln(x-1)$}
\textbf{\underline{\textcolor{green}{Domaine de Validité: D}}}

L'équation n'a de sens que si $2x-1>0$, $x+1>0$ et $x-1>0$.

Posons $2x-1=0$, $x+1=0$ et $x-1=0$.

C'est-à-dire $x=\frac{1}{2}$, $x=-1$ et $x=1$.

\definecolor{cqcqcq}{rgb}{0.7529411764705882,0.7529411764705882,0.7529411764705882}
\begin{tikzpicture}[line cap=round,line join=round,>=triangle 45,x=1cm,y=1cm]
%\draw [color=cqcqcq,, xstep=1cm,ystep=1cm] (-7,-10) grid (-22,17);
\clip(-22,3) rectangle (12,10);
\draw [line width=2pt] (-23,8)-- (-7,8); %première ligne A(-22,8)---B(-7,8)
\draw [line width=2pt] (-22,6)-- (-7,6); %deuxième ligne
\draw [line width=2pt] (-22,5)-- (-7,5); %troisième  ligne
\draw [line width=2pt] (-22,4)-- (-7,4); %quatrième ligne
\draw [line width=2pt] (-22,3)-- (-7,3); %cinquième ligne
\draw [line width=2pt] (-22,3)-- (-22,8); %première colonne (-22,4)<----(-22,8);
\draw [line width=2pt] (-18,8)-- (-18,3); %deuxième colone  (-18,8)--->(-18,4);
\draw [line width=2pt] (-15,6)-- (-15,3); %troisième colone (-13,6)-- (-13,4);
\draw [line width=2pt] (-13,6)-- (-13,3); %quatrième colonne (-13,6)-- (-13,4);
\draw [line width=2pt] (-11,6)-- (-11,3); %cinquième colonne (-13,6)-- (-13,4);
\draw [line width=2pt] (-7,8)-- (-7,3); % colonne (-7,8)-->(-7,4);
\draw (-21,7) node[anchor=north west] {$x$};
\draw (-18,7) node[anchor=north west] {$-\infty$};
\draw (-8,7) node[anchor=north west] {$+\infty$};
\draw (-21,5.7) node[anchor=north west] {$x+1$};
\draw (-15.8,5.7) node[anchor=north west] {$-$};
\draw (-15.3,5.8) node[anchor=north west] {$O$};
\draw (-15,5.7) node[anchor=north west] {$+$};
\draw (-13,5.7) node[anchor=north west] {$+$};
\draw (-10.5,5.7) node[anchor=north west] {$+$};

\draw (-21,4.7) node[anchor=north west] {$2x-1$};
\draw (-15.8,4.7) node[anchor=north west] {$-$};
\draw (-13,4.7) node[anchor=north west] {$+$};
\draw (-15,4.7) node[anchor=north west] {$-$};
\draw (-13.3,4.8) node[anchor=north west] {$O$};
\draw (-10.5,4.7) node[anchor=north west] {$+$};

\draw (-21,4) node[anchor=north west] {$x-1$};
\draw (-15.8,4) node[anchor=north west] {$-$};
\draw (-15,4) node[anchor=north west] {$-$};
\draw (-13,4) node[anchor=north west] {$-$};
\draw (-11.3,4) node[anchor=north west] {$O$};
\draw (-10.5,4) node[anchor=north west] {$+$};

\draw (-15.4,7) node[anchor=north west] {$-1$};
\draw (-13.3,7) node[anchor=north west] {$\frac{1}{2}$};
\draw (-11.3,7) node[anchor=north west] {$1$};
\end{tikzpicture}

Donc $D=\left]1, +\infty\right[$.

\textbf{\underline{\textcolor{green}{Résolution}}}

$\ln(2x-1)+2\ln(x+1)=\ln(x-1) \Longrightarrow \ln((2x-1)(x+1)^{2}) = \ln(x+2)$

$\Longrightarrow (2x-1)(x^{2}+2x+1) = x+2 \Longrightarrow 2x(x^{2}+2x+1) - (x^{2}+2x+1) = x+2$\\

$\Longrightarrow 2x^3 + 4x^{2} + 2x-x^{2}-2x-1 = x+2 \Longrightarrow 2x^3 + 3x^{2} + x-3=0$

\textbf{Grosse erreur de ma part car pas de racines évidentes.} 

-----------------------------------------------------------------------------------------------------------
\subsection*{d)$\ln(x-1)\leq\ln(3-x)$}
\textbf{\underline{\textcolor{green}{Domaine de Validité: D}}}

L'équation n'a de sens que si $x-1>0$ et $3-x>0$

Posons $x-1=0$ et $3-x=0$

C'est-à-dire $x=1$ et $x=3$

\definecolor{cqcqcq}{rgb}{0.7529411764705882,0.7529411764705882,0.7529411764705882}
\begin{tikzpicture}[line cap=round,line join=round,>=triangle 45,x=1cm,y=1cm]
%\draw [color=cqcqcq,, xstep=1cm,ystep=1cm] (-7,-10) grid (-22,17);
\clip(-22,3) rectangle (12,10);
\draw [line width=2pt] (-23,8)-- (-7,8); %première ligne A(-22,8)---B(-7,8)
\draw [line width=2pt] (-22,6)-- (-7,6); %deuxième ligne
\draw [line width=2pt] (-22,5)-- (-7,5); %troisième  ligne
\draw [line width=2pt] (-22,4)-- (-7,4); %quatrième ligne
\draw [line width=2pt] (-22,4)-- (-22,8); %première colonne (-22,4)<----(-22,8);
\draw [line width=2pt] (-18,8)-- (-18,4); %deuxième colone  (-18,8)--->(-18,4);
\draw [line width=2pt] (-7,8)-- (-7,4); %quatrième colonne (-7,8)-->(-7,4);
\draw (-21,7) node[anchor=north west] {$x$};
\draw (-18,7) node[anchor=north west] {$-\infty$};
\draw (-8,7) node[anchor=north west] {$+\infty$};
\draw (-21,5.7) node[anchor=north west] {$x-3$};
\draw (-15.8,5.7) node[anchor=north west] {$-$};
\draw (-15.3,4.8) node[anchor=north west] {$O$};
\draw (-10.5,5.7) node[anchor=north west] {$+$};
\draw (-21,4.7) node[anchor=north west] {$x-1$};
\draw (-15,5.7) node[anchor=north west] {$-$};
\draw (-15.8,4.7) node[anchor=north west] {$-$};
\draw (-11.3,5.8) node[anchor=north west] {$O$};
\draw (-10.5,4.7) node[anchor=north west] {$+$};
\draw (-15,4.7) node[anchor=north west] {$+$};
\draw [line width=2pt] (-15,6)-- (-15,4); %(-13,6)-- (-13,4);
\draw [line width=2pt] (-11,6)-- (-11,4); %(-13,6)-- (-13,4);
\draw (-15.5,7) node[anchor=north west] {$1$};
\draw (-11.3,7) node[anchor=north west] {$3$};
\end{tikzpicture}

Donc D=$\left]3 , +\infty\right[ $

\textbf{\underline{\textcolor{green}{Résolution}}}

$\ln(x-1)\leq\ln(3-x)\Longrightarrow x-1\leq3-x\Longrightarrow x\leq4 \Longrightarrow x\in\left]-\infty , 4\right] $

S=$\left]3 , +\infty\right[\cap\left]-\infty , 4\right] $

\textcolor{green}{\boxed{S=\left]3; 4\right[}}

------------------------------------------------------------------------------------------------------------

e) $\ln(1-x)-\ln(2x+3)\geq\ln(x-1)$

\textbf{\underline{\textcolor{green}{Domaine de Validité: D}}}

L'équation n'a de sens que si $1-x>0$, $2x+3>0$ et $x-1>0$.

Posons $1-x=0$, $2x+3=0$ et $x-1=0$.

C'est-à-dire $x=1$, $x=-\frac{3}{2}$ et $x=1$.

\definecolor{cqcqcq}{rgb}{0.7529411764705882,0.7529411764705882,0.7529411764705882}
\begin{tikzpicture}[line cap=round,line join=round,>=triangle 45,x=1cm,y=1cm]
%\draw [color=cqcqcq,, xstep=1cm,ystep=1cm] (-7,-10) grid (-22,17);
\clip(-22,3) rectangle (12,10);
\draw [line width=2pt] (-23,8)-- (-7,8); %première ligne A(-22,8)---B(-7,8)
\draw [line width=2pt] (-22,6)-- (-7,6); %deuxième ligne
\draw [line width=2pt] (-22,5)-- (-7,5); %troisième  ligne
\draw [line width=2pt] (-22,4)-- (-7,4); %quatrième ligne
\draw [line width=2pt] (-22,3)-- (-7,3); %cinquième ligne
\draw [line width=2pt] (-22,3)-- (-22,8); %première colonne (-22,4)<----(-22,8);
\draw [line width=2pt] (-18,8)-- (-18,3); %deuxième colone  (-18,8)--->(-18,4);
\draw [line width=2pt] (-15,6)-- (-15,3); %troisième colone (-13,6)-- (-13,4);
%\draw [line width=2pt] (-13,6)-- (-13,3); %quatrième colonne (-13,6)-- (-13,4);
\draw [line width=2pt] (-11,6)-- (-11,3); %cinquième colonne (-13,6)-- (-13,4);
\draw [line width=2pt] (-7,8)-- (-7,3); % colonne (-7,8)-->(-7,4);
\draw (-21,7) node[anchor=north west] {$x$};
\draw (-18,7) node[anchor=north west] {$-\infty$};
\draw (-8,7) node[anchor=north west] {$+\infty$};
\draw (-21,5.7) node[anchor=north west] {$2x+3$};
\draw (-15.8,5.7) node[anchor=north west] {$-$};
\draw (-15.3,5.8) node[anchor=north west] {$O$};
\draw (-15,5.7) node[anchor=north west] {$+$};
\draw (-13,5.7) node[anchor=north west] {$+$};
\draw (-10.5,5.7) node[anchor=north west] {$+$};

\draw (-21,4.7) node[anchor=north west] {$x-1$};
\draw (-15.8,4.7) node[anchor=north west] {$-$};
\draw (-13,4.7) node[anchor=north west] {$-$};
\draw (-15,4.7) node[anchor=north west] {$-$};
\draw (-11.3,4.8) node[anchor=north west] {$O$};
\draw (-10.5,4.7) node[anchor=north west] {$+$};

\draw (-21,4) node[anchor=north west] {$x+1$};
\draw (-15.8,4) node[anchor=north west] {$-$};
\draw (-15,4) node[anchor=north west] {$-$};
\draw (-13,4) node[anchor=north west] {$-$};
\draw (-11.3,4) node[anchor=north west] {$O$};
\draw (-10.5,4) node[anchor=north west] {$+$};

\draw (-15.4,7) node[anchor=north west] {$-\frac{3}{2}$};
\draw (-11.3,7) node[anchor=north west] {$1$};
\end{tikzpicture}

Donc $D=\left]1, +\infty\right[$.

\textbf{\underline{\textcolor{green}{Résolution}}}

$\ln(1-x)-\ln(2x+3)\geq\ln(x-1) \Longrightarrow \ln(1-x) \geq \ln(2x+3)+\ln(x-1)$

$\Longrightarrow \ln(1-x) \geq \ln\left[ (2x+3)(x-1)\right]  \Longrightarrow (1-x) \geq (2x+3)(x-1) \Longrightarrow 1-x \geq 2x^{2}-2x+3x-3$

$\Longrightarrow 2x^{2}+2x-4 \leq 0 $.

Posons $x^{2}+x-1 = 0$

$\Delta=5$

$x_{1}=\frac{-1+\sqrt{5}}{2}$ et $x_{1}=\frac{-1-\sqrt{5}}{2}$

Donc $ S=\left]\frac{-1-\sqrt{5}}{2}, \frac{-1+\sqrt{5}}{2} \right[ \cap \left]1, +\infty\right[=\emptyset$

Donc \textcolor{green}{\boxed{S=\emptyset}}
\section*{\underline{\textcolor{green}{Correction Exercice 2: \textbf{6 pts}}}}

\subsection*{1) Développons } \((x+1)(x-3)(x+2)\)

Pour développer \((x+1)(x-3)(x+2)\), procédons en plusieurs étapes en utilisant la distributivité.

D'abord, développons les deux premiers facteurs :

\[
(x+1)(x-3) = x(x-3) + 1(x-3) = x^2 - 3x + x - 3 = x^2 - 2x - 3
\]

Ensuite, multiplions ce résultat par le troisième facteur \((x+2)\) :

\[
(x^2 - 2x - 3)(x+2) = (x^2 - 2x - 3)x + (x^2 - 2x - 3)2
\]

Développons les deux produits :

\[
(x^2 - 2x - 3)x = x^3 - 2x^2 - 3x
\]
\[
(x^2 - 2x - 3)2 = 2x^2 - 4x - 6
\]

En ajoutant ces deux résultats ensemble, nous obtenons :

\[
x^3 - 2x^2 - 3x + 2x^2 - 4x - 6 = x^3 + (-2x^2 + 2x^2) + (-3x - 4x) - 6 = x^3 - 7x - 6
\]

Donc, le résultat final est :
\textcolor{green}{\boxed{(x+1)(x-3)(x+2) = x^3 - 7x - 6}}

\end{document}