\documentclass[12pt]{article}
\usepackage{stmaryrd}
\usepackage{graphicx}
\usepackage[utf8]{inputenc}

\usepackage[french]{babel}
\usepackage[T1]{fontenc}
\usepackage{hyperref}
\usepackage{verbatim}

\usepackage{color, soul}

\usepackage{pgfplots}
\pgfplotsset{compat=1.15}
\usepackage{mathrsfs}

\usepackage{amsmath}
\usepackage{amsfonts}
\usepackage{amssymb}
\usepackage{tkz-tab}

\usepackage{tikz}
\usetikzlibrary{arrows, shapes.geometric, fit}


\usepackage[margin=2cm]{geometry}
\begin{document}

\begin{minipage}{0.5\textwidth}
	Ministère de l'éducation nationale  \\
	Inspection académique de Kédougou   \\
	Cellule de Mathématiques\\
	M.BA\\
	Classe : Tle  \\
\end{minipage}
\begin{minipage}{0.5\textwidth}
	Année scolaire 2023-2024 \\
	Date : 14-05-2024 \\
	Durée : 3h 00 \\
\end{minipage}

\begin{center}
	\textbf{{\underline{\textcolor{green}{Correction du Devoir N2 Du Second Semestre}}}}
\end{center}
\section*{\underline{Exercice 1: }\textbf{6 pts}}
\subsection*{ Resoudre dans $\mathbb{R}$ 1pt+1pt+1,5pts+1pt+1,5pts}
\begin{itemize}
\item[a)] $\ln(2x-1)=\ln(x+1)$

\item[b)] $\ln(x-1)+\ln(x+1)=\ln(x+2)$

\item[c)] $\ln(2x-1)+2\ln(x+1)=\ln(x-1)$

\item[d)] $\ln(x-1)\leq\ln(3-x)$

\item[e)] $\ln(1-x)-\ln(2x+3)\geq\ln(x-1)$
\end{itemize}
\section*{\underline{\textcolor{green}{Correction Exercice 1: \textbf{6 pts}}}}

\subsection*{a) \(\ln(2x-1)=\ln(x+1)\)}

\textbf{\underline{\textcolor{green}{Domaine de Validité: D}}}

L'équation n'a de sens que si $2x-1>0$ et $x+1>0$

Posons $2x-1=0$ et $x+1=0$

C'est-à-dire $x=\frac{1}{2}$ et $x=-1$

\definecolor{cqcqcq}{rgb}{0.7529411764705882,0.7529411764705882,0.7529411764705882}
\begin{tikzpicture}[line cap=round,line join=round,>=triangle 45,x=1cm,y=1cm]
%\draw [color=cqcqcq,, xstep=1cm,ystep=1cm] (-7,-10) grid (-22,17);
\clip(-22,3) rectangle (12,10);
\draw [line width=2pt] (-23,8)-- (-7,8); %première ligne A(-22,8)---B(-7,8)
\draw [line width=2pt] (-22,6)-- (-7,6); %deuxième ligne
\draw [line width=2pt] (-22,5)-- (-7,5); %troisième  ligne
\draw [line width=2pt] (-22,4)-- (-7,4); %quatrième ligne
\draw [line width=2pt] (-22,4)-- (-22,8); %première colonne (-22,4)<----(-22,8);
\draw [line width=2pt] (-18,8)-- (-18,4); %deuxième colone  (-18,8)--->(-18,4);
\draw [line width=2pt] (-7,8)-- (-7,4); %quatrième colonne (-7,8)-->(-7,4);
\draw (-21,7) node[anchor=north west] {$x$};
\draw (-18,7) node[anchor=north west] {$-\infty$};
\draw (-8,7) node[anchor=north west] {$+\infty$};
\draw (-21,5.7) node[anchor=north west] {$2x-1$};
\draw (-15.8,5.7) node[anchor=north west] {$-$};
\draw (-15.3,4.8) node[anchor=north west] {$O$};
\draw (-10.5,5.7) node[anchor=north west] {$+$};
\draw (-21,4.7) node[anchor=north west] {$x+1$};
\draw (-15,5.7) node[anchor=north west] {$-$};
\draw (-15.8,4.7) node[anchor=north west] {$-$};
\draw (-11.3,5.8) node[anchor=north west] {$O$};
\draw (-10.5,4.7) node[anchor=north west] {$+$};
\draw (-15,4.7) node[anchor=north west] {$+$};
\draw [line width=2pt] (-15,6)-- (-15,4); %(-13,6)-- (-13,4);
\draw [line width=2pt] (-11,6)-- (-11,4); %(-13,6)-- (-13,4);
\draw (-15.5,7) node[anchor=north west] {$-1$};
\draw (-11.3,7) node[anchor=north west] {$\frac{1}{2}$};
\end{tikzpicture}

Donc D=$\left]\frac{1}{2} , +\infty\right[ $

\textbf{\underline{\textcolor{green}{Résolution}}}

$\ln(2x-1)=\ln(x+1)\Longrightarrow 2x-1=x+1\Longrightarrow x=2$

Comme $2\in$D Donc \textcolor{green}{\boxed{S=\left\lbrace 2  \right\rbrace }}

--------------------------------------------------------------------------------------------------------------

\subsection*{b) $\ln(x-1)+\ln(x+1)=\ln(x+2)$}

\textbf{\underline{\textcolor{green}{Domaine de Validité: D}}}

L'équation n'a de sens que si $x-1>0$, $x+1>0$ et $x+2>0$.

Posons $x-1=0$, $x+1=0$ et $x+2=0$.

C'est-à-dire $x=1$, $x=-1$ et $x=-2$.

\definecolor{cqcqcq}{rgb}{0.7529411764705882,0.7529411764705882,0.7529411764705882}
\begin{tikzpicture}[line cap=round,line join=round,>=triangle 45,x=1cm,y=1cm]
%\draw [color=cqcqcq,, xstep=1cm,ystep=1cm] (-7,-10) grid (-22,17);
\clip(-22,3) rectangle (12,10);
\draw [line width=2pt] (-23,8)-- (-7,8); %première ligne A(-22,8)---B(-7,8)
\draw [line width=2pt] (-22,6)-- (-7,6); %deuxième ligne
\draw [line width=2pt] (-22,5)-- (-7,5); %troisième  ligne
\draw [line width=2pt] (-22,4)-- (-7,4); %quatrième ligne
\draw [line width=2pt] (-22,3)-- (-7,3); %cinquième ligne
\draw [line width=2pt] (-22,3)-- (-22,8); %première colonne (-22,4)<----(-22,8);
\draw [line width=2pt] (-18,8)-- (-18,3); %deuxième colone  (-18,8)--->(-18,4);
\draw [line width=2pt] (-15,6)-- (-15,3); %troisième colone (-13,6)-- (-13,4);
\draw [line width=2pt] (-13,6)-- (-13,3); %quatrième colonne (-13,6)-- (-13,4);
\draw [line width=2pt] (-11,6)-- (-11,3); %cinquième colonne (-13,6)-- (-13,4);
\draw [line width=2pt] (-7,8)-- (-7,3); % colonne (-7,8)-->(-7,4);
\draw (-21,7) node[anchor=north west] {$x$};
\draw (-18,7) node[anchor=north west] {$-\infty$};
\draw (-8,7) node[anchor=north west] {$+\infty$};
\draw (-21,5.7) node[anchor=north west] {$x-2$};
\draw (-15.8,5.7) node[anchor=north west] {$-$};
\draw (-15.3,5.8) node[anchor=north west] {$O$};
\draw (-15,5.7) node[anchor=north west] {$+$};
\draw (-13,5.7) node[anchor=north west] {$+$};
\draw (-10.5,5.7) node[anchor=north west] {$+$};

\draw (-21,4.7) node[anchor=north west] {$x+1$};
\draw (-15.8,4.7) node[anchor=north west] {$-$};
\draw (-13,4.7) node[anchor=north west] {$+$};
\draw (-15,4.7) node[anchor=north west] {$-$};
\draw (-13.3,4.8) node[anchor=north west] {$O$};
\draw (-10.5,4.7) node[anchor=north west] {$+$};

\draw (-21,4) node[anchor=north west] {$x-1$};
\draw (-15.8,4) node[anchor=north west] {$-$};
\draw (-15,4) node[anchor=north west] {$-$};
\draw (-13,4) node[anchor=north west] {$-$};
\draw (-11.3,4) node[anchor=north west] {$O$};
\draw (-10.5,4) node[anchor=north west] {$+$};

\draw (-15.4,7) node[anchor=north west] {$-2$};
\draw (-13.3,7) node[anchor=north west] {$-1$};
\draw (-11.3,7) node[anchor=north west] {$1$};
\end{tikzpicture}

Donc $D=\left]1, +\infty\right[$.

\textbf{\underline{\textcolor{green}{Résolution}}}

$\ln(x-1) + \ln(x+1) = \ln(x+2) \Longrightarrow \ln((x-1)(x+1)) = \ln(x+2)$

$\Longrightarrow (x-1)(x+1) = x+2 \Longrightarrow x^2 - 1 = x+2 \Longrightarrow x^2 - x - 3 = 0$

$\Delta = 13$
$x=\frac{1+\sqrt{13}}{2}$, $x=\frac{1-\sqrt{13}}{2}$

Comme $\frac{1+\sqrt{13}}{2} \in D$ et $\frac{1+\sqrt{13}}{2} \notin D$, donc $S=\{\frac{1+\sqrt{13}}{2}\}$.

\textcolor{green}{\boxed{S=\left\lbrace \frac{1+\sqrt{13}}{2}  \right\rbrace }}

--------------------------------------------------------------------------------------------------------------

\subsection*{c)$\ln(2x-1)+2\ln(x+1)=\ln(x-1)$}
\textbf{\underline{\textcolor{green}{Domaine de Validité: D}}}

L'équation n'a de sens que si $2x-1>0$, $x+1>0$ et $x-1>0$.

Posons $2x-1=0$, $x+1=0$ et $x-1=0$.

C'est-à-dire $x=\frac{1}{2}$, $x=-1$ et $x=1$.

\definecolor{cqcqcq}{rgb}{0.7529411764705882,0.7529411764705882,0.7529411764705882}
\begin{tikzpicture}[line cap=round,line join=round,>=triangle 45,x=1cm,y=1cm]
%\draw [color=cqcqcq,, xstep=1cm,ystep=1cm] (-7,-10) grid (-22,17);
\clip(-22,3) rectangle (12,10);
\draw [line width=2pt] (-23,8)-- (-7,8); %première ligne A(-22,8)---B(-7,8)
\draw [line width=2pt] (-22,6)-- (-7,6); %deuxième ligne
\draw [line width=2pt] (-22,5)-- (-7,5); %troisième  ligne
\draw [line width=2pt] (-22,4)-- (-7,4); %quatrième ligne
\draw [line width=2pt] (-22,3)-- (-7,3); %cinquième ligne
\draw [line width=2pt] (-22,3)-- (-22,8); %première colonne (-22,4)<----(-22,8);
\draw [line width=2pt] (-18,8)-- (-18,3); %deuxième colone  (-18,8)--->(-18,4);
\draw [line width=2pt] (-15,6)-- (-15,3); %troisième colone (-13,6)-- (-13,4);
\draw [line width=2pt] (-13,6)-- (-13,3); %quatrième colonne (-13,6)-- (-13,4);
\draw [line width=2pt] (-11,6)-- (-11,3); %cinquième colonne (-13,6)-- (-13,4);
\draw [line width=2pt] (-7,8)-- (-7,3); % colonne (-7,8)-->(-7,4);
\draw (-21,7) node[anchor=north west] {$x$};
\draw (-18,7) node[anchor=north west] {$-\infty$};
\draw (-8,7) node[anchor=north west] {$+\infty$};
\draw (-21,5.7) node[anchor=north west] {$x+1$};
\draw (-15.8,5.7) node[anchor=north west] {$-$};
\draw (-15.3,5.8) node[anchor=north west] {$O$};
\draw (-15,5.7) node[anchor=north west] {$+$};
\draw (-13,5.7) node[anchor=north west] {$+$};
\draw (-10.5,5.7) node[anchor=north west] {$+$};

\draw (-21,4.7) node[anchor=north west] {$2x-1$};
\draw (-15.8,4.7) node[anchor=north west] {$-$};
\draw (-13,4.7) node[anchor=north west] {$+$};
\draw (-15,4.7) node[anchor=north west] {$-$};
\draw (-13.3,4.8) node[anchor=north west] {$O$};
\draw (-10.5,4.7) node[anchor=north west] {$+$};

\draw (-21,4) node[anchor=north west] {$x-1$};
\draw (-15.8,4) node[anchor=north west] {$-$};
\draw (-15,4) node[anchor=north west] {$-$};
\draw (-13,4) node[anchor=north west] {$-$};
\draw (-11.3,4) node[anchor=north west] {$O$};
\draw (-10.5,4) node[anchor=north west] {$+$};

\draw (-15.4,7) node[anchor=north west] {$-1$};
\draw (-13.3,7) node[anchor=north west] {$\frac{1}{2}$};
\draw (-11.3,7) node[anchor=north west] {$1$};
\end{tikzpicture}

Donc $D=\left]1, +\infty\right[$.

\textbf{\underline{\textcolor{green}{Résolution}}}

$\ln(2x-1)+2\ln(x+1)=\ln(x-1) \Longrightarrow \ln((2x-1)(x+1)^{2}) = \ln(x+2)$

$\Longrightarrow (2x-1)(x^{2}+2x+1) = x+2 \Longrightarrow 2x(x^{2}+2x+1) - (x^{2}+2x+1) = x+2$\\

$\Longrightarrow 2x^3 + 4x^{2} + 2x-x^{2}-2x-1 = x+2 \Longrightarrow 2x^3 + 3x^{2} + x-3=0$

\textbf{Grosse erreur de ma part car pas de racines évidentes.} 

-----------------------------------------------------------------------------------------------------------
\subsection*{d)$\ln(x-1)\leq\ln(3-x)$}
\textbf{\underline{\textcolor{green}{Domaine de Validité: D}}}

L'équation n'a de sens que si $x-1>0$ et $3-x>0$

Posons $x-1=0$ et $3-x=0$

C'est-à-dire $x=1$ et $x=3$

\definecolor{cqcqcq}{rgb}{0.7529411764705882,0.7529411764705882,0.7529411764705882}
\begin{tikzpicture}[line cap=round,line join=round,>=triangle 45,x=1cm,y=1cm]
%\draw [color=cqcqcq,, xstep=1cm,ystep=1cm] (-7,-10) grid (-22,17);
\clip(-22,3) rectangle (12,10);
\draw [line width=2pt] (-23,8)-- (-7,8); %première ligne A(-22,8)---B(-7,8)
\draw [line width=2pt] (-22,6)-- (-7,6); %deuxième ligne
\draw [line width=2pt] (-22,5)-- (-7,5); %troisième  ligne
\draw [line width=2pt] (-22,4)-- (-7,4); %quatrième ligne
\draw [line width=2pt] (-22,4)-- (-22,8); %première colonne (-22,4)<----(-22,8);
\draw [line width=2pt] (-18,8)-- (-18,4); %deuxième colone  (-18,8)--->(-18,4);
\draw [line width=2pt] (-7,8)-- (-7,4); %quatrième colonne (-7,8)-->(-7,4);
\draw (-21,7) node[anchor=north west] {$x$};
\draw (-18,7) node[anchor=north west] {$-\infty$};
\draw (-8,7) node[anchor=north west] {$+\infty$};
\draw (-21,5.7) node[anchor=north west] {$3-x$};
\draw (-15.8,5.7) node[anchor=north west] {$+$};
\draw (-15.3,4.8) node[anchor=north west] {$O$};
\draw (-10.5,5.7) node[anchor=north west] {$-$};
\draw (-21,4.7) node[anchor=north west] {$x-1$};
\draw (-15,5.7) node[anchor=north west] {$+$};
\draw (-15.8,4.7) node[anchor=north west] {$-$};
\draw (-11.3,5.8) node[anchor=north west] {$O$};
\draw (-10.5,4.7) node[anchor=north west] {$+$};
\draw (-15,4.7) node[anchor=north west] {$+$};
\draw [line width=2pt] (-15,6)-- (-15,4); %(-13,6)-- (-13,4);
\draw [line width=2pt] (-11,6)-- (-11,4); %(-13,6)-- (-13,4);
\draw (-15.5,7) node[anchor=north west] {$1$};
\draw (-11.3,7) node[anchor=north west] {$3$};
\end{tikzpicture}

Donc D=$\left]1 , 3 \right[ $

\textbf{\underline{\textcolor{green}{Résolution}}}

$\ln(x-1)\leq\ln(3-x)\Longrightarrow x-1\leq3-x\Longrightarrow x\leq 2 \Longrightarrow x\in\left]-\infty ; 2\right] $

S=$\left]1 , 3 \right[\cap\left]-\infty ; 2\right]$

\textcolor{green}{\boxed{S=\left]1; 2\right]}}

------------------------------------------------------------------------------------------------------------

e) $\ln(1-x)-\ln(2x+3)\geq\ln(x-1)$

\textbf{\underline{\textcolor{green}{Domaine de Validité: D}}}

L'équation n'a de sens que si $1-x>0$, $2x+3>0$ et $x-1>0$.

Posons $1-x=0$, $2x+3=0$ et $x-1=0$.

C'est-à-dire $x=1$, $x=-\frac{3}{2}$ et $x=1$.

\definecolor{cqcqcq}{rgb}{0.7529411764705882,0.7529411764705882,0.7529411764705882}
\begin{tikzpicture}[line cap=round,line join=round,>=triangle 45,x=1cm,y=1cm]
%\draw [color=cqcqcq,, xstep=1cm,ystep=1cm] (-7,-10) grid (-22,17);
\clip(-22,3) rectangle (12,10);
\draw [line width=2pt] (-23,8)-- (-7,8); %première ligne A(-22,8)---B(-7,8)
\draw [line width=2pt] (-22,6)-- (-7,6); %deuxième ligne
\draw [line width=2pt] (-22,5)-- (-7,5); %troisième  ligne
\draw [line width=2pt] (-22,4)-- (-7,4); %quatrième ligne
\draw [line width=2pt] (-22,3)-- (-7,3); %cinquième ligne
\draw [line width=2pt] (-22,3)-- (-22,8); %première colonne (-22,4)<----(-22,8);
\draw [line width=2pt] (-18,8)-- (-18,3); %deuxième colone  (-18,8)--->(-18,4);
\draw [line width=2pt] (-15,6)-- (-15,3); %troisième colone (-13,6)-- (-13,4);
%\draw [line width=2pt] (-13,6)-- (-13,3); %quatrième colonne (-13,6)-- (-13,4);
\draw [line width=2pt] (-11,6)-- (-11,3); %cinquième colonne (-13,6)-- (-13,4);
\draw [line width=2pt] (-7,8)-- (-7,3); % colonne (-7,8)-->(-7,4);
\draw (-21,7) node[anchor=north west] {$x$};
\draw (-18,7) node[anchor=north west] {$-\infty$};
\draw (-8,7) node[anchor=north west] {$+\infty$};
\draw (-21,5.7) node[anchor=north west] {$2x+3$};
\draw (-15.8,5.7) node[anchor=north west] {$-$};
\draw (-15.3,5.8) node[anchor=north west] {$O$};
\draw (-15,5.7) node[anchor=north west] {$+$};
\draw (-13,5.7) node[anchor=north west] {$+$};
\draw (-10.5,5.7) node[anchor=north west] {$+$};

\draw (-21,4.7) node[anchor=north west] {$x-1$};
\draw (-15.8,4.7) node[anchor=north west] {$-$};
\draw (-13,4.7) node[anchor=north west] {$-$};
\draw (-15,4.7) node[anchor=north west] {$-$};
\draw (-11.3,4.8) node[anchor=north west] {$O$};
\draw (-10.5,4.7) node[anchor=north west] {$+$};

\draw (-21,4) node[anchor=north west] {$x+1$};
\draw (-15.8,4) node[anchor=north west] {$-$};
\draw (-15,4) node[anchor=north west] {$-$};
\draw (-13,4) node[anchor=north west] {$-$};
\draw (-11.3,4) node[anchor=north west] {$O$};
\draw (-10.5,4) node[anchor=north west] {$+$};

\draw (-15.4,7) node[anchor=north west] {$-\frac{3}{2}$};
\draw (-11.3,7) node[anchor=north west] {$1$};
\end{tikzpicture}

Donc $D=\left]1, +\infty\right[$.

\textbf{\underline{\textcolor{green}{Résolution}}}

$\ln(1-x)-\ln(2x+3)\geq\ln(x-1) \Longrightarrow \ln(1-x) \geq \ln(2x+3)+\ln(x-1)$

$\Longrightarrow \ln(1-x) \geq \ln\left[ (2x+3)(x-1)\right]  \Longrightarrow (1-x) \geq (2x+3)(x-1) \Longrightarrow 1-x \geq 2x^{2}-2x+3x-3$

$\Longrightarrow 2x^{2}+2x-4 \leq 0 $.

Posons $x^{2}+x-1 = 0$

$\Delta=5$

$x_{1}=\frac{-1+\sqrt{5}}{2}$ et $x_{1}=\frac{-1-\sqrt{5}}{2}$

Donc $ S=\left]\frac{-1-\sqrt{5}}{2}, \frac{-1+\sqrt{5}}{2} \right[ \cap \left]1, +\infty\right[=\emptyset$

Donc \textcolor{green}{\boxed{S=\emptyset}}
\section*{\underline{\textcolor{green}{Correction Exercice 2: \textbf{6 pts}}}}

\subsection*{1) Développons } \((x+1)(x-3)(x+2)\)

Pour développer \((x+1)(x-3)(x+2)\), procédons en plusieurs étapes en utilisant la distributivité.

D'abord, développons les deux premiers facteurs :

\[
(x+1)(x-3) = x(x-3) + 1(x-3) = x^2 - 3x + x - 3 = x^2 - 2x - 3
\]

Ensuite, multiplions ce résultat par le troisième facteur \((x+2)\) :

\[
(x^2 - 2x - 3)(x+2) = (x^2 - 2x - 3)x + (x^2 - 2x - 3)2
\]

Développons les deux produits :

\[
(x^2 - 2x - 3)x = x^3 - 2x^2 - 3x
\]
\[
(x^2 - 2x - 3)2 = 2x^2 - 4x - 6
\]

En ajoutant ces deux résultats ensemble, nous obtenons :

\[
x^3 - 2x^2 - 3x + 2x^2 - 4x - 6 = x^3 + (-2x^2 + 2x^2) + (-3x - 4x) - 6 = x^3 - 7x - 6
\]

Donc, le résultat final est :
\textcolor{green}{\boxed{(x+1)(x-3)(x+2) = x^3 - 7x - 6}}
\subsection*{2) résolvons } \(e^{3x}-7e^{x}-6=0\)

Pour résoudre l'équation \(e^{3x} - 7e^{x} - 6 = 0\), faisons un changement de variable. Posons \(y = e^x\). Ainsi, l'équation devient :

\[
e^{3x} = (e^x)^3 = y^3
\]

L'équation se réécrit donc :

\[
y^3 - 7y - 6 = 0
\]

Or, pla forme factorisée de \(y^3 - 7y - 6 \) est  :

\[
(y + 1)(y - 3)(y + 2)
\]

\[\text{Ainsi, }
y^3 - 7y - 6=0 \Longrightarrow (y + 1)(y - 3)(y + 2)=0 \Longrightarrow (e^x + 1)(e^x - 3)(e^x + 2)=0
\]
\[\text{Donc, }
(e^x + 1)(e^x - 3)(e^x + 2)=0 \Longrightarrow e^x + 1=0 \text{ ou } e^x - 3=0 \text{ ou } e^x + 2=0
\]

\[\text{Cela donne les solutions suivantes :} \begin{cases}
e^x=-1 \text{ impossible}\\
e^x=3 \\
e^x=-2 \text{ impossible}
\end{cases}
\Longrightarrow
\begin{cases}
x=\ln 3 
\end{cases}
\]


Donc, la solution de l'équation \(e^{3x} - 7e^x - 6 = 0\) est :

\[
\textcolor{green}{\boxed{S=\ln 3}}
\]

\subsection*{3) résolvons } \(x^{4}-5x^{2}+6=0\) puis \(e^{4x}-5e^{2x}+6=0\)

Pour résoudre \(x^{4} - 5x^{2} + 6 = 0\), faisons un changement de variable. Posons \(y = x^2\). Ainsi, l'équation devient :

\[
y^2 - 5y + 6 = 0
\]

Ici, \(a = 1\), \(b = -5\), et \(c = 6\). Calculons le discriminant :

\[
b^2 - 4ac = (-5)^2 - 4 \times 1 \times 6 = 25 - 24 = 1
\]

Donc, les solutions pour \(y\) sont :

\[
y_{1} = \frac{5 - 1}{2} = 2, y_{2} = \frac{5 + 1}{2} = 3
\]

Ce qui donne :

\[
y_{1} = 2 \quad \text{et } y_{2} = 3
\]

Revenons à la variable \(x\), nous avons \(y = x^2\), donc :

\[
x^2 = 3 \implies x = \pm \sqrt{3}
\]
\[
x^2 = 2 \implies x = \pm \sqrt{2}
\]

Les solutions pour \(x\) sont donc :

\[
x = \pm \sqrt{3}, \quad x = \pm \sqrt{2}
\]

\[
\textcolor{green}{\boxed{S=\left\lbrace -\sqrt{3},\sqrt{3},-\sqrt{2},\sqrt{2}\right\rbrace }}
\]

Maintenant, résolvons \(e^{4x} - 5e^{2x} + 6 = 0\). Faisons un changement de variable similaire. Posons \(z = e^{2x}\). Ainsi, l'équation devient :

\[
z^2 - 5z + 6 = 0
\]

Cette équation est identique à l'équation précédente en \(y\). Les solutions sont :

\[
z = 3 \quad \text{et} \quad z = 2
\]

Revenons à la variable \(x\), nous avons \(z = e^{2x}\), donc :

\[
e^{2x} = 3 \implies 2x = \ln 3 \implies x = \frac{\ln 3}{2}
\]
\[
e^{2x} = 2 \implies 2x = \ln 2 \implies x = \frac{\ln 2}{2}
\]

Les solutions pour \(x\) sont donc :

\[
x = \frac{\ln 3}{2}, x = \frac{\ln 2}{2}
\]

\[
\textcolor{green}{\boxed{S=\left\lbrace \frac{\ln 3}{2},\frac{\ln 2}{2}\right\rbrace }}
\]
\textbf{En résumé, les solutions sont :}

Pour \(x^4 - 5x^2 + 6 = 0\) :
\[
\textcolor{green}{\boxed{S=\left\lbrace -\sqrt{3},\sqrt{3},-\sqrt{2},\sqrt{2}\right\rbrace }}
\]

Pour \(e^{4x} - 5e^{2x} + 6 = 0\) :
\[
\textcolor{green}{\boxed{S=\left\lbrace \frac{\ln 3}{2},\frac{\ln 2}{2}\right\rbrace }}
\]
\subsection*{4) Développons } \((3+x)(2x-1)\) et \((x-2)(3+x)(2x-1)\)

Développons d'abord \((3+x)(2x-1)\) :

\[
(3+x)(2x-1) = 3(2x-1)+ x (2x-1)= 6x - 3 + 2x^2 - x = 2x^2 + 5x - 3
\]
\[
\textcolor{green}{\boxed{(3+x)(2x-1)= 2x^2 + 5x - 3}}
\]
Ensuite, développons \((x-2)(3+x)(2x-1)\). Nous utilisons le résultat précédent :

\[
(x-2)(2x^2 + 5x - 3)
\]

Donc, le résultat final est :
\[
\textcolor{green}{\boxed{(x-2)(3+x)(2x-1) = 2x^3 + x^2 - 13x + 6}}
\]
\subsection*{5) résolvons } \(2e^{-2x}+5e^{-x}-3=0\) et \(2e^{3x+1}+e^{2x+1}-13e^{x+1}+6e=0\)

\textbf{\underline{Pour résoudre \(2e^{-2x} + 5e^{-x} - 3 = 0\)}}:

Faisons un changement de variable. Posons \(y = e^{-x}\). Ainsi, l'équation devient :

\[
2y^2 + 5y - 3 = 0
\]
D'après ce qui précède, $2y^2 + 5y - 3=(3+y)(2y-1)$

Donc $2y^2 + 5y - 3=0 \implies y=-3$ ou $y=\frac{1}{2} $ 

Revenons à la variable \(x\), nous avons \(z = e^{-x}\), donc :

\[
e^{-x} = \frac{1}{2} \implies -x =\ln \frac{1}{2} \implies x =\ln 2
\]

La solution pour \(x\) est donc :

\[
x = \ln 2
\]

\[
\textcolor{green}{\boxed{S=\left\lbrace \ln 2\right\rbrace }}
\]

\textbf{\underline{Pour résoudre \(2e^{3x+1} + e^{2x+1} - 13e^{x+1} + 6e = 0\)}}:

Faisons un changement de variable. Posons \(z = e^{x}\). Ainsi, l'équation devient :

\[
2z^3e + z^2e - 13ze + 6e = 0\implies 2z^3 + z^2 - 13z + 6 = 0
\]
D'après ce qui précède, $2z^3 + z^2 - 13z + 6=(z-2)(3+z)(2z-1)$

Donc $2z^3 + z^2 - 13z + 6=0 \implies z=2$ ou $z=-3 $ ou $z=\frac{1}{2}$

Revenons à la variable \(x\), nous avons \(z = e^{x}\), donc :

\[
e^{x} = 2 \implies x =\ln 2
\]
\[
e^{x} = \frac{1}{2} \implies x =\ln \frac{1}{2} \implies x =-\ln 2
\]
Les solutions pour \(x\) sont donc :

\[
x = \ln 2, x = -\ln 2
\]

\[
\textcolor{green}{\boxed{S=\left\lbrace \ln 2, -\ln 2\right\rbrace }}
\]
\subsection*{6) résolvons dans \(\mathbb{R}^{2}\)} 
\[
\begin{cases}
x + y = 2 \\
\ln x + \ln y = 0
\end{cases}
\]

Pour résoudre ce système d'équations dans \(\mathbb{R}^{2}\), commençons par la deuxième équation. Utilisons la propriété des logarithmes \(\ln x + \ln y = \ln(xy)\) :

\[
\ln(xy) = 0
\]

Donc,

\[
xy = e^0 = 1
\]

Nous avons maintenant le système :

\[
\begin{cases}
x + y = 2 \\
xy = 1
\end{cases}
\]

Pour résoudre ce système, nous pouvons exprimer \(y\) en fonction de \(x\) à partir de la première équation :

\[
y = 2 - x
\]

Substituons cette expression dans la deuxième équation :

\[
x(2 - x) = 1
\]

Ce qui donne :

\[
2x - x^2 = 1 \implies x^2 - 2x + 1 = 0 \implies (x - 1)^2 = 0
\]

La seule solution est :

\[
x = 1
\]

En substituant \(x = 1\) dans \(y = 2 - x\), nous obtenons :

\[
y = 2 - 1 = 1
\]

Donc, la solution du système est :
\[
\textcolor{green}{\boxed{S=\left\lbrace (1, 1)\right\rbrace }}
\]
\section*{\underline{Problème: }\textbf{8 pts}}
Soit $f(x)=\ln(x^{2}-6x+9)$
\begin{itemize}
\item[1)a-] Montrer que l'esemble de définition de $f$ est $Df=\mathbb{R}\setminus\left\lbrace 3 \right\rbrace $ et détermine les limites aux bornes de $Df$.$\textbf{0,5pt+1pt}$

\item[b-] Etuider les variations de $f$.$\textbf{1,5pt}$
\end{itemize}

2)Soit la courbe (Cf) représentative de $f$ dans un repère orthonormé (unité 1 cm).
\begin{itemize}
\item[a-]Déterminer les points d'intersections de $Cf$ avec les axes du repère.$\textbf{1pt}$

\item[b-]Ecrire une équation de la tangente (T) à (Cf) au point d'abscisse 0.$\textbf{0,5pt}$

\item[c-]Montrer que la droite d'équation $x=3$ est axe de  symétrie de (Cf). $\textbf{1pt}$

\item[d-]Tracer (Cf) et la tangente (T). $\textbf{1,5pt}$
\end{itemize}
3) Montrer que $f(x)=2\ln(x-3)$ sur $ \left]3 +\infty \right[ $. $\textbf{1pt}$

\section*{\underline{\textcolor{green}{Correction du problème: \textbf{8 pts}}}}
Soit \( f(x) = f(x)=\ln(x^{2}-6x+9) \).

\subsection*{\textcolor{green}{1) a - Ensemble de définition et limites}}
\subsubsection*{Montrer que l'ensemble de définition de \( f \) est \( D_f = \mathbb{R} \setminus \{3\} \)}
La fonction \( f \) est définie lorsque l'argument du logarithme est strictement positif :
\[
x^{2}-6x+9 > 0
\]

Nous remarquons que :
\[
x^{2}-6x+9 = (x - 3)^2
\]
Donc, l'inéquation devient :
\[
(x - 3)^2 > 0
\]

Ainsi, $\forall x\in \mathbb{R}\setminus \{3\}$ f existe donc
\[
D_f = \mathbb{R} \setminus \{3\}
\]

\subsubsection*{\textcolor{green}{Déterminer les limites aux bornes de \( D_f \)}}
Cherchons la limites à gauche et à droite de 3.

1. Lorsque \( x \to 3^{-} \) (par la gauche) :
\[
\lim_{x \to 3^{-}}\ln (x^{2}-6x+9)=\ln(0^{+})=-\infty
\]
%\[
%\lim_{x \to 3^{-}}\ln (x^{2}-6x+9):
%\begin{cases}
%\lim_{x \to 3^{-}}(x^{2}-6x+9)=0^{+}\\
%\lim_{x \to 0^{-}}(x^{2}-6x+9)=0^{+}\\
%\end{cases}
%\]
Donc,
\[
\lim_{x \to 3^-} f(x) = -\infty
\]

2. Lorsque \( x \to 3^+ \) (par la gauche) :

\[
\lim_{x \to 3^{+}}\ln (x^{2}-6x+9)=\ln(0^{+})=-\infty
\]
%\[
%\lim_{x \to 3^{-}}\ln (x^{2}-6x+9):
%\begin{cases}
%\lim_{x \to 3^{-}}(x^{2}-6x+9)=0^{+}\\
%\lim_{x \to 0^{-}}(x^{2}-6x+9)=0^{+}\\
%\end{cases}
%\]
Donc,
\[
\lim_{x \to 3^+} f(x) = -\infty
\]

1. Lorsque \( x \to -\infty \) :
\[
\lim_{x \to \infty}\ln(x^{2}-6x+9)=\ln(+\infty)=+\infty
\]
Donc,
\[
\lim_{x \to -\infty} f(x) = +\infty
\]

2. Lorsque \( x \to +\infty \) :
\[
\lim_{x \to +\infty}\ln(x^{2}-6x+9)=\ln(+\infty)=+\infty
\]
Donc,
\[
\lim_{x \to +\infty} f(x) = +\infty
\]
\subsection*{\textcolor{green}{b - Étudions les variations de \( f \)}}

Pour étudier les variations de \( f \), nous allons calculer sa dérivée et analyser son signe.

\subsubsection*{\textcolor{green}{Calcul de la dérivée de \( f \)}}
La fonction \( f(x) \) est définie par :
\[
f(x) = \ln(x^{2}-6x+9)
\]
Calculons \( u'(x) \) :
\[
u(x) = x^{2}-6x+9 \implies u'(x) = 2x - 6
\]

Donc, la dérivée de \( f \) est :
\[
f'(x) = \frac{2x - 6}{x^{2}-6x+9}
\]

\subsubsection*{\textcolor{green}{Signe de \( f'(x) \)}}
Analysons le signe de \( f'(x) \) :
\[
f'(x) = \frac{2(x - 3)}{(x - 3)^2}
\]
Le signe de \( f'(x) \) dépend du signe de \( (x - 3) \) :

\(\forall x \in ]-\infty, 3[ \), \( f'(x) < 0 \) est croissante.

\(\forall x \in ]3, +\infty[ \), \( f'(x) > 0 \) est décroissante.

\subsection*{\textcolor{green}{2) Intersection de la courbe \( Cf \) avec les axes}}

\subsubsection*{\textcolor{green}{a - Intersection avec l'axe des ordonnées}}

Pour trouver l'intersection avec l'axe des ordonnées, nous devons évaluer \( f(0) \) :
\[
f(0) = \ln(0^2 - 6 \cdot 0 + 9) = \ln(9) = \ln(3^2) = 2 \ln(3)
\]

Donc, la courbe \( Cf \) intersecte l'axe des ordonnées au point \((0, 2\ln(3))\).

\subsubsection*{\textcolor{green}{b - Intersection avec l'axe des abscisses}}

Pour trouver l'intersection avec l'axe des abscisses, nous devons résoudre \( f(x) = 0 \):
\[
\ln(x^{2}-6x+9) = 0
\]
\[
x^{2}-6x+9 = e^0 = 1
\]
\[
x^{2}-6x+9 - 1 = 0
\]
\[
x^{2}-6x+8 = 0
\]

Résolvons ce trinôme du second degré :

où \( a = 1 \), \( b = -6 \), et \( c = 9 \).

\( \Delta' = 1\)

\[
x = 2
\]
\[
x = 4
\]

Donc, la courbe \( Cf \) intersecte l'axe des abscisses aux points \((2, 0)\) et \((4, 0)\).

\subsection*{\textcolor{green}{2) b - Équation de la tangente à \( Cf \) au point d'abscisse \( 0 \)}}

\subsubsection*{\textcolor{green}{1. Calcul de \( f(0) \)}}
\[
f(0) = \ln(0^2 - 6 \cdot 0 + 9) = \ln(9) = \ln(3^2) = 2 \ln(3)
\]

Le point de tangence est donc \((0, 2 \ln(3))\).

\subsubsection*{\textcolor{green}{2. Calcul de la dérivée \( f'(x) \) et évaluation en \( x = 0 \)}}
La dérivée de \( f \) est :
\[
f'(x) = \frac{2}{x-3}
\]

Évaluons \( f'(x) \) en \( x = 0 \) :
\[
f'(0) = \frac{2}{0 - 3} = -\frac{2}{3}
\]

\subsubsection*{\textcolor{green}{3. Équation de la tangente}}
L'équation de la tangente \( T \) en \( x = 0 \) est donnée par :
\[
y = f(x) + f'(x)(x - x_{0})
\]
\[
y = f(0) + f'(0)(x - 0)
\]
\[
y = 2 \ln(3) - \frac{2}{3} \cdot x
\]
\[
y = - \frac{2}{3} x + 2 \ln(3)
\]

Ainsi, l'équation de la tangente \( T \) à \( Cf \) au point d'abscisse \( 0 \) est :
\[\textcolor{green}{\boxed{
(T):y = - \frac{2}{3} x + 2 \ln(3)
}}\]

\subsection*{\textcolor{green}{2) c - Symétrie de la courbe par rapport à la droite \( x = 3 \)}}

Pour montrer que la droite d'équation \( x = 3 \) est un axe de symétrie de \( Cf \), nous devons prouver que \( f(-x+6) = f(x) \).

Pour ce faire, vérifions si \( \ln(x^{2}-6x+9) = \ln((6-x)^{2}-6(6-x)+9) \).

Calculons \( \ln((6-x)^{2}-6(6-x)+9) \):

\[
(6-x)^{2} = 36 - 12x + x^{2}
\]
\[
-6(6-x) = -36 + 6x
\]
Donc,
\[
(6-x)^{2} - 6(6-x) + 9 = x^{2} - 6x + 9
\]

Ainsi, \( \ln(x^{2}-6x+9) = \ln((6-x)^{2}-6(6-x)+9) \).

Cela montre que \( \ln(x^{2}-6x+9) \) est symétrique par rapport à \( x = 3 \),\\ car pour tout \( x \), \( \ln((6-x)^{2}-6(6-x)+9) = \ln(x^{2}-6x+9) \).

Conclusion : \( \ln(x^{2}-6x+9) \) a comme axe de symétrie la droite verticale \( x = 3 \).
\subsection*{\textcolor{green}{Tableau de variation}}
\definecolor{cqcqcq}{rgb}{0.7529411764705882,0.7529411764705882,0.7529411764705882}
\begin{tikzpicture}[line cap=round,line join=round,>=triangle 45,x=1cm,y=1cm]
%\draw [color=cqcqcq,, xstep=1cm,ystep=1cm] (-7,-10) grid (-22,17);
\clip(-22,-5) rectangle (12,10);
\draw [line width=2pt] (-23,8)-- (-7,8); %première ligne A(-22,8)---B(-7,8)
\draw [line width=2pt] (-22,6)-- (-7,6); %deuxième ligne
\draw [line width=2pt] (-22,4)-- (-7,4); %troisième ligne
\draw [line width=2pt] (-22,-2)-- (-7,-2);%dernière ligne
\draw [line width=2pt] (-22,-2)-- (-22,8); %première colonne
\draw [line width=2pt] (-19,8)-- (-19,-2); %deuxième colone
\draw [line width=2pt] (-13,6)-- (-13,-2); %troisième colonne
\draw [line width=2pt] (-13.1,6)-- (-13.1,-2); %troisième colonne
\draw [line width=2pt] (-7,8)-- (-7,-2); %quatrième colonne
\draw (-21,1.5) node[anchor=north west] {$f(x)$};
\draw (-21,5.5) node[anchor=north west] {$f'(x)$};
\draw (-21,7) node[anchor=north west] {$x$};
\draw (-19,7) node[anchor=north west] {$-\infty$};
\draw (-13.4,6.5) node[anchor=north west] {$3$};
\draw (-8,7) node[anchor=north west] {$+\infty$};
%signe de la dérivé
\draw (-17,5.3) node[anchor=north west] {$-$};
%\draw (-13.39,5.3) node[anchor=north west] {$O$};
\draw (-13.39,5.3) node[anchor=north west] {\textbf{\textcolor{red}{O}}};
\draw (-10,5.3) node[anchor=north west] {$+$};

\draw [->,line width=2pt] (-18,3) -- (-13.5,-1.3);
\draw (-18.8,3.9) node[anchor=north west] {$+\infty$};
%\draw (-13.4,-1) node[anchor=north west] {\textbf{\textcolor{blue}{-4}}};
\draw (-12.9,-1.2) node[anchor=north west] {\textbf{\textcolor{blue}{$-\infty$}}};
\draw (-14.2,-1.2) node[anchor=north west] {\textbf{\textcolor{blue}{$-\infty$}}};
\draw [->,line width=2pt] (-12.5,-1.3) -- (-8,3.5);
\draw (-8,3.9) node[anchor=north west] {$+\infty$};
\end{tikzpicture}
\subsection*{\textcolor{green}{2) d - Tracé de \( Cf \) et de la tangente \( T \)}}

\begin{center}
\begin{tikzpicture}
\begin{axis}[
    axis lines = middle,
    xlabel = $x$,
    ylabel = $y$,
    domain=-6:2,
    samples=100,
    width=12cm,
    height=8cm,
    xtick={-6,-4,-2,0,2},
    ytick={-10, -5, 0, 5, 10},
    legend pos=outer north east,
    grid = major,
    grid style={dashed, gray!30}
]

% Tracé de la courbe Cf
\addplot [
    domain=-6:2, 
    samples=100, 
    color=blue,
    thick
] {ln(x^2 - 6*x + 9)};
\addlegendentry{$f(x) = \ln(x^2 - 6x + 9)$}

% Tracé de la tangente T
\addplot [
    domain=-6:2, 
    samples=100, 
    color=red,
    dashed,
    thick
] {x + 2*ln(2)};
\addlegendentry{Tangente $y =-\frac{2}{3} x + 2\ln(3)$}

% Tracé de la droite x = -2
\addplot [
    domain=-10:10,
    samples=2,
    color=green,
    dashed
] coordinates {(-2,-10) (-2,10)};
\addlegendentry{Symétrie $x = -2$}

\end{axis}
\end{tikzpicture}
\end{center}

\subsection*{\textcolor{green}{3) Preuve que \( f(x) = 2\ln(x-3) \) sur \( \left]-2, +\infty \right[ \)}}

Commençons par simplifier l'expression de \( f(x) \):
\[
f(x) = \ln(x^2 - 6x + 9)
\]

Nous reconnaissons que \( (x^2 - 6x + 9) \) peut être factorisé en un carré parfait :
\[
x^2 - 6x + 9 = (x - 3)^2
\]

Ainsi,
\[
f(x) = \ln((x - 3)^2)
\]

Nous utilisons maintenant la propriété des logarithmes qui dit que \( \ln(a^b) = b \ln(a) \):
\[
f(x) = \ln((x - 3)^2) = 2 \ln(x - 3)
\]

Par conséquent, nous avons montré que :
\[
f(x) = 2 \ln(x - 3)
\]

Il est important de noter que l'expression \( \ln((x - 3)^2) = 2 \ln(x - 3) \) est définie pour \( x - 3 > 0 \), c'est-à-dire \( x > 3 \). Donc, sur l'intervalle \( \left]3, +\infty \right[ \), nous avons :
\[
f(x) = 2 \ln(x - 3)
\]

\end{document}