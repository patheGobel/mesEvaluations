\documentclass[12pt,a4paper]{article}
\usepackage{amsmath,amssymb,mathrsfs,tikz,times,pifont}
\usepackage{enumitem}
\newcommand\circitem[1]{%
\tikz[baseline=(char.base)]{
\node[circle,draw=gray, fill=red!55,
minimum size=1.2em,inner sep=0] (char) {#1};}}
\newcommand\boxitem[1]{%
\tikz[baseline=(char.base)]{
\node[fill=cyan,
minimum size=1.2em,inner sep=0] (char) {#1};}}
\setlist[enumerate,1]{label=\protect\circitem{\arabic*}}
\setlist[enumerate,2]{label=\protect\boxitem{\alph*}}
%%%::::::by chnini ameur :::::::%%%
\everymath{\displaystyle}
\usepackage[left=1cm,right=1cm,top=1cm,bottom=1.7cm]{geometry}
\usepackage{array,multirow}
\usepackage[most]{tcolorbox}
\usepackage{varwidth}
\tcbuselibrary{skins,hooks}
\usetikzlibrary{patterns}
%%%::::::by chnini ameur :::::::%%%
\newtcolorbox{exa}[2][]{enhanced,breakable,before skip=2mm,after skip=5mm,
colback=yellow!20!white,colframe=black!20!blue,boxrule=0.5mm,
attach boxed title to top left ={xshift=0.6cm,yshift*=1mm-\tcboxedtitleheight},
fonttitle=\bfseries,
title={#2},#1,
% varwidth boxed title*=-3cm,
boxed title style={frame code={
\path[fill=tcbcolback!30!black]
([yshift=-1mm,xshift=-1mm]frame.north west)
arc[start angle=0,end angle=180,radius=1mm]
([yshift=-1mm,xshift=1mm]frame.north east)
arc[start angle=180,end angle=0,radius=1mm];
\path[left color=tcbcolback!60!black,right color = tcbcolback!60!black,
middle color = tcbcolback!80!black]
([xshift=-2mm]frame.north west) -- ([xshift=2mm]frame.north east)
[rounded corners=1mm]-- ([xshift=1mm,yshift=-1mm]frame.north east)
-- (frame.south east) -- (frame.south west)
-- ([xshift=-1mm,yshift=-1mm]frame.north west)
[sharp corners]-- cycle;
},interior engine=empty,
},interior style={top color=yellow!5}}
%%%%%%%%%%%%%%%%%%%%%%%
\usepackage{fancyhdr}
\usepackage{lastpage}
\fancyhf{}
\pagestyle{fancy}
\renewcommand{\footrulewidth}{1pt}
\renewcommand{\headrulewidth}{0pt}
\renewcommand{\footruleskip}{10pt}
\fancyfoot[R]{
\color{blue}\ding{45}\ \textbf{Bac 2023}
}
\fancyfoot[L]{
\color{blue}\ding{45}\ \textbf{Prof: math math}
}
\cfoot{\bf
\thepage /
\pageref{LastPage}}
\begin{document}
\renewcommand{\arraystretch}{1.5}
\renewcommand{\arrayrulewidth}{1.2pt}
\begin{tikzpicture}[overlay,remember picture]
\node[draw=blue,line width=1.2pt,fill=purple,text=blue,inner sep=3mm,rounded corners,pattern=dots]at ([yshift=-2.5cm]current page.north) {\begingroup\setlength{\fboxsep}{0pt}\colorbox{white}{\begin{tabular}{|*1{>{\centering \arraybackslash}p{0.28\textwidth}} |*2{>{\centering \arraybackslash}p{0.2\textwidth}|} *1{>{\centering \arraybackslash}p{0.19\textwidth}|} }
\hline
\multicolumn{3}{|c|}{$\diamond$$\diamond$$\diamond$\ \textbf{Lycée Math math math math}\ $\diamond$$\diamond$$\diamond$ }& \textbf{A.S. : 2023/2024} \\ \hline
\textbf{Matière: Mathématiques}& \textbf{Niveau : 4}$ ^\text{\bf e} $\textbf{Maths} &\textbf{Date: 16/3/2023} & \textbf{Durée : 4 heures} \\ \hline
\multicolumn{4}{|c|}{\parbox[c]{7cm}{\begin{center}
\textbf{{\Large\sffamily Devoir de contrôle n$ ^{\circ} $ 2}}
\end{center}}} \\ \hline
\end{tabular}}\endgroup};
\end{tikzpicture}
\vspace{3cm}
\begin{center}
\begin{tcolorbox}
[arc=2mm,outer arc=4mm,width=12cm,
boxrule=1.2pt,left=1mm,right=1mm,leftrule=5pt,rightrule=5pt,
titlerule=0mm,toptitle=0mm,bottomtitle=0mm,top=1mm,
colframe=red,colback=white,coltitle=black,
]
\centering \bf\textbf{NB}: ce document contient 4 exercices
\end{tcolorbox}
\end{center}
\vskip3mm
\begin{exa}[colbacktitle=green]{Exercice 1 :5 points}
Soit $f$ la fonction définie sur $] 0,+\infty$ [ par $f(x)=\dfrac{\ln (x)}{\ln (x+1)}$.\\
Soit $ C $ la courbe de $f$ dans un repère orthonormée $(O, \vec{i}, \vec{j})$.
\begin{enumerate}
\item
\begin{enumerate}
\item Calculer $\lim _{x \to 0^{+}} f(x)$.Interpréter graphiquement le résultat.
\item Vérifier que $\forall x>0, \ln (x+1)=\ln (x)+\ln \left(1+\dfrac{1}{x}\right)$.
\item Déduire que $\lim _{x \to +\infty} f(x)=1$. Interpréter le résultat.
\end{enumerate}
\item
\begin{enumerate}
\item Montrer que $\forall x>0, f'(x)=\dfrac{x(\ln (x+1)-\ln (x))+\ln (x+1)}{x(x+1) \ln ^{2}(x+1)}$.
\item En déduire que $f$ est strictement croissante sur $] 0,+\infty[$.
\item Dresser le tableau de variation de la fonction $f$.
\item Tracer la courbe $ C $ en précisant son intersection avec l'axe des abscisses.
\end{enumerate}
\item Montrer que $f$ admet une réciproque $f^{-1}$ définie sur $] -\infty, 1[$.
\item Pour tout entier naturel $n \geq 2$, on pose $a_{n}=f^{-1}\left(\dfrac{1}{n}\right)$.
\begin{enumerate}
\item Calculer $\lim {n \to +\infty} a{n}$.
\item Montrer que $a_{n}$ est une solution de l'équation $x^{n}=x+1$.
\item Calculer $\lim {n \to +\infty}\left(a{n}\right)^{n}$.
\end{enumerate}
\end{enumerate}
\end{exa}
\begin{exa}[colbacktitle=green]{Exercice 2 :5pts}
Soit $f$ la fonction définie sur $] 0,+\infty$ [ par $f(x)=\dfrac{\ln (x)}{\ln (x+1)}$.\\
Soit $ C $ la courbe de $f$ dans un repère orthonormée $(O, \vec{i}, \vec{j})$.
\begin{enumerate}
\item
\begin{enumerate}
\item Calculer $\lim _{x \to 0^{+}} f(x)$.Interpréter graphiquement le résultat.
\item Vérifier que $\forall x>0, \ln (x+1)=\ln (x)+\ln \left(1+\dfrac{1}{x}\right)$.
\item Déduire que $\lim _{x \to +\infty} f(x)=1$. Interpréter le résultat.
\end{enumerate}
\item
\begin{enumerate}
\item Montrer que $\forall x>0, f'(x)=\dfrac{x(\ln (x+1)-\ln (x))+\ln (x+1)}{x(x+1) \ln ^{2}(x+1)}$.
\item En déduire que $f$ est strictement croissante sur $] 0,+\infty[$.
\item Dresser le tableau de variation de la fonction $f$.
\item Tracer la courbe $ C $ en précisant son intersection avec l'axe des abscisses.
\end{enumerate}
\item Montrer que $f$ admet une réciproque $f^{-1}$ définie sur $] -\infty, 1[$.
\item Pour tout entier naturel $n \geq 2$, on pose $a_{n}=f^{-1}\left(\dfrac{1}{n}\right)$.
\begin{enumerate}
\item Calculer $\lim {n \to +\infty} a{n}$.
\item Montrer que $a_{n}$ est une solution de l'équation $x^{n}=x+1$.
\item Calculer $\lim {n \to +\infty}\left(a{n}\right)^{n}$.
\end{enumerate}
\end{enumerate}
\end{exa}
\end{document}