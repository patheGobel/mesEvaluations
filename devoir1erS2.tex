\documentclass{article}
\usepackage{amsmath}
\usepackage{amssymb}
\usepackage[margin=2cm]{geometry}

\begin{document}

\begin{minipage}{0.5\textwidth}
	Ministère de l'éducation nationale  \\
	Inspection académique de Kédougou   \\
	Lycée de Dindéfelo            \\
	Cellule de mathématiques            \\
	M. BA                          \\
	Classe : $1erS$  \\
\end{minipage}
\begin{minipage}{0.5\textwidth}
	Année scolaire 2023-2024 \\
	Date : 27-04-2024 \\
	Durée : 2h 00 \\
\end{minipage}

\begin{center}
	\textbf{{\underline{Devoir N1 Du Second Semestre}}}
\end{center}

\section*{Exercice 1 (6 points) :}
\subsection*{a) Calculer les limites suivantes (1 pt $\times$ 4)}
\[ \lim_{x \to +\infty}-\frac{8}{5}x^{3}+\frac{7}{2}x+8\quad\quad \lim_{x \to -\infty}\frac{5x+3}{x^{2}-4x+1}\quad\quad \lim_{x \to 3}\frac{x^{2}-9}{x-3}\quad\quad \lim_{x \to \frac{1}{5}}\frac{x-2\sqrt{x}}{1-5x} \]
\subsection*{b) Etudier la continuité de f sur son $D_{f}$: (1 pt $\times$ 2)}
\[ f(x) = \begin{cases} 
  \frac{x^2-1}{x-1}, & \text{si } x < 1 \\
   2x,\quad\quad\quad\quad\quad \text{si } 1 \leq x \leq 2\\
  \sqrt{x+2}, & \text{si } x > 2
\end{cases} \]
\section*{Exercice 2 (4 points) :}
\subsection*{a) Etudier la dérivabilté de $f$ sur domaine de définition $D_{f}$}
\[ f(x) = \begin{cases} 
  3x^{2}, & \text{si } x \leq 0 \\
  -2x^{2}, & \text{si } x > 0 
\end{cases} \]
\subsection*{b) Soit $f(x)=|x|$ }
Montrer que $f$ est continue en 0 mais n'est pas dérivable en 0 
\section*{Exercice 3 (6 points) :}
\subsection*{ Dans chacun des cas suivants, calculer la dérivé de $f$}
\begin{itemize}
\item[a)]$f(x)=3x^{2}-5x+2$\quad\quad $b)f(x)=\frac{1}{2}x^{4}+\frac{5}{3}x^{3}-x^{2}-5x$\quad\quad $c)f(x)=(x^{2}+1)(5x-7)$
\item[d)]$d)f(x)=\frac{-2x+1}{3x+5}$\quad\quad $e)f(x)=x+3+\frac{4}{x-1}$\quad\quad $f)f(x)=\sqrt{2x+1}$
\end{itemize}
\section*{Exercice 4 (4 points) :}
\subsection*{a) Dans chaque cas donner la mesure principale de $\alpha$ (0,75 pt $\times$ 2+0,5 )}
\[\alpha =\frac{129\pi}{8}\quad\quad \alpha =\frac{108\pi}{7}\quad\quad \alpha = -26\pi \]
\subsection*{a) Compléter les formules (0,75 pt $\times$ 2+0,5 )}
\[\cos(\pi+x)=\cdots\quad\quad \sin(\pi+x)=\cdots\quad\quad \cos(\frac{\pi}{2}-x)=\cdots\]
\end{document}