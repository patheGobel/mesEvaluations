\documentclass[12pt]{article}
\usepackage{stmaryrd}
\usepackage{graphicx}
\usepackage[utf8]{inputenc}

\usepackage[french]{babel}
\usepackage[T1]{fontenc}
\usepackage{hyperref}
\usepackage{verbatim}

\usepackage{color, soul}

\usepackage{pgfplots}
\pgfplotsset{compat=1.15}
\usepackage{mathrsfs}

\usepackage{amsmath}
\usepackage{amsfonts}
\usepackage{amssymb}
\usepackage{tkz-tab}

\usepackage{tikz}
\usetikzlibrary{arrows, shapes.geometric, fit}


\usepackage[margin=2cm]{geometry}

\begin{document}

\begin{minipage}{0.5\textwidth}
	Ministère de l'éducation nationale  \\
	Inspection académique de Kédougou   \\
	Lycée de Dindéferlo            \\
	Cellule de mathématiques            \\
	M. BA                          \\
	Classe : TS  \\
\end{minipage}
\begin{minipage}{0.5\textwidth}
	Année scolaire 2023-2024 \\
	Date : 10-06-2024 \\
	Durée : 4h 00 \\
\end{minipage}

\begin{center}
	\textbf{{\underline{\textcolor{green}{Composition Du Second Semestre}}}}
\end{center}

\section*{\textcolor{red}{\underline{Exercice 1} (4 points) :}}
Le plan complexe est muni d’un repère orthonormé $(O,\vec{u},\vec{v})$
On considère les points A et B d’affixes respectives a et b tels que

$a=-(\frac{1+\sqrt{3}}{2})(1-i)$ et $b=-(\frac{1+\sqrt{3}}{2})(1+i)$ 

1) Ecrire a et b sous forme trigonométrique.\textbf{1pt}

2) Quelle est la nature du triangle OAB ?\textbf{1pt}

3) On désigne par C le point d’affixe 1.

a) Ecrire le nombre complexe $\frac{1-a}{b-a}$  sous forme trigonométrique.\textbf{1pt}

b) En déduire la nature du triangle ABC et  utiliser ce résultat pour placer les points A, B, C sur la figure.\textbf{1pt}
\section*{\textcolor{green}{\underline{Correction Exercice 1} (4 points) :}}
$a=-(\frac{1+\sqrt{3}}{2})(1-i)$ et $b=-(\frac{1+\sqrt{3}}{2})(1+i)$ 

1) Ecrivons a et b sous forme trigonométrique

$a=-(\frac{1+\sqrt{3}}{2})\left[\sqrt{2}(\frac{\sqrt{2}}{2}-i\frac{\sqrt{2}}{2})\right]=-\sqrt{2}(\frac{1+\sqrt{3}}{2})\left[(\frac{\sqrt{2}}{2}-i\frac{\sqrt{2}}{2})\right]=-\sqrt{2}(\frac{1+\sqrt{3}}{2})\left[\cos(\frac{\pi}{4})-i\sin(\frac{\pi}{4})\right] $

\[\color{green}{\boxed{\text{Donc, }a=-\sqrt{2}\left( \frac{1+\sqrt{3}}{2}\right) \left[\cos(\frac{\pi}{4})-i\sin(\frac{\pi}{4})\right]} }\]

$b=-(\frac{1+\sqrt{3}}{2})\left[\sqrt{2}(\frac{\sqrt{2}}{2}+i\frac{\sqrt{2}}{2})\right]=-\sqrt{2}(\frac{1+\sqrt{3}}{2})\left[(\frac{\sqrt{2}}{2}+i\frac{\sqrt{2}}{2})\right]=-\sqrt{2}(\frac{1+\sqrt{3}}{2})\left[\cos(\frac{\pi}{4})+i\sin(\frac{\pi}{4})\right] $

\[\color{green}{\boxed{\text{Donc, }b=-\sqrt{2}\left( \frac{1+\sqrt{3}}{2}\right) \left[\cos(\frac{\pi}{4})+i\sin(\frac{\pi}{4})\right]}}\]
2)Donnons la nature du triangle nature du triangle OAB.

$|OB|=\sqrt{2}\left( \frac{1+\sqrt{3}}{2}\right)$

$|OA|=\sqrt{2}\left( \frac{1+\sqrt{3}}{2}\right)$

$|AB|=|-\sqrt{2}(\frac{1+\sqrt{3}}{2})\left[2\cos(\frac{\pi}{4})\right]|=|-\sqrt{2}(\frac{1+\sqrt{3}}{2})\sqrt{2}|=|(\frac{1+\sqrt{3}}{2})|$

Donc le triangle OAB est un triangle isocèle en O.

3) 

a) Ecrivons le nombre complexe $\frac{1-a}{b-a}$  sous forme trigonométrique.

l'idée est d'écrire $a$ et $b$ sous forme exponentielle \\

\[a=-\sqrt{2}\left( \frac{1+\sqrt{3}}{2}\right) \left[\cos(\frac{\pi}{4})-i\sin(\frac{\pi}{4})\right]=-\sqrt{2}\left( \frac{1+\sqrt{3}}{2}\right)e^{-i\frac{\pi}{4}}\]\\
\[b=-\sqrt{2}\left( \frac{1+\sqrt{3}}{2}\right) \left[\cos(\frac{\pi}{4})+i\sin(\frac{\pi}{4})\right]=-\sqrt{2}\left( \frac{1+\sqrt{3}}{2}\right)e^{i\frac{\pi}{4}}\]
$\frac{1-a}{b-a}=\frac{1+(\frac{1+\sqrt{3}}{2})(1-i)}{(\frac{1+\sqrt{3}}{2})(1-i)+(\frac{1+\sqrt{3}}{2})(1+i)}=
\frac{\left( \frac{3+\sqrt{3}}{2}\right) -i\left(\frac{1+\sqrt{3}}{2}\right)}{-i\left( 1+\sqrt{3}\right) }$


Donc $\frac{1-a}{b-a}=\frac{\left( \frac{3+\sqrt{3}}{2}\right) -i\left(\frac{1+\sqrt{3}}{2}\right)}{-i\left( 1+\sqrt{3}\right)}=\frac{\left(\frac{1+\sqrt{3}}{2}\right)+\left(\frac{3+\sqrt{3}}{2}\right)i}{1+\sqrt{3}}=\frac{1}{2}+\frac{\sqrt{3}}{2}i$

\[\text{Donc, }\color{green}{\boxed{\frac{1-a}{b-a}=\cos(\frac{\pi}{3})+i\sin(\frac{\pi}{3})}}\]
b)Comme arg($\frac{1-a}{b-a})$=$\frac{\pi}{3}[2\pi]$ donc le triangle ABC best un triangle équilatérale
%$\frac{1-a}{b-a}=\frac{1+\sqrt{2}\left( \frac{1+\sqrt{3}}{2}\right)e^{-i\frac{\pi}{4}}}{-\sqrt{2}\left( \frac{1+\sqrt{3}}{2}\right)e^{i\frac{\pi}{4}}+\sqrt{2}\left( \frac{1+\sqrt{3}}{2}\right)e^{-i\frac{\pi}{4}}}=\frac{1+\sqrt{2}\left( \frac{1+\sqrt{3}}{2}\right)e^{-i\frac{\pi}{4}}}{\sqrt{2}\left( \frac{1+\sqrt{3}}{2}\right)\left[e^{-i\frac{\pi}{4}}-e^{i\frac{\pi}{4}}\right]}=\frac{1+\sqrt{2}\left( \frac{1+\sqrt{3}}{2}\right)e^{-i\frac{\pi}{4}}}{\sqrt{2}\left( \frac{1+\sqrt{3}}{2}\right)\left[-i\sqrt{2}\right]}=\frac{i+\sqrt{2}\left( \frac{1+\sqrt{3}}{2}\right)e^{-i\frac{\pi}{4}}.i}{\left(1+\sqrt{3}\right)}$

%$\frac{1-a}{b-a}=\frac{e^{\frac{i\pi}{2}}+\sqrt{2}\left( \frac{1+\sqrt{3}}{2}\right)e^{-i\frac{\pi}{4}}.e^{\frac{i\pi}{2}}}{1+\sqrt{3}}=\frac{e^{\frac{i\pi}{2}}+\sqrt{2}\left( \frac{1+\sqrt{3}}{2}\right)e^{-i\frac{\pi}{2}}}{\left(1+\sqrt{3}\right)}$
%$\frac{1-a}{b-a}=\frac{1+(\frac{1+\sqrt{3}}{2})(1-i)}{-(\frac{1+\sqrt{3}}{2})(1+i)+(\frac{1+\sqrt{3}}{2})(1-i)}=\frac{}{(\frac{1+\sqrt{3}}{2})}$
\section*{\textcolor{red}{\underline{Exercice 2} (6 points) :}}
\subsection*{Partie A: (2pts)}
Une urne contient trois boules jaunes, cinq boules rouges et deux boules vertes.

1) On tire simultanément trois boules de l'urne.

a) Quel est le nombre de tirages unicolores ? \textbf{0,5pt}

b)Quel est le nombre de tirages comportant exactement deux boules de même couleur ? \textbf{0,5pt}

2) On tire successivement sans remise trois boules.

a) Quel est le nombre de tirages comportant des boules rouges uniquement ? \textbf{0,5pt}

b)Quel est le nombre de tirages ne comportant pas de boule verte aux deuxième tirage ?\textbf{0,5pt}
\subsection*{Partie B: (4pts)}
Un porte-monnaie contient quatre pièce de \textbf{500F} et une pièce de \textbf{200F}.

Un enfant tire au hasard 3 pièces de ce porte-monnaie.

1)Calculer la probabilité de l'évènement A : << tiré 3 pièce de \textbf{500F} >>.\textbf{0,5pt}

2)Soit X la variable aléatoire égale au nombre de pièce de \textbf{500F} figurant parmis les 3 pièces tirées.

Déterminer la loi de probabilité de X puis représenter la fonction de répartition de X.\textbf{1pt+1pt}

3)Calculer l'espérance mathématique et l'écart-type. \textbf{0,5pt+0,5pt}

4)L'enfant rèpète 5 fois l'expérience A en remettant à chaque fois les trois pièces tirées.

Quelle est la probabilité que l'évenement A se réalise tois fois a l'issu des 5 tirages ? \textbf{0,5pt}

L'enfant efectue n fois cette épreuve. Détermine la plus petite valeur de n pour que la probabilité d'obtenir au moins une fois l'évènement A soit supérieur à 0,99. \textbf{1pt}
\section*{\textcolor{green}{\underline{Correction Exercice 2} (6 points) :}}
\subsection*{Partie A: (2pts)}

\subsection*{1) On tire simultanément trois boules de l'urne.}

\subsubsection*{a) Le nombre de tirages unicolores : \textbf{0,5 pt}}

Pour qu'un tirage soit unicolore, les trois boules tirées doivent être de la même couleur.

- Nombre de façons de tirer 3 boules jaunes (il n'y a que 3 boules jaunes) :
  \[
  C_{3}^{3} = 1
  \]

- Nombre de façons de tirer 3 boules rouges parmi 5 boules rouges :
  \[
  C_{5}^{3} = \frac{5!}{3!(5-3)!} = 10
  \]

- Nombre de façons de tirer 3 boules vertes (il n'y a que 2 boules vertes) :
  \[
  C_{2}^{3} = 0
  \]

Ainsi, le nombre total de tirages unicolores est :
\[
1 + 10 + 0 = 11
\]
\textbf{\color{green}{Autrement dit :}}

Soit A:<<Tirer 3 boules jaunes parmis les jaunes, tirer 3 boules rouges parmis les rouges et tirer 3 boules vertes par les vertes>>

\[\text{Donc, }\color{green}{\boxed{\text{card(A)=}  C_{3}^{3} + C_{5}^{3} + C_{2}^{3}=11}}\]
\subsubsection*{b) Le nombre de tirages comportant exactement deux boules de même couleur : \textbf{0,5 pt}}

Pour qu'un tirage comporte exactement deux boules de même couleur, nous devons considérer les différentes combinaisons possibles de deux boules de même couleur et une boule d'une autre couleur.

- **Deux boules jaunes et une autre boule** :

  - Nombre de façons de tirer 2 boules jaunes parmi 3 boules jaunes :
    \[
    \binom{3}{2} = 3
    \]
  - Nombre de façons de tirer 1 boule parmi les 7 autres boules (5 rouges et 2 vertes) :
    \[
    \binom{7}{1} = 7
    \]
  - Total pour cette combinaison :
    \[
    3 \times 7 = 21
    \]

- **Deux boules rouges et une autre boule** :

  - Nombre de façons de tirer 2 boules rouges parmi 5 boules rouges :
    \[
    \binom{5}{2} = 10
    \]
  - Nombre de façons de tirer 1 boule parmi les 5 autres boules (3 jaunes et 2 vertes) :
    \[
    \binom{5}{1} = 5
    \]
  - Total pour cette combinaison :
    \[
    10 \times 5 = 50
    \]

- **Deux boules vertes et une autre boule** :

  - Nombre de façons de tirer 2 boules vertes parmi 2 boules vertes :
    \[
    \binom{2}{2} = 1
    \]
  - Nombre de façons de tirer 1 boule parmi les 8 autres boules (3 jaunes et 5 rouges) :
    \[
    \binom{8}{1} = 8
    \]
  - Total pour cette combinaison :
    \[
    1 \times 8 = 8
    \]

Ainsi, le nombre total de tirages comportant exactement deux boules de même couleur est :
\[
21 + 50 + 8 = 79
\]
\textbf{\color{green}{Autrement dit :}}

Soit $B$ : << Deux boules jaunes et une autre boule, ou deux boules rouges et une autre boule, ou deux boules vertes et une autre boule >>.

\[\text{card(B)=}C_{3}^{2}\times C_{7}^{1} + C_{5}^{2}\times C_{5}^{1} + C_{2}^{2}\times C_{8}^{1}\]
\[\text{card(B)=}\frac{3!}{(3-2)!2!}\times 7 +\frac{5!}{(5-2)!2!}\times 5 + \frac{2!}{(2-2)!2!}\times 8\]
\[\text{Donc, }\color{green}{\boxed{\text{card(A)=}3\times 7 +10\times 5 + 1\times 8=79}}\]

\subsection*{2) On tire successivement sans remise trois boules.}

\subsubsection*{a) Le nombre de tirages comportant des boules rouges uniquement :}

Pour que les trois boules tirées soient rouges, nous devons choisir 3 boules rouges parmi les 5 disponibles. Comme l'ordre de tirage est important, nous devons calculer les arrangements de 5 objets pris 3 à 3 :

\[
A_{5}^{3} = \frac{5!}{(5-3)!} = 5 \times 4 \times 3 = 60
\]

Ainsi, le nombre de tirages comportant uniquement des boules rouges est de 60.

\textbf{\color{green}{Autrement dit :}}

Soit $C$ : << Deux boules jaunes et une autre boule, ou deux boules rouges et une autre boule, ou deux boules vertes et une autre boule >>.
\[\text{card(C)=}A_{5}^{3} = \frac{5!}{(5-3)!} = 5 \times 4 \times 3 = 60\]
\[\text{Donc, }\color{green}{\boxed{\text{card(C)=}60}}\]
\subsubsection*{b) Le nombre de tirages ne comportant pas de boule verte au deuxième tirage :}

Pour que la deuxième boule ne soit pas verte, nous devons considérer deux cas : tirer une boule jaune ou rouge en deuxième position.
\begin{itemize}
\item **Cas 1 : La première boule est jaune (3 possibilités)** :
	\begin{itemize}
  	\item La deuxième boule peut être jaune ou rouge (7 possibilités : 2 jaunes restantes + 5 rouges).
  	\[
		A_{3}^{1}\times A_{7}^{1}
	\]
  	\item La troisième boule peut être n'importe laquelle des 8 restantes (1 jaune, 5 rouges, 2 vertes).
	 \[
		A_{8}^{1}
	\]
	\end{itemize}
\item **Cas 2 : La première boule est rouge (5 possibilités)** :
	\begin{itemize}
 	 	\item La deuxième boule peut être jaune ou rouge (7 possibilités : 3 jaunes + 4 rouges restantes).
 	 	 \[
			A_{5}^{1}\times A_{7}^{1}
		\]
  		\item La troisième boule peut être n'importe laquelle des 8 restantes (3 jaunes, 4 rouges, 1 verte).
  		\[
			A_{8}^{1}
		\]
\end{itemize}
\end{itemize}
Le total des tirages ne comportant pas de boule verte au deuxième tirage est donc :
\[
A_{3}^{1}\times A_{7}^{1}\times A_{8}^{1} + A_{5}^{1}\times A_{7}^{1} \times A_{8}^{1}=3 \times 7 \times 8 + 5 \times 7 \times 8 = 448
\]

Ainsi, le nombre de tirages ne comportant pas de boule verte au deuxième tirage est de 448.

\textbf{\color{green}{Autrement dit :}}

Soit $D$ : << La première boule est jaune ou La première boule est verte >>.

\[\text{Card(D)=}
A_{3}^{1}\times A_{7}^{1}\times A_{8}^{1} + A_{5}^{1}\times A_{7}^{1} \times A_{8}^{1}
\]
\[\text{Donc, }\color{green}{\boxed{\text{card(D)=}3 \times 7 \times 8 + 5 \times 7 \times 8 = 448}}\]
\subsection*{Partie B: (4pts)}
Un porte-monnaie contient quatre pièce de \textbf{500F} et une pièce de \textbf{200F}.

Un enfant tire au hasard 3 pièces de ce porte-monnaie.

1)Calculons la probabilité de l'évènement A : << tiré 3 pièce de \textbf{500F} >>.\textbf{0,5pt}

\[card(A)=4 \text{ et } card(\Omega)=5\]

\[p(A)=\frac{card(A)}{card(\Omega)}=\frac{4}{5}\]

\[\text{Donc, }\color{green}{\boxed{\text{p(A)=}\frac{4}{5}}}\]

\section*{\textcolor{red}{\underline{Problème} (10 points) :}}
\subsection*{\underline{Partie A:} (2 pts)}
On considère l'équation différentielle \textbf{(E)} : $\frac{1}{2}y' + y = 3e^{-2x} + 2$

\begin{itemize}
    \item[1.] Déterminer \textbf{a} pour que la fonction $v$ définie par $v(x) = \textbf{a}xe^{-2x} + 2$ soit une solution de l'équation \textbf{(E')}. \textbf{0,5 pt}
    \item[2.] Donner les solutions de l'équation \textbf{(E')} : $\frac{1}{2}y' + y = 0$. \textbf{0,5 pt}
    \item[3.] 
    \begin{itemize}
        \item[a)] Montrer que $u$ est une solution de \textbf{(E)} si et seulement si $v - u$ est solution de \textbf{(E')}. \textbf{0,5 pt}
        \item[b)] En déduire les solutions de \textbf{(E)}. \textbf{0,25 pt}
    \end{itemize}
    \item[4.] Déterminer la solution $u$ de l'équation \textbf{(E)} vérifiant $u(0) = 0$. \textbf{0,25 pt}
\end{itemize}
\subsection*{\underline{Partie B:} (8 pts)}
On définit la fonction $f$ sur $\mathbb{R}$ par :
\[
f(x) = \begin{cases} 
  2(3x - 1)e^{-2x} + 2, & \text{si } x \leq 0 \\
  \frac{x\ln x}{1 + x}, & \text{si } x > 0 
\end{cases}
\]
On note $C_{f}$ sa courbe représentative dans un repère d'unité graphique 4 cm.

\begin{itemize}
    \item[I.] On définit la fonction $g$ sur $\left]0 ; \infty\right[$ par $g(x) = 1 + x + \ln(x)$
    \begin{itemize}
        \item[1.] 
        \begin{itemize}
            \item[a)] Calculer les limites de $g$ en 0 et en $+\infty$. \textbf{0,25 pt + 0,25 pt}
            \item[b)] Étudier le sens de variation de $g$. \textbf{0,5 pt}
            \item[c)] Dresser le tableau de variation de $g$. \textbf{0,5 pt}
        \end{itemize}
        \item[2.] Montrer que l'équation $g(x) = 0$ admet une unique solution $\alpha$ dans $\left]0 ; \infty\right[$. En déduire que $\alpha \in \left]0,2 ; 0,3\right[$. \textbf{0,5 pt}
        \item[3.] Déduire le signe de $g(x)$ suivant les valeurs de $x$ sur $\left]0 ; \infty\right[$. \textbf{0,25 pt}
    \end{itemize}
    \item[II.] Étude de la fonction $f$
    \begin{itemize}
        \item[1.] Étudier la continuité et la dérivabilité de $f$ en 0 puis interpréter graphiquement les résultats. \textbf{1 pt}
        \item[2.] Étudier les limites de $f$ en $+\infty$ et en $-\infty$. \textbf{0,25 pt + 0,25 pt}
        \item[3.] Étudier les branches infinies en l'infini. \textbf{0,5 pt + 0,5 pt}
        \item[4.] Étudier le sens de variation de $f$ sur $\mathbb{R}$ (on montrera que pour tout $x > 0$, $f'(x) = \frac{g(x)}{(x + 1)^{2}}$). \textbf{0,5 pt}
        \item[5.] Dresser le tableau de variation de $f$. \textbf{0,5 pt}
        \item[6.] Montrer que $f(\alpha) = -\alpha$ puis déterminer l'intersection de $C_f$ avec les axes. \textbf{0,5 pt}
        \item[7.] Tracer $C_f$. \textbf{0,75 pt}
    \end{itemize}
\end{itemize}
\section*{\underline{Partie C:}}
Soit $\beta < 0$, on note $A(\beta)$ l'aire de la partie du plan délimitée par les droites d'équations $x = \beta$, $x = 0$, $y = 0$ et la courbe $C_f$.

\begin{itemize}
    \item[1.] On pose $F(x) = (ax + b)e^{-2x}$. Déterminer $a$ et $b$ pour que $F'(x) = (3x - 1)e^{-2x}$. \textbf{0,5 pt}
    \item[2.] Calculer $A(\beta)$. \textbf{0,5 pt}
    \item[3.] Calculer sa limite en $-\infty$. \textbf{0,25 pt}
\end{itemize}
\section*{\textcolor{green}{\underline{Correction Problème} (10 points) :}}
\end{document}