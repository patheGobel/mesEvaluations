\documentclass[12pt]{article}
\usepackage{stmaryrd}
\usepackage{graphicx}
\usepackage[utf8]{inputenc}

\usepackage[french]{babel}
\usepackage[T1]{fontenc}
\usepackage{hyperref}
\usepackage{verbatim}

\usepackage{color, soul}

\usepackage{pgfplots}
\pgfplotsset{compat=1.15}
\usepackage{mathrsfs}

\usepackage{amsmath}
\usepackage{amsfonts}
\usepackage{amssymb}
\usepackage{tkz-tab}

\usepackage{tikz}
\usetikzlibrary{arrows, shapes.geometric, fit}


\usepackage[margin=2cm]{geometry}

\begin{document}

\begin{minipage}{0.5\textwidth}
	Ministère de l'éducation nationale  \\
	Inspection académique de Kédougou   \\
	Lycée de Dindéfélo            \\
	Cellule de mathématiques            \\
	M. BA                          \\
	Classe : TS  \\
\end{minipage}
\begin{minipage}{0.5\textwidth}
	Année scolaire 2023-2024 \\
	Date : 10-06-2024 \\
	Durée : 4h 00 \\
\end{minipage}

\begin{center}
	\textbf{{\underline{\textcolor{green}{Composition Du Second Semestre}}}}
\end{center}

\section*{\textcolor{red}{\underline{Exercice 1} (4 points) :}}
Le plan complexe est muni d’un repère orthonormé $(O,\vec{u},\vec{v})$
On considère les points A et B d’affixes respectives a et b tels que

$a=-(\frac{1+\sqrt{3}}{2})(1-i)$ et $b=-(\frac{1+\sqrt{3}}{2})(1+i)$ 

1) Ecrire a et b sous forme trigonométrique.\textbf{1pt}

2) Quelle est la nature du triangle OAB ?\textbf{1pt}

3) On désigne par C le point d’affixe 1.

a) Ecrire le nombre complexe $\frac{1-a}{b-a}$  sous forme trigonométrique.\textbf{1pt}

b) En déduire la nature du triangle ABC et  utiliser ce résultat pour placer les points A, B, C sur la figure.\textbf{1pt}
\section*{\textcolor{green}{\underline{Correction Exercice 1} (4 points) :}}
$a=-(\frac{1+\sqrt{3}}{2})(1-i)$ et $b=-(\frac{1+\sqrt{3}}{2})(1+i)$ 

1) Ecrivons a et b sous forme trigonométrique

$a=-(\frac{1+\sqrt{3}}{2})\left[\sqrt{2}(\frac{\sqrt{2}}{2}-i\frac{\sqrt{2}}{2})\right]=-\sqrt{2}(\frac{1+\sqrt{3}}{2})\left[(\frac{\sqrt{2}}{2}-i\frac{\sqrt{2}}{2})\right]=-\sqrt{2}(\frac{1+\sqrt{3}}{2})\left[\cos(\frac{\pi}{4})-i\sin(\frac{\pi}{4})\right] $

\[\color{green}{\boxed{\text{Donc, }a=-\sqrt{2}\left( \frac{1+\sqrt{3}}{2}\right) \left[\cos(\frac{\pi}{4})-i\sin(\frac{\pi}{4})\right]} }\]

$b=-(\frac{1+\sqrt{3}}{2})\left[\sqrt{2}(\frac{\sqrt{2}}{2}+i\frac{\sqrt{2}}{2})\right]=-\sqrt{2}(\frac{1+\sqrt{3}}{2})\left[(\frac{\sqrt{2}}{2}+i\frac{\sqrt{2}}{2})\right]=-\sqrt{2}(\frac{1+\sqrt{3}}{2})\left[\cos(\frac{\pi}{4})+i\sin(\frac{\pi}{4})\right] $

\[\color{green}{\boxed{\text{Donc, }b=-\sqrt{2}\left( \frac{1+\sqrt{3}}{2}\right) \left[\cos(\frac{\pi}{4})+i\sin(\frac{\pi}{4})\right]}}\]
2)Donnons la nature du triangle nature du triangle OAB.

$|OB|=\sqrt{2}\left( \frac{1+\sqrt{3}}{2}\right)$

$|OA|=\sqrt{2}\left( \frac{1+\sqrt{3}}{2}\right)$

$|AB|=|-\sqrt{2}(\frac{1+\sqrt{3}}{2})\left[2\cos(\frac{\pi}{4})\right]|=|-\sqrt{2}(\frac{1+\sqrt{3}}{2})\sqrt{2}|=|(\frac{1+\sqrt{3}}{2})|$

Donc le triangle OAB est un triangle isocèle en O.

3) 

a) Ecrivons le nombre complexe $\frac{1-a}{b-a}$  sous forme trigonométrique.

l'idée est d'écrire $a$ et $b$ sous forme exponentielle \\

\[a=-\sqrt{2}\left( \frac{1+\sqrt{3}}{2}\right) \left[\cos(\frac{\pi}{4})-i\sin(\frac{\pi}{4})\right]=-\sqrt{2}\left( \frac{1+\sqrt{3}}{2}\right)e^{-i\frac{\pi}{4}}\]\\
\[b=-\sqrt{2}\left( \frac{1+\sqrt{3}}{2}\right) \left[\cos(\frac{\pi}{4})+i\sin(\frac{\pi}{4})\right]=-\sqrt{2}\left( \frac{1+\sqrt{3}}{2}\right)e^{i\frac{\pi}{4}}\]
$\frac{1-a}{b-a}=\frac{1+(\frac{1+\sqrt{3}}{2})(1-i)}{(\frac{1+\sqrt{3}}{2})(1-i)+(\frac{1+\sqrt{3}}{2})(1+i)}=
\frac{\left( \frac{3+\sqrt{3}}{2}\right) -i\left(\frac{1+\sqrt{3}}{2}\right)}{-i\left( 1+\sqrt{3}\right) }$


Donc $\frac{1-a}{b-a}=\frac{\left( \frac{3+\sqrt{3}}{2}\right) -i\left(\frac{1+\sqrt{3}}{2}\right)}{-i\left( 1+\sqrt{3}\right)}=\frac{\left(\frac{1+\sqrt{3}}{2}\right)+\left(\frac{3+\sqrt{3}}{2}\right)i}{1+\sqrt{3}}=\frac{1}{2}+\frac{\sqrt{3}}{2}i$

\[\text{Donc, }\color{green}{\boxed{\frac{1-a}{b-a}=\cos(\frac{\pi}{3})+i\sin(\frac{\pi}{3})}}\]
b)Comme arg($\frac{1-a}{b-a})$=$\frac{\pi}{3}[2\pi]$ donc le triangle ABC best un triangle équilatérale
%$\frac{1-a}{b-a}=\frac{1+\sqrt{2}\left( \frac{1+\sqrt{3}}{2}\right)e^{-i\frac{\pi}{4}}}{-\sqrt{2}\left( \frac{1+\sqrt{3}}{2}\right)e^{i\frac{\pi}{4}}+\sqrt{2}\left( \frac{1+\sqrt{3}}{2}\right)e^{-i\frac{\pi}{4}}}=\frac{1+\sqrt{2}\left( \frac{1+\sqrt{3}}{2}\right)e^{-i\frac{\pi}{4}}}{\sqrt{2}\left( \frac{1+\sqrt{3}}{2}\right)\left[e^{-i\frac{\pi}{4}}-e^{i\frac{\pi}{4}}\right]}=\frac{1+\sqrt{2}\left( \frac{1+\sqrt{3}}{2}\right)e^{-i\frac{\pi}{4}}}{\sqrt{2}\left( \frac{1+\sqrt{3}}{2}\right)\left[-i\sqrt{2}\right]}=\frac{i+\sqrt{2}\left( \frac{1+\sqrt{3}}{2}\right)e^{-i\frac{\pi}{4}}.i}{\left(1+\sqrt{3}\right)}$

%$\frac{1-a}{b-a}=\frac{e^{\frac{i\pi}{2}}+\sqrt{2}\left( \frac{1+\sqrt{3}}{2}\right)e^{-i\frac{\pi}{4}}.e^{\frac{i\pi}{2}}}{1+\sqrt{3}}=\frac{e^{\frac{i\pi}{2}}+\sqrt{2}\left( \frac{1+\sqrt{3}}{2}\right)e^{-i\frac{\pi}{2}}}{\left(1+\sqrt{3}\right)}$
%$\frac{1-a}{b-a}=\frac{1+(\frac{1+\sqrt{3}}{2})(1-i)}{-(\frac{1+\sqrt{3}}{2})(1+i)+(\frac{1+\sqrt{3}}{2})(1-i)}=\frac{}{(\frac{1+\sqrt{3}}{2})}$
\section*{\textcolor{red}{\underline{Exercice 2} (6 points) :}}
\subsection*{Partie A: (2pts)}
Une urne contient trois boules jaunes, cinq boules rouges et deux boules vertes.

1) On tire simultanément trois boules de l'urne.

a) Quel est le nombre de tirages unicolores ? \textbf{0,5pt}

b)Quel est le nombre de tirages comportant exactement deux boules de même couleur ? \textbf{0,5pt}

2) On tire successivement sans remise trois boules.

a) Quel est le nombre de tirages comportant des boules rouges uniquement ? \textbf{0,5pt}

b)Quel est le nombre de tirages ne comportant pas de boule verte aux deuxième tirage ?\textbf{0,5pt}
\subsection*{Partie B: (4pts)}
Un porte-monnaie contient quatre pièce de \textbf{500F} et une pièce de \textbf{200F}.

Un enfant tire au hasard 3 pièces de ce porte-monnaie.

1)Calculer la probabilité de l'évènement A : << tiré 3 pièce de \textbf{500F} >>.\textbf{0,5pt}

2)Soit X la variable aléatoire égale au nombre de pièce de \textbf{500F} figurant parmis les 3 pièces tirées.

Déterminer la loi de probabilité de X puis représenter la fonction de répartition de X.\textbf{1pt+1pt}

3)Calculer l'espérance mathématique et l'écart-type. \textbf{0,5pt+0,5pt}

4)L'enfant rèpète 5 fois l'expérience A en remettant à chaque fois les trois pièces tirées.

Quelle est la probabilité que l'évenement A se réalise tois fois a l'issu des 5 tirages ? \textbf{0,5pt}

L'enfant efectue n fois cette épreuve. Détermine la plus petite valeur de n pour que la probabilité d'obtenir au moins une fois l'évènement A soit supérieur à 0,99. \textbf{1pt}
\section*{\textcolor{green}{\underline{Correction Exercice 2} (6 points) :}}
\subsection*{Partie A: (2pts)}

\subsection*{1) On tire simultanément trois boules de l'urne.}

\subsubsection*{a) Le nombre de tirages unicolores : \textbf{0,5 pt}}

Pour qu'un tirage soit unicolore, les trois boules tirées doivent être de la même couleur.

- Nombre de façons de tirer 3 boules jaunes (il n'y a que 3 boules jaunes) :
  \[
  C_{3}^{3} = 1
  \]

- Nombre de façons de tirer 3 boules rouges parmi 5 boules rouges :
  \[
  C_{5}^{3} = \frac{5!}{3!(5-3)!} = 10
  \]

- Nombre de façons de tirer 3 boules vertes (il n'y a que 2 boules vertes) :
  \[
  C_{2}^{3} = 0
  \]

Ainsi, le nombre total de tirages unicolores est :
\[
1 + 10 + 0 = 11
\]
\textbf{\color{green}{Autrement dit :}}

Soit A:<<Tirer 3 boules jaunes parmis les jaunes, tirer 3 boules rouges parmis les rouges et tirer 3 boules vertes par les vertes>>

\[\text{Donc, }\color{green}{\boxed{\text{card(A)=}  C_{3}^{3} + C_{5}^{3} + C_{2}^{3}=11}}\]
\subsubsection*{b) Le nombre de tirages comportant exactement deux boules de même couleur : \textbf{0,5 pt}}

Pour qu'un tirage comporte exactement deux boules de même couleur, nous devons considérer les différentes combinaisons possibles de deux boules de même couleur et une boule d'une autre couleur.

- **Deux boules jaunes et une autre boule** :

  - Nombre de façons de tirer 2 boules jaunes parmi 3 boules jaunes :
    \[
    C_{3}^{2} = 3
    \]
  - Nombre de façons de tirer 1 boule parmi les 7 autres boules (5 rouges et 2 vertes) :
    \[
    C_{7}^{1} = 7
    \]
  - Total pour cette combinaison :
    \[
    3 \times 7 = 21
    \]

- **Deux boules rouges et une autre boule** :

  - Nombre de façons de tirer 2 boules rouges parmi 5 boules rouges :
    \[
    C_{5}^{2} = 10
    \]
  - Nombre de façons de tirer 1 boule parmi les 5 autres boules (3 jaunes et 2 vertes) :
    \[
    C_{5}^{1} = 5
    \]
  - Total pour cette combinaison :
    \[
    10 \times 5 = 50
    \]

- **Deux boules vertes et une autre boule** :

  - Nombre de façons de tirer 2 boules vertes parmi 2 boules vertes :
    \[
    C_{2}^{2} = 1
    \]
  - Nombre de façons de tirer 1 boule parmi les 8 autres boules (3 jaunes et 5 rouges) :
    \[
    C_{8}^{1} = 8
    \]
  - Total pour cette combinaison :
    \[
    1 \times 8 = 8
    \]

Ainsi, le nombre total de tirages comportant exactement deux boules de même couleur est :
\[
21 + 50 + 8 = 79
\]
\textbf{\color{green}{Autrement dit :}}

Soit $B$ : << Deux boules jaunes et une autre boule, ou deux boules rouges et une autre boule, ou deux boules vertes et une autre boule >>.

\[\text{card(B)=}C_{3}^{2}\times C_{7}^{1} + C_{5}^{2}\times C_{5}^{1} + C_{2}^{2}\times C_{8}^{1}\]
\[\text{card(B)=}\frac{3!}{(3-2)!2!}\times 7 +\frac{5!}{(5-2)!2!}\times 5 + \frac{2!}{(2-2)!2!}\times 8\]
\[\text{Donc, }\color{green}{\boxed{\text{card(A)=}3\times 7 +10\times 5 + 1\times 8=79}}\]

\subsection*{2) On tire successivement sans remise trois boules.}

\subsubsection*{a) Le nombre de tirages comportant des boules rouges uniquement :}

Pour que les trois boules tirées soient rouges, nous devons choisir 3 boules rouges parmi les 5 disponibles. Comme l'ordre de tirage est important, nous devons calculer les arrangements de 5 objets pris 3 à 3 :

\[
A_{5}^{3} = \frac{5!}{(5-3)!} = 5 \times 4 \times 3 = 60
\]

Ainsi, le nombre de tirages comportant uniquement des boules rouges est de 60.

\textbf{\color{green}{Autrement dit :}}

Soit $C$ : << Deux boules jaunes et une autre boule, ou deux boules rouges et une autre boule, ou deux boules vertes et une autre boule >>.
\[\text{card(C)=}A_{5}^{3} = \frac{5!}{(5-3)!} = 5 \times 4 \times 3 = 60\]
\[\text{Donc, }\color{green}{\boxed{\text{card(C)=}60}}\]
\subsubsection*{b) Le nombre de tirages ne comportant pas de boule verte au deuxième tirage :}

Pour que la deuxième boule ne soit pas verte, nous devons considérer deux cas : tirer une boule jaune ou rouge en deuxième position.
\begin{itemize}
\item **Cas 1 : La première boule est jaune (3 possibilités)** :
	\begin{itemize}
  	\item La deuxième boule peut être jaune ou rouge (7 possibilités : 2 jaunes restantes + 5 rouges).
  	\[
		A_{3}^{1}\times A_{7}^{1}
	\]
  	\item La troisième boule peut être n'importe laquelle des 8 restantes (1 jaune, 5 rouges, 2 vertes).
	 \[
		A_{8}^{1}
	\]
	\end{itemize}
\item **Cas 2 : La première boule est rouge (5 possibilités)** :
	\begin{itemize}
 	 	\item La deuxième boule peut être jaune ou rouge (7 possibilités : 3 jaunes + 4 rouges restantes).
 	 	 \[
			A_{5}^{1}\times A_{7}^{1}
		\]
  		\item La troisième boule peut être n'importe laquelle des 8 restantes (3 jaunes, 4 rouges, 1 verte).
  		\[
			A_{8}^{1}
		\]
\end{itemize}
\end{itemize}
Le total des tirages ne comportant pas de boule verte au deuxième tirage est donc :
\[
A_{3}^{1}\times A_{7}^{1}\times A_{8}^{1} + A_{5}^{1}\times A_{7}^{1} \times A_{8}^{1}=3 \times 7 \times 8 + 5 \times 7 \times 8 = 448
\]

Ainsi, le nombre de tirages ne comportant pas de boule verte au deuxième tirage est de 448.

\textbf{\color{green}{Autrement dit :}}

Soit $D$ : << La première boule est jaune ou La première boule est verte >>.

\[\text{Card(D)=}
A_{3}^{1}\times A_{7}^{1}\times A_{8}^{1} + A_{5}^{1}\times A_{7}^{1} \times A_{8}^{1}
\]
\[\text{Donc, }\color{green}{\boxed{\text{card(D)=}3 \times 7 \times 8 + 5 \times 7 \times 8 = 448}}\]
\subsection*{Partie B: (4pts)}
Un porte-monnaie contient quatre pièce de \textbf{500F} et une pièce de \textbf{200F}.

Un enfant tire au hasard 3 pièces de ce porte-monnaie.

1)Calculons la probabilité de l'évènement A : << tiré 3 pièce de \textbf{500F} >>.\textbf{0,5pt}

Le nombre total de façons de tirer 3 pièces parmi 5:
\[
C_{5}^{3} = \frac{5!}{3!(5-3)!} = 10
\]
Le nombre de façons de tirer 3 pièces de 500 F parmi 4:
\[
C_{4}^{3} = \frac{4!}{3!(4-3)!} = 4
\]
La probabilité de l'événement A est donc:
\[
P(A) = \frac{4}{10} = \frac{2}{5}
\]
\[\color{green}{\boxed{\text{P(A)=}\frac{4}{10} = \frac{2}{5}}}\]
2)Soit X la variable aléatoire égale au nombre de pièce de \textbf{500F} figurant parmis les 3 pièces tirées.

Déterminons la loi de probabilité de X puis représentons la fonction de répartition de X.\textbf{1pt+1pt}

Pour ce faire, représentons l'arbe de probabilité :
 
Soit C << Tirer une piéce de \textbf{500F}>>

Soit D << Tirer une piéce de \textbf{200F}>>

\textbf{\textcolor{red}{Variable aléatoire}} X donnant le nombre de pièce de 500F pour chaque éventualité de E.

\begin{tabular}{|c|c|c|}
\hline
Eventualité de E & $CCD$ & $CCC$ \\
\hline
Valeurs prises par X &$2$ &$3$\\
\hline
\end{tabular}\\
D'où:$X(E) = \left\lbrace 2 ; 3 \right\rbrace $\\

\textbf{Probabilité sur l'univers $X(E)$ au loi de probabilité de la variable aléatoire X}

- Pour \(X = 2\) : tirer 2 pièces de 500 F parmi 4 et 1 pièce de 200 F parmi 1.
\[
p_{1}=P(X = 2) = \frac{C_{4}^{2} \times C_{1}^{1}}{C_{5}^{3}} = \frac{6 \times 1}{10} = \frac{3}{5}
\]

- Pour \(X = 3\) : tirer 3 pièces de 500 F parmi 4 et 0 pièce de 200 F parmi 1.
\[
p_{1}=P(X = 3) = \frac{C_{4}^{3}}{C_{5}^{3}} = \frac{4 \times 1}{10} = \frac{2}{5}
\]

\begin{tabular}{|c|c|c|}
\hline
$x_{i}$ & $x_{1}=2$ & $x_{2}=3$\\
\hline
$p_{i} = p(X = x_{i}) $ &$\frac{3}{5}$ &$\frac{2}{5}$ \\
\hline
\end{tabular}


\begin{equation*}
F(x)=\begin{cases}
0 & \text{si } x < x_{1}\\
p_{1} & \text{si } x_{1} \leq x < x_{2}\\
p_{1} + p_{2} & \text{si } x_{2} \leq x < x_{3}\\
\vdots \\
p_{1} + p_{2} + p_{3} + \ldots + p_{k} & \text{si } x_{k} \leq x < x_{k+1}\\
\vdots \\
p_{1} + p_{2} + p_{3} + \ldots + p_{n-1} & \text{si } x_{n-1} \leq x < x_{n}\\
1 & \text{si } x \geq x_{n}
\end{cases}
\end{equation*}

En considérant ce tableau 

\begin{tabular}{|c|c|c|}
\hline
$x_{i}$ & $x_{1}=2$ & $x_{2}=3$\\
\hline
$p_{i} = p(X = x_{i}) $ &$\frac{3}{5}$ &$\frac{2}{5}$ \\
\hline
\end{tabular}

Ce qui entraine

\begin{equation*}
F(x)=\begin{cases}
0 & \text{si } x < 2\\
\frac{3}{5} & \text{si } 2 \leq x < 3\\
1 & \text{si } x \geq 3
\end{cases}
\end{equation*}

\definecolor{ududff}{rgb}{0.30196078431372547,0.30196078431372547,1}
\definecolor{xdxdff}{rgb}{0.49019607843137253,0.49019607843137253,1}
\begin{tikzpicture}[line cap=round,line join=round,>=triangle 45,x=1cm,y=1cm]
\begin{axis}[
x=1cm,y=1cm,
axis lines=middle,
ymajorgrids=true,
xmajorgrids=true,
xmin=-4,
xmax=10,
ymin=-0.1,
ymax=3,
xtick={-8,-7,...,11},
ytick={-3,-2,-1,0,1,2,3,4,5},]
\clip(-8.58,-3.13) rectangle (11.22,5.13);
\draw [line width=2pt] (-9,0)-- (2,0);
\draw [line width=2pt] (2,0.6)-- (3,0.6);
\draw [line width=2pt] (3,1)-- (11,1);
\begin{scriptsize}
\draw [fill=xdxdff] (2,0) circle (2.5pt);
\draw[color=xdxdff] (-0.2,0.6) node {$\frac{3}{2}$};
\draw [fill=xdxdff] (2,0.6) circle (2.5pt);
\draw [fill=ududff] (3,1) circle (2.5pt);
\end{scriptsize}
\end{axis}
\end{tikzpicture}

--------------------------------------------------------------------------------------------------------------

3)Calculons l'espérance mathématique et l'écart-type. \textbf{0,5pt+0,5pt}
\begin{itemize}
\item \underline{Espérance}
\[E(X)=\sum_{i=1}^{n}(p_{i}x_{i})\]
\[E(X)=\sum_{i=1}^{2}(p_{i}x_{i})=2\times \frac{3}{5}+3\times \frac{2}{5}=\frac{12}{5}\]
\[\text{Donc, }\color{green}{\boxed{\text{E(X)=}\frac{12}{5}}}\]
\item \underline{Ecart-type $\sigma$}
\[V(X)=\sum_{i=1}^{n}p_{i}(x_{i}-E(X))^{2}=\left( \sum_{i=1}^{n}p_{i}x_{i}^{2}\right) -E(X)^{2}\]
\[V(X)=\frac{3}{5}\left( 2-\frac{12}{5}\right)^{2}+\frac{2}{5}\left( 3-\frac{12}{5}\right)^{2}\]
\[V(X)=\frac{5}{12}\]
\[\text{Donc, }\color{green}{\boxed{\text{E(X)=}\frac{5}{12}}}\]
\[
\text{Donc, }\color{green}{\boxed{\sigma = \sqrt{\text{V(X)}} = \sqrt{0.24} \approx 0.49}}
\]
\end{itemize}

--------------------------------------------------------------------------------------------------------------

4)L'enfant rèpète 5 fois l'expérience A en remettant à chaque fois les trois pièces tirées.

La probabilité que l'évenement A se réalise tois fois à l'issu des 5 tirages: \textbf{0,5pt}

C'est un problème de loi binomiale \(B(n = 5, p = \frac{2}{5})\) avec \(k = 3\).
\[P(X=k)=C_{n}^{k}\times p^{k}\times (1-p)^{n-k} \text{avec} (0\leq k\leq n)\] 

\[P(X=3)=C_{5}^{3}\times \left( \frac{2}{5}\right )^{3}\times \left( \frac{3}{5}\right) ^{2}\]

\[
P(X = 3) = 10 \times \frac{8}{125} \times \frac{9}{25} = 10 \times \frac{72}{3125} = \frac{720}{3125} = 0,23
\]
\[\text{Donc, }\color{green}{\boxed{\text{P(X = 3)=}0,23}}\]

---------------------------------------------------------------------------------------------------------

L'enfant effectue \(n\) fois cette épreuve. Déterminons la plus petite valeur de \(n\) pour que la probabilité d'obtenir au moins une fois l'événement A soit supérieure à 0,99. \textbf{1 pt}

La probabilité de ne pas obtenir l'événement A en un tirage est:
\[
P(\overline{A})=1-P(A)=1 - \frac{2}{5} = \frac{3}{5}
\]
La probabilité de ne pas obtenir l'événement A en \(n\) tirages est:
\[
\left(\frac{3}{5}\right)^n
\]
Nous voulons que cette probabilité soit inférieure à 0,01:
\[
\left(\frac{3}{5}\right)^n < 0,01
\]
En prenant le logarithme:
\[
n \ln\left(\frac{3}{5}\right) < \ln(0,01)
\]
\[
n > \frac{\ln(0,01)}{\ln\left(\frac{3}{5}\right)}
\]
\[
n > \frac{\ln(0,01)}{\ln(3) - \ln(5)}
\]
\[
n > \frac{-4.605}{-0.5108} \approx 9.02
\]
La plus petite valeur de \(n\) est donc 10.
\section*{\textcolor{red}{\underline{Problème} (10 points) :}}
\subsection*{\underline{Partie A:} (2 pts)}
On considère l'équation différentielle \textbf{(E)} : $\frac{1}{2}y' + y = 3e^{-2x} + 2$

\begin{itemize}
    \item[1.] Déterminer \textbf{a} pour que la fonction $v$ définie par $v(x) = \textbf{a}xe^{-2x} + 2$ soit une solution de l'équation \textbf{(E)}. \textbf{0,5 pt}
    \item[2.] Donner les solutions de l'équation \textbf{(E')} : $\frac{1}{2}y' + y = 0$. \textbf{0,5 pt}
    \item[3.] 
    \begin{itemize}
        \item[a)] Montrer que $u$ est une solution de \textbf{(E)} si et seulement si $v - u$ est solution de \textbf{(E')}. \textbf{0,5 pt}
        \item[b)] En déduire les solutions de \textbf{(E)}. \textbf{0,25 pt}
    \end{itemize}
    \item[4.] Déterminer la solution $u$ de l'équation \textbf{(E)} vérifiant $u(0) = 0$. \textbf{0,25 pt}
\end{itemize}
\subsection*{\underline{Partie B:} (8 pts)}
On définit la fonction $f$ sur $\mathbb{R}$ par :
\[
f(x) = \begin{cases} 
  2(3x - 1)e^{-2x} + 2, & \text{si } x \leq 0 \\
  \frac{x\ln x}{1 + x}, & \text{si } x > 0 
\end{cases}
\]
On note $C_{f}$ sa courbe représentative dans un repère d'unité graphique 4 cm.

\begin{itemize}
    \item[I.] On définit la fonction $g$ sur $\left]0 ; \infty\right[$ par $g(x) = 1 + x + \ln(x)$
    \begin{itemize}
        \item[1.] 
        \begin{itemize}
            \item[a)] Calculer les limites de $g$ en 0 et en $+\infty$. \textbf{0,25 pt + 0,25 pt}
            \item[b)] Étudier le sens de variation de $g$. \textbf{0,5 pt}
            \item[c)] Dresser le tableau de variation de $g$. \textbf{0,5 pt}
        \end{itemize}
        \item[2.] Montrer que l'équation $g(x) = 0$ admet une unique solution $\alpha$ dans $\left]0 ; \infty\right[$. En déduire que $\alpha \in \left]0,2 ; 0,3\right[$. \textbf{0,5 pt}
        \item[3.] Déduire le signe de $g(x)$ suivant les valeurs de $x$ sur $\left]0 ; \infty\right[$. \textbf{0,25 pt}
    \end{itemize}
    \item[II.] Étude de la fonction $f$
    \begin{itemize}
        \item[1.] Étudier la continuité et la dérivabilité de $f$ en 0 puis interpréter graphiquement les résultats. \textbf{1 pt}
        \item[2.] Étudier les limites de $f$ en $+\infty$ et en $-\infty$. \textbf{0,25 pt + 0,25 pt}
        \item[3.] Étudier les branches infinies en l'infini. \textbf{0,5 pt + 0,5 pt}
        \item[4.] Étudier le sens de variation de $f$ sur $\mathbb{R}$ (on montrera que pour tout $x > 0$, $f'(x) = \frac{g(x)}{(x + 1)^{2}}$). \textbf{0,5 pt}
        \item[5.] Dresser le tableau de variation de $f$. \textbf{0,5 pt}
        \item[6.] Montrer que $f(\alpha) = -\alpha$ puis déterminer l'intersection de $C_f$ avec les axes. \textbf{0,5 pt}
        \item[7.] Tracer $C_f$. \textbf{0,75 pt}
    \end{itemize}
\end{itemize}
\section*{\underline{Partie C:}}
Soit $\beta < 0$, on note $A(\beta)$ l'aire de la partie du plan délimitée par les droites d'équations $x = \beta$, $x = 0$, $y = 0$ et la courbe $C_f$.

\begin{itemize}
    \item[1.] On pose $F(x) = (ax + b)e^{-2x}$. Déterminer $a$ et $b$ pour que $F'(x) = (3x - 1)e^{-2x}$. \textbf{0,5 pt}
    \item[2.] Calculer $A(\beta)$. \textbf{0,5 pt}
    \item[3.] Calculer sa limite en $-\infty$. \textbf{0,25 pt}
\end{itemize}
\section*{\textcolor{green}{\underline{Correction Problème} (10 points) :}}
\subsection*{\underline{Partie A:} (2 pts)}
On considère l'équation différentielle \textbf{(E)} : $\frac{1}{2}y' + y = 3e^{-2x} + 2$

\begin{itemize}
    \item[1.] Déterminons \textbf{a} pour que la fonction $v$ définie par $v(x) = \textbf{a}xe^{-2x} + 2$ soit une solution de l'équation \textbf{(E)}. \textbf{0,5 pt}
    
$v(x) = \textbf{a}xe^{-2x} + 2$  est solution ssi $\frac{1}{2}v(x)' + v(x) = 3e^{-2x} + 2$
\[\frac{1}{2}\left[ \textbf{a}xe^{-2x} + 2\right]' + \textbf{a}xe^{-2x} + 2 = 3e^{-2x} + 2\]
\[\frac{1}{2}\textbf{a}\left[ xe^{-2x}\right]' + \textbf{a}xe^{-2x} + 2 = 3e^{-2x} + 2\]
\[\frac{1}{2}\textbf{a}\left[ e^{-2x}-2xe^{-2x}\right] + \textbf{a}xe^{-2x} + 2 = 3e^{-2x} + 2\]
\[\frac{1}{2}\textbf{a}e^{-2x}-xe^{-2x} + \textbf{a}xe^{-2x} + 2 = 3e^{-2x} + 2\]
\[\frac{1}{2}\textbf{a}e^{-2x}+\left( -1 + \textbf{a}\right)xe^{-2x}  + 2 = 3e^{-2x} + 2\]
\[\text{Par identification, }\frac{1}{2}\textbf{a}=3\]
\[\text{Donc, }\color{green}{\boxed{\textbf{a=6}}}\]

    \item[2.] Donnons les solutions de l'équation \textbf{(E')} : $\frac{1}{2}y' + y = 0$. \textbf{0,5 pt}
\[\color{green}{\boxed{y_{p}(x)=Ke^{-\frac{1}{2}x}}}\]
    \item[3.] 
    \begin{itemize}
        \item[a)] Montrons que $u$ est une solution de \textbf{(E)} si et seulement si $v - u$ est solution de \textbf{(E')}. \textbf{0,5 pt}
        
        Comme c'est une équivalence, montrons les deux sens de l'implication.
        \begin{itemize}
        \item  \textbf{Supposons que $u$ est solution de \textbf{(E)} et montrons que $v - u$ est solution de \textbf{(E')}:}
        \[\text{u est solution de \textbf{(E)} donc }\frac{1}{2}u'(x) + u(x) =3e^{-2x} + 2 \textbf{ (1)}\]
        \[\text{Or v est solution de \textbf{(E)} donc }\frac{1}{2}v'(x)+v(x)= 3e^{-2x} + 2 \textbf{ (2)}\]
        \[\text{En sommant membre à membre (1) et (2)}\]
        \[\textbf{ (2)-(1)} \implies  \frac{1}{2}\left[ v'(x) - u'(x) \right]+\left[v(x) - u(x)\right]=0	\implies \frac{1}{2}\left[ v(x) - u(x) \right]'+\left[v(x) - u(x)\right]=0\]
        Donc si u est une solution de \textbf{(E)} alors $v - u$ est une solution de \textbf{(E')}
        \item  \textbf{Supposons que $v - u$ est solution de \textbf{(E')} et montrons que $u$ est solution de \textbf{(E)}:}
        \[\text{v-u est solution de \textbf{(E')} donc }\frac{1}{2}\left[ v'(x) - u'(x) \right]+\left[v(x) - u(x)\right]=0\]
        \[\frac{1}{2}u'(x) + u(x)=\frac{1}{2}v'(x) + v(x) \textbf{ (**) }\]
                \[\text{Or v est solution de \textbf{(E)} donc }\frac{1}{2}v'(x)+v(x)= 3e^{-2x} + 2\]
        \[\text{En remplaçant } \frac{1}{2}v'(x)+v(x) \text{ par son expression dans } \textbf{(**), } \frac{1}{2}u'(x) + u(x)=3e^{-2x} + 2\]
        Donc si $v - u$ est une solution de \textbf{(E')} alors $u$ est une solution de \textbf{(E)}
        \end{itemize}
        \item[b)] Déduisons-en les solutions de \textbf{(E)}. \textbf{0,25 pt}
        
        Les solutions de \textbf{(E)} sont de la forme 
        
        \[u(x)=v(x)+y_{p}(x)=Ke^{-\frac{1}{2}x}+\textbf{6}xe^{-2x} + 2\]
        \[\color{green}{\boxed{u(x)=Ke^{-\frac{1}{2}x}+\textbf{6}xe^{-2x} + 2}}\]
    \end{itemize}
    \item[4.] Déterminons la solution $u$ de l'équation \textbf{(E)} vérifiant $u(0) = 0$. \textbf{0,25 pt}
    \[u(x)=Ke^{-\frac{1}{2}x}+\textbf{6}xe^{-2x} + 2\]
    \[u(0) = 0 \implies K + 2 = 0 \implies K=-2\]
	\[\color{green}{\boxed{u(x)=-2e^{-\frac{1}{2}x}+\textbf{6}xe^{-2x} + 2}}\]
\end{itemize}
\subsection*{\underline{Partie B:} (8 pts)}
On définit la fonction $f$ sur $\mathbb{R}$ par :
\[
f(x) = \begin{cases} 
  2(3x - 1)e^{-2x} + 2, & \text{si } x \leq 0 \\
  \frac{x\ln x}{1 + x}, & \text{si } x > 0 
\end{cases}
\]
On note $C_{f}$ sa courbe représentative dans un repère d'unité graphique 4 cm.

\begin{itemize}
    \item[I.] \underline{\textcolor{blue}{On définit la fonction $g$ sur $\left]0 ; \infty\right[$ par $g(x) = 1 + x + \ln(x)$}}
    \begin{itemize}
        \item[1.] 
        \begin{itemize}
            \item[a)] Calculons les limites de $g$ en 0 et en $+\infty$. \textbf{0,25 pt + 0,25 pt}
            \begin{itemize}
            \item \textbf{En $0^{+}$}
            \[\lim_{x \to 0^{+}}g(x) = \lim_{x \to 0^{+}} 1 + x + \ln(x)=-\infty\]
            \[\text{Donc, }\color{green}{\boxed{\lim_{x \to 0^{+}}g(x)=-\infty}}\]
            \item \textbf{En $+\infty$}
            \[\lim_{x \to +\infty}g(x) = \lim_{x \to +\infty} 1 + x + \ln(x)=+\infty\]
            \[\text{Donc, }\color{green}{\boxed{\lim_{x \to 0^{+}}g(x)=+\infty}}\]
            \end{itemize}
            \item[b)] Étudions le sens de variation de $g$. \textbf{0,5 pt}
            
            Pour cela, calculons \(g'(x)\)
            \[g'(x)=1 + \frac{1}{x}\]
            \[\text{Donc, }\color{green}{\boxed{g'(x)=\frac{x+1}{x}}}\]
            \textcolor{green}{Ainsi, \( \forall x \in ]0 ; +\infty[, g'(x)<0 \) donc  g est croissante.}
            \item[c)] Dresser le tableau de variation de $g$. \textbf{0,5 pt}
\begin{center}
\definecolor{cqcqcq}{rgb}{0.7529411764705882,0.7529411764705882,0.7529411764705882}
\begin{tikzpicture}[line cap=round,line join=round,>=triangle 45,x=1cm,y=1cm]
%\draw [color=cqcqcq,, xstep=1cm,ystep=1cm] (-7,-10) grid (-22,17);
\clip(-22,-5) rectangle (12,10);
\draw [line width=2pt] (-23,8)-- (-7,8); %première ligne A(-22,8)---B(-7,8)
\draw [line width=2pt] (-22,6)-- (-7,6); %deuxième ligne
\draw [line width=2pt] (-22,4)-- (-7,4); %troisième ligne
\draw [line width=2pt] (-22,-2)-- (-7,-2);%dernière ligne
\draw [line width=2pt] (-22,-2)-- (-22,8); %première colonne
\draw [line width=2pt] (-19,8)-- (-19,-2); %deuxième colone
\draw [line width=2pt] (-7,8)-- (-7,-2); %troisième colonne
\draw (-21,7) node[anchor=north west] {$x$};
\draw (-18.5,-1) node[anchor=north west] {$-\infty$};
\draw (-8,3.5) node[anchor=north west] {$+\infty$};
\draw (-21,1.5) node[anchor=north west] {$g(x)$};
\draw (-21,5.5) node[anchor=north west] {$g'(x)$};
\draw (-19,7) node[anchor=north west] {$0$};
\draw (-8,7) node[anchor=north west] {$+\infty$};
\draw (-13.5,5.3) node[anchor=north west] {$+$};
\draw [line width=2pt] (-13,6)-- (-13,1);
\draw (-13.2,6.5) node[anchor=north west] {$\alpha$};
\draw (-13.2,1.2) node[anchor=north west] {$0$};
\draw [->,line width=2pt] (-18,-1) -- (-8,3);
\end{tikzpicture}
\end{center}
        \end{itemize}
        \item[2.] Montrons que l'équation $g(x) = 0$ admet une unique solution $\alpha$ dans $\left]0 ; \infty\right[$. 
        \begin{itemize}
        \item \textcolor{green}{Existance de la solution $\alpha$}
        
        \[\text{Comme }\lim_{x \to 0^{+}}g(x)\times \lim_{x \to +\infty}g(x)<0 \text{ donc la solution existe.}\]
        
        \item \textcolor{green}{Unicité de la solution $\alpha$}
        
        $g(x)$  est continue comme somme de deux fonctions \\continues, \textcolor{green}{1+x} et \textcolor{green}{$\ln(x)$} et est strictement croissante sur $]0 ; +\infty[$.
        
        	Donc g admet une bijection de $]0 ; +\infty[$ vers $]-\infty ; +\infty[$. 
        	
        	Comme $0 \in ]-\infty ; +\infty[$ donc $\alpha$ existe et est unique.
        \end{itemize}
        Déduisons-en que $\alpha \in \left]0,2 ; 0,3\right[$. \textbf{0,5 pt}
        
        $\alpha \in \left]0,2 ; 0,3\right[ \iff g(0,2)\times g(0,3)<0$. 
        
        En effet, $g(0,2)=-0.409$ et $g(0,3)=0.096$ donc $g(0,2)\times g(0,3)<0$
        
        Donc $\alpha \in \left]0,2 ; 0,3\right[$
       \item[3.] Déduisons-en le signe de $g(x)$ suivant les valeurs de $x$ sur $\left]0 ; \infty\right[$. \textbf{0,25 pt}
       
       D'après le tableau de variation:
       \begin{itemize}
       \item $\forall x \in ]0 ; \alpha[$ alors $g(x) < 0$
       \item $\forall x \in ]\alpha ; +\infty[$ alors $g(x) > 0$
       \end{itemize}
    \end{itemize}
    \item[II.] \underline{\textcolor{blue}{Étude de la fonction $f$}}
    \begin{itemize}
        \item[1.] Étudions la continuité et la dérivabilité de $f$ en 0 puis interprétons graphiquement les résultats. \textbf{1 pt}
        \begin{itemize}
        	\item \underline{\textcolor{green}{continuité en 0}}
        	
        		\textcolor{green}{\underline{En $0^{-}$}:$f(x)=2(3x - 1)e^{-2x} + 2$}
        		\[\lim_{x \to 0^{-}}f(x)=2(3x - 1)e^{-2x} + 2=0\]
        		\[\text{Donc, }\color{green}{\boxed{\lim_{x \to 0^{-}}f(x)=0 \text{ et } f(0)=0 }}\]
        		
        		 \textcolor{green}{\underline{En $0^{+}$}:$f(x)=\frac{x\ln x}{1 + x}$}
        		\[\lim_{x \to 0^{+}}f(x)=\frac{x\ln x}{1 + x}=0\]
        		\[\text{Donc, }\color{green}{\boxed{\lim_{x \to 0^{+}}f(x)=0}}\]
        	\[\textcolor{green}{\text{Comme }\lim_{x \to 0^{+}}f(x)=\lim_{x \to 0^{-}}f(x)=f(0)\text{ donc f est continue en 0 }}\]
        	\item \underline{\textcolor{green}{Dérivabilité en 0}}
        	\textcolor{green}{\underline{En $0^{-}$}:$f(x)=2(3x - 1)e^{-2x} + 2$}
        	\[
        	\lim_{x \to 0^{-}}\frac{f(x)-f(0)}{x}=\lim_{x \to 0^{-}}\frac{2(3x - 1)e^{-2x} + 2}{x}=
        	\lim_{x \to 0^{-}}6e^{-2x}-2\frac{e^{-2x}-1}{x} 
        	\begin{cases}
        	\lim_{x \to 0^{-}}6e^{-2x}=6\\
        	\lim_{x \to 0^{-}}-2\frac{e^{-2x}-1}{x}  
        	\end{cases}
        	\]
        	\[
        	\text{Posons }X=-2x \implies x=-\frac{1}{2}X:
        	\begin{cases}
        	\text{Si } {x \to 0^{-}}\\
        	\text{alors } {X \to 0^{+}}
        	\end{cases}
        	\]
        	\[\text{Or }\lim_{x \to 0^{-}}-2\frac{e^{-2x}-1}{x}=\lim_{X \to 0^{+}}-2\frac{e^{X}-1}{-\frac{1}{2}X}=\lim_{X \to 0^{+}}4\frac{e^{X}-1}{X}=4\]
        	\[\text{Par somme, }\lim_{x \to 0^{-}}\frac{f(x)-f(0)}{x}=10\]
        	\[\text{Donc, }\color{green}{\boxed{\lim_{x \to 0^{-}}\frac{f(x)-f(0)}{x}=10}}\]
        	\textcolor{green}{\underline{En $0^{+}$}:$f(x)=\frac{x\ln x}{1+x}$}
        	\[
        	\lim_{x \to 0^{+}}\frac{f(x)-f(0)}{x}=\lim_{x \to 0^{+}}\frac{\frac{x\ln x}{1+x}}{x}=
        	\lim_{x \to 0^{+}}\frac{\ln x}{1+x}=-\infty
        	\]
        	\[\text{Donc, }\color{green}{\boxed{\lim_{x \to 0^{+}}\frac{f(x)-f(0)}{x}=-\infty}}\]
        	\item \underline{\textcolor{green}{Interprétons graphiquement:}}
        	\[\text{Comme, }\lim_{x \to 0^{-}}\frac{f(x)-f(0)}{x}=10 \text{ donc }\]
        	\[\text{Comme, }\lim_{x \to 0^{+}}\frac{f(x)-f(0)}{x}=-\infty \text{ donc f admet une demi-tangent verticale orienté vers le bas }\]
        \end{itemize}
        \item[2.] Étudions les limites de $f$ en $+\infty$ et en $-\infty$. \textbf{0,25 pt + 0,25 pt}
        \begin{itemize}
        \item \textcolor{green}{\underline{En $-\infty$}:$f(x)=2(3x - 1)e^{-2x} + 2$}
        \[\lim_{x \to -\infty}f(x)=\lim_{x \to -\infty}2(3x - 1)e^{-2x} + 2=-\infty\]
        \[\text{Donc, }\color{green}{\boxed{\lim_{x \to -\infty}f(x)=-\infty}}\]
         \item \textcolor{green}{\underline{En $+\infty$}:$f(x)=\frac{\ln x}{1+x}$}
        \[\lim_{x \to +\infty}f(x)=\lim_{x \to +\infty}\frac{\frac{\ln x}{x}}{\frac{1+x}{x}}=0\]
        \[\text{Donc, }\color{green}{\boxed{\lim_{x \to +\infty}f(x)=0}}\]
        \textcolor{green}{\boxed{\text{$y=0$:(Cf) est asymptote verticale}}}
        \end{itemize}
        \item[3.] Étudions les branches infinies en l'infinie. \textbf{0,5 pt + 0,5 pt}
        \item \textcolor{green}{\underline{En $-\infty$}:$f(x)=2(3x - 1)e^{-2x} + 2$}
        \[\text{ comme }\lim_{x \to -\infty}f(x)=-\infty, \text{ calculons } \lim_{x \to -\infty}\frac{f(x)}{x}\]
         \[\lim_{x \to -\infty}\frac{f(x)}{x}=\lim_{x \to -\infty}\frac{2(3x - 1)e^{-2x} + 2}{x}=\lim_{x \to -\infty}\frac{x\left[ 2\left( 3 - \frac{1}{x}\right)e^{-2x} + \frac{2}{x}\right] }{x}=\lim_{x \to -\infty}2e^{-2x}\left( 3 - \frac{1}{x}\right) + \frac{2}{x}\]
         \[\lim_{x \to -\infty}\frac{f(x)}{x}=+\infty\]
         \textcolor{green}{\boxed{\text{En $-\infty$:(Cf) admet une branche parabolique de diretion (OJ)}}}
        \item[4.] Étudions le sens de variation de $f$ sur $\mathbb{R}$ (on montrera que pour tout $x > 0$, $f'(x) = \frac{g(x)}{(x + 1)^{2}}$). \textbf{0,5 pt}
        %Calculons les dérivées de f
        \begin{itemize}
        	\item Si $x \leq 0, f(x)=2(3x - 1)e^{-2x} + 2$
        	
        	\[f'(x)=6e^{-2x}-4(3x - 1)e^{-2x}=-12xe^{-2x}+10e^{-2x}=2\left(-6x + 5\right) e^{-2x}\]
        	\[\textcolor{green}{\boxed{f'(x)=2\left(-6x + 5\right) e^{-2x}}}\]
Le signe de  f' dépend de $-6x + 5$ or $\forall x\in]-\infty ; 0]$, -6x + 5>0 donc f est croissant sur $]-\infty ; 0]$
			\item Si $x > 0, f(x)=\frac{x\ln x}{1 + x}$, montrons que $f'(x)=\frac{g(x)}{(x + 1)^{2}}$
			\[f'(x)=\frac{(x\ln x)'(1 + x)-(1 + x)'(x\ln x)}{ \left( 1 + x\right)^{2} }=\frac{(\ln x + 1)(1 + x)-x\ln x}{ \left( 1 + x\right)^{2} }\]
			\[f'(x)=\frac{(\ln x + x\ln x + 1 + x)-x\ln x}{ \left( 1 + x\right)^{2} } =\frac{1 + x + \ln x}{\left( 1 + x\right)^{2}} \]
			\[\textcolor{green}{\boxed{f'(x)=\frac{g(x)}{\left( 1 + x\right)^{2}}}}\]
Le signe de f' dépend de $g(x)$ or, d'après la partie A, 

$\forall x\in]0 ; \alpha[$, $g(x) < 0$ donc f est décroissant sur $]0 ; \alpha[$ \textbf{et} 

$\forall x\in]\alpha ; +\infty[$, $g(x) > 0$ donc f est croissant sur $]\alpha ; +\infty[$

\textbf{En résumé:}

\textcolor{green}{sur $]-\infty ; 0]$ f'>0 donc f est croissant \\sur $]0 ; \alpha[$ f'<0 donc f est décroissant\\sur $]\alpha ; +\infty[$ f'>0 donc f est croissant}
        \end{itemize}
        \item[5.] Dressons le tableau de variation de $f$. \textbf{0,5 pt}
        \begin{center}
        \definecolor{cqcqcq}{rgb}{0.7529411764705882,0.7529411764705882,0.7529411764705882}
\begin{tikzpicture}[line cap=round,line join=round,>=triangle 45,x=1cm,y=1cm]
%\draw [color=cqcqcq,, xstep=1cm,ystep=1cm] (-7,-10) grid (-22,17);
\clip(-22,-5) rectangle (12,10);
\draw [line width=2pt] (-23,8)-- (-7,8); %première ligne A(-22,8)---B(-7,8)
\draw [line width=2pt] (-22,6)-- (-7,6); %deuxième ligne
\draw [line width=2pt] (-22,4)-- (-7,4); %troisième ligne
\draw [line width=2pt] (-22,-2)-- (-7,-2);%dernière ligne
\draw [line width=2pt] (-22,-2)-- (-22,8); %première colonne
\draw [line width=2pt] (-19,8)-- (-19,-2); %deuxième colone
\draw [line width=2pt] (-15,6)-- (-15,4); %troisième colonne
\draw [line width=2pt] (-11,6)-- (-11,-2); %quatrième colonne
\draw [line width=2pt] (-7,8)-- (-7,-2); %cinquième colonne
\draw (-21,1.5) node[anchor=north west] {$f(x)$};
\draw (-21,5.5) node[anchor=north west] {$f'(x)$};
\draw (-21,7) node[anchor=north west] {$x$};
\draw (-19,7) node[anchor=north west] {$-\infty$};
\draw (-15.3,6.5) node[anchor=north west] {0};
\draw (-11.3,6.5) node[anchor=north west] {$\alpha$};
\draw (-8,7) node[anchor=north west] {$+\infty$};
%signe de la dérivé
\draw (-17,5.3) node[anchor=north west] {$+$};
%\draw (-15.3,5.3) node[anchor=north west] {$O$};
\draw (-13.5,5.3) node[anchor=north west] {$-$};
\draw (-11.25,5.3) node[anchor=north west] {\textbf{\textcolor{blue}{\textbf{0}}}};
\draw (-10,5.3) node[anchor=north west] {$+$};

\draw [->,line width=2pt] (-18.8,-1.2) -- (-15.24,3.9);
\draw (-19,-1.2) node[anchor=north west] {\textbf{\textcolor{blue}{$-\infty$}}};
\draw (-15.24,3.9) node[anchor=north west] {\textbf{\textcolor{blue}{\textbf{0}}}};

\draw [->,line width=2pt] (-14.6,3.7) -- (-11.23,-1.2);
\draw (-11.5,-1.2) node[anchor=north west] {\textbf{\textcolor{blue}{\textbf{f($\alpha$)}}}};
\draw [->,line width=2pt] (-10.6,-1.3) -- (-8,3.5);
\draw (-8,3.9) node[anchor=north west] {$+\infty$};
\end{tikzpicture}
        \end{center}
        \item[6.] Montrons que $f(\alpha) = -\alpha$ puis déterminons l'intersection de $C_f$ avec les axes. \textbf{0,5 pt}
        \begin{itemize}
        \item Pour $f(\alpha) = -\alpha$
        \[f(\alpha) = \frac{\alpha \ln \alpha }{1+\alpha} \text{ Or } g(\alpha)=1+\alpha + \ln \alpha=0 \implies  \ln \alpha = -(1+\alpha)\]
        \[\text{En remplaçant }\ln \alpha  \text{ dans } f(\alpha) = \frac{\alpha \ln \alpha }{1+\alpha}, \text{ On a } f(\alpha) = \frac{-(1+\alpha)\alpha }{1+\alpha}=-\alpha\]
        \[\text{ D'où }f(\alpha)=-\alpha\]
        \end{itemize}
       % \item[7.] Tracer $C_f$. \textbf{0,75 pt}
    \end{itemize}
\end{itemize}
\end{document}