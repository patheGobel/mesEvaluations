\documentclass[12pt]{article}
\usepackage{stmaryrd}
\usepackage{graphicx}
\usepackage[utf8]{inputenc}

\usepackage[french]{babel}
\usepackage[T1]{fontenc}
\usepackage{hyperref}
\usepackage{verbatim}

\usepackage{color, soul}

\usepackage{pgfplots}
\pgfplotsset{compat=1.15}
\usepackage{mathrsfs}

\usepackage{amsmath}
\usepackage{amsfonts}
\usepackage{amssymb}
\usepackage{tkz-tab}

\usepackage{tikz}
\usetikzlibrary{arrows, shapes.geometric, fit}


\usepackage[margin=2cm]{geometry}

\begin{document}

\begin{minipage}{0.5\textwidth}
	Ministère de l'éducation nationale  \\
	Inspection académique de Kédougou   \\
	Lycée de Dindéferlo            \\
	Cellule de mathématiques            \\
	M. BA                          \\
	Classe : TS  \\
\end{minipage}
\begin{minipage}{0.5\textwidth}
	Année scolaire 2023-2024 \\
	Date : 10-06-2024 \\
	Durée : 4h 00 \\
\end{minipage}

\begin{center}
	\textbf{{\underline{\textcolor{red}{Correction de la Composition Du Second Semestre}}}}
\end{center}
\section*{\textcolor{red}{\underline{Exercice 1} (4 points) :}}
On considère les intégrales suivantes : $I_{0}=\int_{0}^{\frac{\pi}{3}} \frac{dx}{\cos (x)}$ et $J_{n}=\int_{0}^{\frac{\pi}{3}} \frac{\sin^{n}(x)}{\cos(x)}dx,$ $n \in \mathbb{N}^{*}$

\begin{enumerate}
    \item Calculer $I_{0}=\int_{0}^{\frac{\pi}{3}} \sin^{n}(x)\cos(x)dx$ puis en déduire $I_{n+2}-I_{n}$ en fonction de $n$.\\ \textbf{0,5 pt + 0,5 pt}
    \item Calculer $I_{1}$ puis en déduire $I_{3}$ et $I_{5}$. \textbf{0,5 pt + 0,5 pt + 0,5 pt}
    \item 
    \begin{enumerate}
        \item Soit $f$ la fonction qui à tout $x \in \left[0, \frac{\pi}{3}\right]$ associe $f(x) = \ln\left( \tan\left(\frac{x}{2}+\frac{\pi}{4}\right)\right)$.\\ Montrer que $f$ est une primitive de la fonction $g$ définie par $g(x) = \frac{1}{\cos(x)}$, $x \in \left[0, \frac{\pi}{3}\right]$. \textbf{0,5 pt}
        \item En déduire $I_{0}$ puis $I_{2}$. \textbf{0,5 pt + 0,5 pt}
    \end{enumerate}
\end{enumerate}


\section*{\textcolor{green}{\underline{Correction Exercice 1} (4 points) :}}

\begin{enumerate}
    \item On calcule $I_{0}$ :
    \[
    I_{0}=\int_{0}^{\frac{\pi}{3}} \frac{dx}{\cos(x)} = \int_{0}^{\frac{\pi}{3}} \sec(x) \, dx = \left[ \ln(\tan\left(\frac{x}{2}+\frac{\pi}{4}\right)) \right]_{0}^{\frac{\pi}{3}} = \ln(\sqrt{3} + 1)
    \]
    Pour $J_n$, en utilisant l'intégration par parties :
    \[
    J_{n} = \int_{0}^{\frac{\pi}{3}} \frac{\sin^n(x)}{\cos(x)}dx = \int_{0}^{\frac{\pi}{3}} \sin^n(x) \sec(x) \, dx
    \]
    On en déduit que :
    \[
    I_{n+2} - I_{n} = \int_{0}^{\frac{\pi}{3}} (\sin^{n+2}(x) - \sin^n(x)) \sec(x) \, dx
    \]
    \item Calculons $I_{1}$, $I_{3}$ et $I_{5}$ :
    \[
    I_{1} = \int_{0}^{\frac{\pi}{3}} \sin(x) \, dx = \left[ -\cos(x) \right]_{0}^{\frac{\pi}{3}} = 1 - \frac{1}{2} = \frac{1}{2}
    \]
    Pour $I_{3}$ et $I_{5}$, en utilisant des relations de récurrence similaires :
    \[
    I_{3} = \int_{0}^{\frac{\pi}{3}} \sin^3(x) \sec(x) \, dx
    \]
    \[
    I_{5} = \int_{0}^{\frac{\pi}{3}} \sin^5(x) \sec(x) \, dx
    \]
    \item 
    \begin{enumerate}
        \item La fonction $f(x) = \ln\left( \tan\left(\frac{x}{2}+\frac{\pi}{4}\right)\right)$ est une primitive de $g(x) = \frac{1}{\cos(x)}$ :
        \[
        f'(x) = \frac{1}{\tan\left(\frac{x}{2}+\frac{\pi}{4}\right)} \cdot \sec^2\left(\frac{x}{2}+\frac{\pi}{4}\right) \cdot \frac{1}{2} = \frac{1}{\cos(x)}
        \]
        \item En déduire $I_{0}$ et $I_{2}$ :
        \[
        I_{0} = \ln\left( \tan\left(\frac{\pi}{6}+\frac{\pi}{4}\right) \right) = \ln\left( \sqrt{3} + 1 \right)
        \]
        \[
        I_{2} = \int_{0}^{\frac{\pi}{3}} \sin^2(x) \sec(x) \, dx = \frac{1}{2} \int_{0}^{\frac{\pi}{3}} (1 - \cos(2x)) \sec(x) \, dx = \frac{1}{2} \left( I_{0} - \int_{0}^{\frac{\pi}{3}} \cos(2x) \sec(x) \, dx \right)
        \]
    \end{enumerate}
\end{enumerate}
\section*{\textcolor{green}{\underline{Correction Exercice 3} (6 points) :}}

\section*{\textcolor{green}{\underline{Correction du Problème} (10 points) :}}



\end{document}
