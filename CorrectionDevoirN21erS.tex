\documentclass[12pt]{article}
\usepackage{stmaryrd}
\usepackage{graphicx}
\usepackage[utf8]{inputenc}

\usepackage[french]{babel}
\usepackage[T1]{fontenc}
\usepackage{hyperref}
\usepackage{verbatim}

\usepackage{color, soul}

\usepackage{pgfplots}
\pgfplotsset{compat=1.15}
\usepackage{mathrsfs}

\usepackage{amsmath}
\usepackage{amsfonts}
\usepackage{amssymb}
\usepackage{tkz-tab}

\usepackage{tikz}
\usetikzlibrary{arrows, shapes.geometric, fit}


\usepackage[margin=2cm]{geometry}
\begin{document}

\begin{minipage}{0.5\textwidth}
	Ministère de l'éducation nationale  \\
	Inspection académique de Kédougou   \\
	Lycée de Dindéfélo            \\
	Cellule de mathématiques            \\
	M. BA                          \\
	Classe : 1erS  \\
\end{minipage}
\begin{minipage}{0.5\textwidth}
	Année scolaire 2023-2024 \\
	Date : 14-05-2024 \\
	Durée : 4h 00 \\
\end{minipage}

\begin{center}
	\textbf{{\underline{Devoir N2 Du Second Semestre}}}
\end{center}

\section*{\underline{Exercice 1} (9 points) :}
\subsection*{a) Calculer les limites suivantes (0,5 pt $\times$ 4)}
\[ \lim_{x \to 2}\frac{x^{2}-5x-14}{x^{2}+4x+4}\quad\quad 
\lim_{x \to 2}\frac{4x^{2}-3x-1}{x^{3}-4x^{2}+2x+1}\quad\quad 
\lim_{x \to 1}\frac{2x-\sqrt{x+1}-4}{x^{2}-2x-3}\quad\quad 
\lim_{x \to 0}x\mid x+\frac{1}{x}\mid\]
\subsection*{1) Dans chacun des cas suivants, calculer la limites en $+\infty$ et en $-\infty$ (1 pt $\times$ 4)}
\[ f(x)=\frac{\sqrt{3x^{2}+1}}{3x-1}\quad\quad 
g(x)=x+2+\sqrt{x^{2}-3x+1}\quad\quad 
h(x)=\sqrt{x^{2}-2x-1}-\sqrt{x^{2}-7x+3} \]
\[I(x)=\frac{x-\sqrt{x^{2}-3x+1}}{2x+\sqrt{4x^{2}+x}} \]
\subsection*{2) Etudier de continuité: (0,5pt +1pt)}
\[ f(x) = \begin{cases} 
  -x+1-\frac{2x}{x^{2}+1}, & \text{si } x \leq 1 \\
  \frac{x-1}{\sqrt{x^{2}-2x+6}}, & \text{si } x > 1
\end{cases} \]
a)Déterminer le domaine de définition $D_{f}$\\
b)Etudier la continuité de f en $x_{0}=1$
\subsection*{3) Prolongement par continuité: (0,5pt +1pt)}
Soit $h$ la fonction définie par:
\[ h(x) = \begin{cases} 
  \sqrt{x^{2}-4}, & \text{si } x > 2 \\
  \sqrt{x^{2}-9x+14}, & \text{si } x < 2
\end{cases} \]
a)Déterminer $D_{h}$.\\
b)La fonction h est-elle prolongeable par continuité en $2$ ?
\section*{\textcolor{red}{\underline{Correction Exercice 1} (9 points) :}}
\subsection*{a) Calculons les limites suivantes (0,5 pt $\times$ 4)}
\[ \lim_{x \to 2}\frac{x^{2}-5x-14}{x^{2}+4x+4}=-\frac{5}{4}\]
\[\lim_{x \to 2}\frac{4x^{2}-3x-1}{x^{3}-4x^{2}+2x+1}=-3\]
\[\lim_{x \to 1}\frac{2x-\sqrt{x+1}-4}{x^{2}-2x-3}=\frac{2+\sqrt{2}}{4}\]
\[\lim_{x \to 0}x\mid x+\frac{1}{x}\mid \text{ Ecrivons sans valeur absolue.}\]
Posons $x+\frac{1}{x}=0\Rightarrow\frac{x^{2}+1}{x}=0\Rightarrow x^{2}=-1$ et $x=0$
Par quotient, forme infinie\\
\definecolor{cqcqcq}{rgb}{0.7529411764705882,0.7529411764705882,0.7529411764705882}
\begin{tikzpicture}[line cap=round,line join=round,>=triangle 45,x=1cm,y=1cm]
%\draw [color=cqcqcq,, xstep=1cm,ystep=1cm] (-7,-10) grid (-22,17);
\clip(-22,3) rectangle (12,10);
\draw [line width=2pt] (-23,8)-- (-7,8); %première ligne A(-22,8)---B(-7,8)
\draw [line width=2pt] (-22,6)-- (-7,6); %deuxième ligne
\draw [line width=2pt] (-22,5)-- (-7,5); %troisième ligne
\draw [line width=2pt] (-22,4)-- (-7,4); %quatrième ligne
\draw [line width=2pt] (-22,4)-- (-22,8); %première colonne (-22,4)<----(-22,8);
\draw [line width=2pt] (-18,8)-- (-18,4); %deuxième colone  (-18,8)--->(-18,4);
\draw [line width=2pt] (-12,6)-- (-12,4); %troisième colonne(-13,6)--->(-13,4);
\draw [line width=2pt] (-12.1,6)-- (-12.1,4); %troisième colonne(-13,6)--->(-13,4);
\draw [line width=2pt] (-7,8)-- (-7,4); %quatrième colonne (-7,8)-->(-7,4);
\draw (-21,7) node[anchor=north west] {$x$};
\draw (-21,5.9) node[anchor=north west] {$\frac{x^{2}+1}{x}$};
\draw (-21,5) node[anchor=north west] {$\mid\frac{x^{2}+1}{x}\mid$};
\draw (-15.8,5.9) node[anchor=north west] {$-$};
\draw (-15.8,5) node[anchor=north west] {$-\frac{x^{2}+1}{x}$};
\draw (-10.5,5.9) node[anchor=north west] {$+$};
\draw (-10.5,5) node[anchor=north west] {$\frac{x^{2}+1}{x}$};
\draw (-18,7) node[anchor=north west] {$-\infty$};
\draw (-12.2,7) node[anchor=north west] {$0$};
\draw (-8,7) node[anchor=north west] {$+\infty$};
\end{tikzpicture}
\[ x\mid x+\frac{1}{x}\mid\ = \begin{cases} 
  -x^{2}-1, & \text{si } x < 0 \\
  x^{2}+1, & \text{si } x > 0
\end{cases} \]
\textcolor{red}{\underline{En $0^{-}$:}}
\[\lim_{x \to 0}x\mid x+\frac{1}{x}\mid=\lim_{x \to 0^{-}}-x^{2}-1=-1\]
\[\textcolor{red}{\boxed{\lim_{x \to 0}x\mid x+\frac{1}{x}\mid=-1}}\]
\textcolor{red}{\underline{En $0^{+}$:}}
\[\lim_{x \to 0}x\mid x+\frac{1}{x}\mid=\lim_{x \to 0^{+}}x^{2}+1=1\]
\[\textcolor{red}{\boxed{\lim_{x \to 0}x\mid x+\frac{1}{x}\mid=1}}\]
\subsection*{1) Calculons la limites en $+\infty$ et en $-\infty$ (1 pt $\times$ 4)}
\textcolor{blue}{\[ f(x)=\frac{\sqrt{3x^{2}+1}}{3x-1}\]} 
\textcolor{red}{\underline{En $-\infty$:}}
\[\lim_{x \to -\infty}f(x)=\lim_{x \to -\infty}\frac{\sqrt{x^{2}(3+\frac{1}{x^{2}})}}{x(3-\frac{1}{x})}=\lim_{x \to -\infty}\frac{\sqrt{x^{2}}\sqrt{(3+\frac{1}{x^{2}})}}{x(3-\frac{1}{x})}=\lim_{x \to -\infty}\frac{|x|\sqrt{(3+\frac{1}{x^{2}})}}{x(3-\frac{1}{x})}=\lim_{x \to -\infty}\frac{-x\sqrt{(3+\frac{1}{x^{2}})}}{x(3-\frac{1}{x})}\]
\[\lim_{x \to -\infty}f(x)=\lim_{x \to -\infty}\frac{-\sqrt{(3+\frac{1}{x^{2}})}}{(3-\frac{1}{x})}=-\frac{\sqrt{3}}{3}\]
\[\textcolor{red}{\boxed{\lim_{x \to -\infty}f(x)=-\frac{\sqrt{3}}{3}}}\]
\textcolor{red}{\underline{En $+\infty$:}}
\[\lim_{x \to \infty}f(x)=\lim_{x \to +\infty}\frac{\sqrt{x^{2}(3+\frac{1}{x^{2}})}}{x(3-\frac{1}{x})}=\lim_{x \to +\infty}\frac{\sqrt{x^{2}}\sqrt{(3+\frac{1}{x^{2}})}}{x(3-\frac{1}{x})}=\lim_{x \to +\infty}\frac{|x|\sqrt{(3+\frac{1}{x^{2}})}}{x(3-\frac{1}{x})}=\lim_{x \to +\infty}\frac{x\sqrt{(3+\frac{1}{x^{2}})}}{x(3-\frac{1}{x})}\]
\[\lim_{x \to +\infty}f(x)=\lim_{x \to +\infty}\frac{\sqrt{\left( 3+\frac{1}{x^{2}}\right) }}{\left( 3-\frac{1}{x}\right)}=\frac{\sqrt{3}}{3}\]
\[\textcolor{red}{\boxed{\lim_{x \to +\infty}f(x)=\frac{\sqrt{3}}{3}}}\]
\textcolor{blue}{\[g(x)=x+2+\sqrt{x^{2}-3x+1}\]}
\textcolor{red}{\underline{En $-\infty$:}}
\[\lim_{x \to -\infty}g(x)=\lim_{x \to -\infty}x+2+\sqrt{x^{2}-3x+1}=\lim_{x \to -\infty}\frac{(x+2)^{2}-(x^{2}-3x+1)}{x+2-\sqrt{x^{2}-3x+1}}\]
\[\lim_{x \to -\infty}g(x)=\lim_{x \to -\infty}\frac{5x+3}{x+2-\sqrt{x^{2}-3x+1}}=\lim_{x \to -\infty}\frac{x(5+\frac{3}{x})}{x\left(1+\frac{2}{x}+\sqrt{1-\frac{3}{x}+\frac{1}{x^{2}}}\right)}\]
\[\lim_{x \to -\infty}g(x)=\lim_{x \to -\infty}\frac{(5+\frac{3}{x})}{\left(1+\frac{2}{x}+\sqrt{1-\frac{3}{x}+\frac{1}{x^{2}}}\right)}=\frac{5}{2}\]
\[\textcolor{red}{\boxed{\lim_{x \to -\infty}g(x)=\frac{5}{2}}}\]

\textcolor{red}{\underline{En $+\infty$:}}
\[\lim_{x \to +\infty}g(x)=\lim_{x \to +\infty}x+2+\sqrt{x^{2}-3x+1}=\lim_{x \to +\infty}x\left(1+\frac{2}{x}+\sqrt{1-\frac{3}{x}+\frac{1}{x^{2}}}\right)=+\infty\] 
\[\textcolor{red}{\boxed{\lim_{x \to +\infty}g(x)=+\infty}}\]

\textcolor{blue}{\[h(x)=\sqrt{x^{2}-2x-1}-\sqrt{x^{2}-7x+3}\]}
\textcolor{red}{\underline{En $-\infty$:}}
\[\lim_{x \to -\infty}h(x)=\lim_{x \to -\infty}\sqrt{x^{2}-2x-1}-\sqrt{x^{2}-7x+3}=\lim_{x \to -\infty}\frac{\left( x^{2}-2x-1\right) -\left( x^{2}-7x+3\right)}{\sqrt{x^{2}-2x-1}+\sqrt{x^{2}-7x+3}}\]
 
\[\lim_{x \to -\infty}h(x)=\lim_{x \to -\infty}\frac{5x-4}{\sqrt{x^{2}-2x-1}+\sqrt{x^{2}-7x+3}}=\lim_{x \to -\infty}\frac{x(5-\frac{4}{x})}{-x\sqrt{1-\frac{2}{x}-\frac{1}{x^{2}}}-x\sqrt{1-\frac{7}{x}+\frac{3}{x^{2}}}}\]
\[\lim_{x \to -\infty}h(x)=\lim_{x \to -\infty}\frac{(5-\frac{4}{x})}{-\left[\sqrt{1-\frac{2}{x}-\frac{1}{x^{2}}}+\sqrt{1-\frac{7}{x}+\frac{3}{x^{2}}}\right]}=-\frac{5}{2}\] 
\[\textcolor{red}{\boxed{\lim_{x \to -\infty}h(x)=-\frac{5}{2}}}\]

\textcolor{red}{\underline{En $+\infty$:}}
\[\lim_{x \to +\infty}h(x)=\lim_{x \to +\infty}\sqrt{x^{2}-2x-1}-\sqrt{x^{2}-7x+3}=\lim_{x \to +\infty}\frac{\left( x^{2}-2x-1\right) -\left( x^{2}-7x+3\right)}{\sqrt{x^{2}-2x-1}+\sqrt{x^{2}-7x+3}}\]
 
\[\lim_{x \to +\infty}h(x)=\lim_{x \to +\infty}\frac{5x-4}{\sqrt{x^{2}-2x-1}+\sqrt{x^{2}-7x+3}}=\lim_{x \to +\infty}\frac{x(5-\frac{4}{x})}{x\sqrt{1-\frac{2}{x}-\frac{1}{x^{2}}}x\sqrt{1-\frac{7}{x}+\frac{3}{x^{2}}}}\]
\[\lim_{x \to +\infty}h(x)=\lim_{x \to +\infty}\frac{(5-\frac{4}{x})}{\left[\sqrt{1-\frac{2}{x}-\frac{1}{x^{2}}}+\sqrt{1-\frac{7}{x}+\frac{3}{x^{2}}}\right]}=\frac{5}{2}\] 
\[\textcolor{red}{\boxed{\lim_{x \to -\infty}h(x)=\frac{5}{2}}}\]
\[h(x)=\sqrt{x^{2}-2x-1}-\sqrt{x^{2}-7x+3} \]
\[I(x)=\frac{x-\sqrt{x^{2}-3x+1}}{2x+\sqrt{4x^{2}+x}} \]
\section*{\underline{Exercice 2}(3 points) :}
\subsection*{a) Dans chaque cas, justifier que f est dérivable sur I. (1,5 pts)}
\[f(x)=(2x^{2}-1)(x-3)^{4};\quad I=\mathbb{R}\quad\quad g(x)=\frac{-3}{x^{3}-1};\quad I=\mathbb{R}\setminus\left\lbrace 1 \right\rbrace \quad\quad I(x)=\sqrt{3x^{2}+x+7};\quad I=\mathbb{R}\]
\subsection*{b) Soit $f(x)=|x+1|$ }
Montrer que $f$ est continue en -1. Qu'en est-il de sa dérivable en -1 ? \textbf{(1,5 pts)}
\section*{\textcolor{red}{\underline{Correction Exercice 2} (3 points) :}}
\subsection*{a) Dans chaque cas, justifions que f est dérivable sur I. (1,5 pts)}
$f(x)=(2x^{2}-1)(x-3)^{4}$ est le produit deux fonctions polynomes donc dérivable sur $I=\mathbb{R}$

$g(x)=\frac{-3}{x^{3}-1}$ est une fonction rationnel donc dérivable sur $I=\mathbb{R}\setminus\left\lbrace 1 \right\rbrace$

$I(x)=\sqrt{3x^{2}+x+7}$ est une fonction irrationnel donc dérivable sur $I=\mathbb{R}$ car\\ $\forall x\in \mathbb{R}, 3x^{2}+x+7\neq 0$
\subsection*{b) Soit $f(x)=|x+1|$ }
Montrons que $f$ est continue en -1.

\[\text{A-t-on }\lim_{x \to 0^{-}}f(x)=\lim_{x \to 0^{+}}f(x)=f(0)?\]
\[\textcolor{red}{\text{ Ecrivons sans valeur absolue.}}\]
Posons $x+1=0\Rightarrow x=-1$

\definecolor{cqcqcq}{rgb}{0.7529411764705882,0.7529411764705882,0.7529411764705882}
\begin{tikzpicture}[line cap=round,line join=round,>=triangle 45,x=1cm,y=1cm]
%\draw [color=cqcqcq,, xstep=1cm,ystep=1cm] (-7,-10) grid (-22,17);
\clip(-22,3) rectangle (12,10);
\draw [line width=2pt] (-23,8)-- (-7,8); %première ligne A(-22,8)---B(-7,8)
\draw [line width=2pt] (-22,6)-- (-7,6); %deuxième ligne
\draw [line width=2pt] (-22,5)-- (-7,5); %troisième ligne
\draw [line width=2pt] (-22,4)-- (-7,4); %quatrième ligne
\draw [line width=2pt] (-22,4)-- (-22,8); %première colonne (-22,4)<----(-22,8);
\draw [line width=2pt] (-18,8)-- (-18,4); %deuxième colone  (-18,8)--->(-18,4);
\draw [line width=2pt] (-12,6)-- (-12,4); %troisième colonne(-13,6)--->(-13,4);
%\draw [line width=2pt] (-12.1,6)-- (-12.1,4); %troisième colonne(-13,6)--->(-13,4);
\draw [line width=2pt] (-7,8)-- (-7,4); %quatrième colonne (-7,8)-->(-7,4);
\draw (-21,7) node[anchor=north west] {$x$};
\draw (-21,5.9) node[anchor=north west] {$x+1$};
\draw (-21,5) node[anchor=north west] {$\mid x+1\mid$};
\draw (-15.8,5.9) node[anchor=north west] {$-$};
\draw (-15.8,5) node[anchor=north west] {$-x-1$};
\draw (-10.5,5.9) node[anchor=north west] {$+$};
\draw (-10.5,5) node[anchor=north west] {$x+1$};
\draw (-18,7) node[anchor=north west] {$-\infty$};
\draw (-12.2,7) node[anchor=north west] {$-1$};
\draw (-8,7) node[anchor=north west] {$+\infty$};
\end{tikzpicture}
\[ \mid x+1\mid\ = \begin{cases} 
  -x-1, & \text{si } x < -1 \\
  x+1, & \text{si } x > -1
\end{cases} \]
\textcolor{red}{\underline{En $-1^{-}$:}}
\[\lim_{x \to -1}\mid x+1\mid=\lim_{x \to -1^{-}}-x-1=0\]
\[\textcolor{red}{\boxed{\lim_{x \to -1}\mid x+1\mid=0}}\]
\textcolor{red}{\underline{En $1^{+}$:}}
\[\lim_{x \to -1}x\mid x+1\mid=\lim_{x \to -1^{+}}x+1=0\]
\[\textcolor{red}{\boxed{\lim_{x \to -1}\mid x+1\mid=0}}\]
\textcolor{green}{\[\text{Oui, }\lim_{x \to 0^{-}}f(x)=\lim_{x \to 0^{+}}f(x)=f(0)\text{ Donc f est continue en -1.}\]}

La dérivable de f en -1:
\[\text{A-t-on }\lim_{x \to -1^{-}}\frac{f(x)-f(-1)}{x+1}=\lim_{x \to -1^{+}}\frac{f(x)-f(-1)}{x+1}?\]

\textcolor{red}{\underline{En $-1^{-}$:}}

\[\lim_{x \to -1^{-}}\frac{f(x)-f(-1)}{x+1}=\lim_{x \to -1^{-}}\frac{-(x+1)}{x+1}=-1\]

\textcolor{red}{\underline{En $-1^{+}$:}}

\[\lim_{x \to -1^{+}}\frac{f(x)-f(-1)}{x+1}=\lim_{x \to -1^{+}}\frac{x+1}{x+1}=+1\]

\textcolor{green}{\[\text{Non, }\lim_{x \to -1^{-}}\frac{f(x)-f(-1)}{x+1}\neq\lim_{x \to -1^{+}}\frac{f(x)-f(-1)}{x+1}\text{ Donc f n'est dérivable en -1}\]}
\section*{\underline{Exercice 3}(5 points) :}
\subsection*{ Dans chacun des cas suivants, calculer la dérivé de $f$}
\[
f(x)=\sqrt{\frac{x+1}{x-1}}\quad\quad g(x)=(x-1)\sqrt{2-3x}\quad\quad h(x)=(\frac{x-1}{x-2})^{3}\quad I(x)=\frac{\sqrt{x}-1}{\sqrt{x}+1}\quad\quad J(x)=\frac{-1}{(2x+1^{2})(x+2)}
\]
\section*{\textcolor{red}{\underline{Correction Exercice 3} (5 points) :}}
\subsection*{ Calculons la dérivée de $f$ dans chaque cas}
\[
f'(x)=\sqrt{\frac{x+1}{x-1}}=\frac{(\frac{x+1}{x-1})'}{2\sqrt{\frac{x+1}{x-1}}}=\frac{\frac{(x-1)-(x+1)}{(x-1)^{2}}}{2\sqrt{\frac{x+1}{x-1}}}=\frac{\frac{-2}{(x-1)^{2}}}{2\sqrt{\frac{x+1}{x-1}}}=\frac{-2}{2(x-1)^{2}\sqrt{\frac{x+1}{x-1}}}
\]
\[\textcolor{red}{\boxed{f'(x)=\frac{-2}{2(x-1)^{2}\sqrt{\frac{x+1}{x-1}}}}}\]

\[g'(x)=(x-1)\sqrt{2-3x}=\sqrt{2-3x}+\frac{-3(x-1)}{2\sqrt{2-3x}}=\frac{2(2-3x)-3(x-1)}{2\sqrt{2-3x}}=\frac{-9x+7}{2\sqrt{2-3x}}\]
\[\textcolor{red}{\boxed{g'(x)=\frac{-9x+7}{2\sqrt{2-3x}}}}\]
\section*{\underline{Exercice 4} (3 points) :}
\subsection*{a) Cercle trigonométrique (0,5 pt $\times$ 3 )}
Contruire le cerlce trigonométrique en graduant l'axe des \textbf{cos} et des 
\textbf{sin} et en plaçant les angles remarquables.
%\[\cos(\frac{\pi}{6})=\cdots\quad\quad \cos(\frac{\pi}{4})=\cdots \quad\quad \cos(\frac{\pi}{3})=\cdots\quad\quad \cos(\frac{\pi}{2})=\cdots \]
\subsection*{a) Compléter les formules (0,5 pt $\times$ 3 )}
\[\cos(\pi-x)=\cdots\quad\quad \sin(\pi-x)=\cdots\quad\quad \cos(\frac{\pi}{2}+x)=\cdots\]
\section*{\textcolor{red}{\underline{Correction Exercice 4} (3 points) :}}
\end{document}