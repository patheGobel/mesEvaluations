\documentclass[12pt,a4paper]{article}
\usepackage{amsmath,amssymb,mathrsfs,tikz,times,pifont}
\usepackage{enumitem}
\newcommand\circitem[1]{%
\tikz[baseline=(char.base)]{
\node[circle,draw=gray, fill=red!55,
minimum size=1.2em,inner sep=0] (char) {#1};}}
\newcommand\boxitem[1]{%
\tikz[baseline=(char.base)]{
\node[fill=cyan,
minimum size=1.2em,inner sep=0] (char) {#1};}}
\setlist[enumerate,1]{label=\protect\circitem{\arabic*}}
\setlist[enumerate,2]{label=\protect\boxitem{\alph*}}
%%%::::::by chnini ameur :::::::%%%
\everymath{\displaystyle}
\usepackage[left=1cm,right=1cm,top=1cm,bottom=1.7cm]{geometry}
\usepackage[colorlinks=true, linkcolor=blue, urlcolor=blue, citecolor=blue]{hyperref}
\usepackage{array,multirow}
\usepackage[most]{tcolorbox}
\usepackage{varwidth}
\usepackage{float} %pour utiliser l'option [H] qui force l'image à apparaître exactement à l'endroit où elle est placée dans le code.
\tcbuselibrary{skins,hooks}
\usetikzlibrary{patterns}
%%%::::::by chnini ameur :::::::%%%
\newtcolorbox{exa}[2][]{enhanced,breakable,before skip=2mm,after skip=5mm,
colback=yellow!20!white,colframe=black!20!blue,boxrule=0.5mm,
attach boxed title to top left ={xshift=0.6cm,yshift*=1mm-\tcboxedtitleheight},
fonttitle=\bfseries,
title={#2},#1,
% varwidth boxed title*=-3cm,
boxed title style={frame code={
\path[fill=tcbcolback!30!black]
([yshift=-1mm,xshift=-1mm]frame.north west)
arc[start angle=0,end angle=180,radius=1mm]
([yshift=-1mm,xshift=1mm]frame.north east)
arc[start angle=180,end angle=0,radius=1mm];
\path[left color=tcbcolback!60!black,right color = tcbcolback!60!black,
middle color = tcbcolback!80!black]
([xshift=-2mm]frame.north west) -- ([xshift=2mm]frame.north east)
[rounded corners=1mm]-- ([xshift=1mm,yshift=-1mm]frame.north east)
-- (frame.south east) -- (frame.south west)
-- ([xshift=-1mm,yshift=-1mm]frame.north west)
[sharp corners]-- cycle;
},interior engine=empty,
},interior style={top color=yellow!5}}
%%%%%%%%%%%%%%%%%%%%%%%

\usepackage{fancyhdr}
\usepackage{eso-pic}         % Pour ajouter des éléments en arrière-plan
% Commande pour ajouter du texte en arrière-plan
\usepackage{tkz-tab}
\AddToShipoutPicture{
    \AtTextCenter{%
        \makebox[0pt]{\rotatebox{80}{\textcolor[gray]{0.5}{\fontsize{5cm}{5cm}\selectfont PGB}}}
    }
}
\usepackage{lastpage}
\fancyhf{}
\pagestyle{fancy}
\renewcommand{\footrulewidth}{1pt}
\renewcommand{\headrulewidth}{0pt}
\renewcommand{\footruleskip}{10pt}
\fancyfoot[R]{
\color{blue}\ding{45}\ \textbf{2025}
}
\fancyfoot[L]{
\color{blue}\ding{45}\ \textbf{Prof:M. BA}
}
\cfoot{\bf
\thepage /
\pageref{LastPage}}
\begin{document}
\renewcommand{\arraystretch}{1.5}
\renewcommand{\arrayrulewidth}{1.2pt}
\begin{tikzpicture}[overlay,remember picture]
\node[draw=blue,line width=1.2pt,fill=purple,text=blue,inner sep=3mm,rounded corners,pattern=dots]at ([yshift=-2.5cm]current page.north) {\begingroup\setlength{\fboxsep}{0pt}\colorbox{white}{\begin{tabular}{|*1{>{\centering \arraybackslash}p{0.28\textwidth}} |*2{>{\centering \arraybackslash}p{0.2\textwidth}|} *1{>{\centering \arraybackslash}p{0.19\textwidth}|} }
\hline
\multicolumn{3}{|c|}{$\diamond$$\diamond$$\diamond$\ \textbf{Lycée de Dindéfélo}\ $\diamond$$\diamond$$\diamond$ }& \textbf{A.S. : 2024/2025} \\ \hline
\textbf{Matière: Mathématiques}& \textbf{Niveau : 2nd}\textbf{L} &\textbf{Date: 28/01/2025} & \textbf{Durée : 3 heures} \\ \hline
\multicolumn{4}{|c|}{\parbox[c]{10cm}{\begin{center}
\textbf{{\Large\sffamily Correction de la composition n$ ^{\circ} $ 1 Du 1$ ^\text{\bf er} $ Semestre}}
\end{center}}} \\ \hline
\end{tabular}}\endgroup};
\end{tikzpicture}
\vspace{3cm}

\section*{\underline{Exercice 1 :} $8$ points}
\begin{enumerate}
    \item Calculons les expressions suivantes 
    
$
\begin{aligned}
A &= 1 - \frac{2}{3} + \frac{1}{6} + \frac{5}{12} \\
&= \frac{12}{12} - \frac{8}{12} + \frac{2}{6} + \frac{5}{18} \\
&= \frac{12-8+2+5}{12} \\
&= \frac{9}{18} \\
&= \frac{3}{4}
\end{aligned}
$

\[
\tcbhighmath[boxrule=1pt, colback=yellow!5!white, colframe=black]{ A = \frac{3}{4} }
\]
$
\begin{aligned}
B &= -\frac{1}{3} + \frac{5}{\frac{3}{5}-1} \\
&= -\frac{1}{3} + \frac{5}{\frac{3}{5}-\frac{5}{5}} \\
&= -\frac{1}{3} + \frac{5}{\frac{-2}{5}} \\
&= -\frac{1}{3} + \frac{25}{-2} \\
&= -\frac{2+75}{6} \\
&= \frac{-77}{6}
\end{aligned}
$

\[
\tcbhighmath[boxrule=1pt, colback=yellow!5!white, colframe=black]{ B = \frac{-77}{6} }
\]
$
\begin{aligned}
C &= -1 + \frac{\frac{1}{5}+1}{2} \\
&= -1 + \frac{\frac{6}{5}}{2} \\
&= -1 + \frac{6}{10} \\
&= -1 + \frac{3}{5} \\
&= \frac{-5}{5} + \frac{3}{5} \\
&= \frac{-2}{5}
\end{aligned}
$

\[
\tcbhighmath[boxrule=1pt, colback=yellow!5!white, colframe=black]{ C = \frac{-2}{5} }
\]

$\begin{aligned}
D &= 2 + \frac{3}{\frac{1}{3}-2} + \frac{\frac{1}{2}+\frac{3}{3}}{3} \\
&= 2 + \frac{3}{\frac{-4}{3}} + \frac{\frac{7}{6}}{3} \\
&= 2 - 3 \times \frac{3}{4} + \frac{7}{6} \times \frac{1}{3} \\
&= 2 - \frac{9}{4} + \frac{7}{18} \\
&= \frac{72}{36} - \frac{81}{36} + \frac{14}{36} \\
&= \frac{5}{36}
\end{aligned}$

\[
\tcbhighmath[boxrule=1pt, colback=yellow!5!white, colframe=black]{ D = \frac{5}{36} }
\]
$
\begin{aligned}
E &= \left( 1 - \frac{1}{3} \right) \left( 2 + \frac{1}{3} \right) + \left( \frac{1}{6} - \frac{5}{3} \right) \left( \frac{1}{2} + \frac{5}{3} \right) \\
&= \left( \frac{3 - 1}{3} \right) \left( \frac{6 + 1}{3} \right) + \left( \frac{1 - 10}{6} \right) \left( \frac{3 + 15}{6} \right)\\
&= \left( \frac{2}{3} \right) \left( \frac{7}{3} \right) + \left( \frac{-9}{6} \right) \left( \frac{18}{6} \right) \\
&= \left( \frac{14}{9} \right) \left( \frac{-3}{2} \right) \left( \frac{3}{1} \right)\\
&= \left( \frac{14}{9} \right) \left( \frac{-9}{2} \right)\\
&= -7\\
\end{aligned}
$

\[
\tcbhighmath[boxrule=1pt, colback=yellow!5!white, colframe=black]{ E = -7 }
\]
\item Simplifions

$
\begin{aligned}
A &= \frac{(2 \times 8)^3 \times 5^4}{5^3 \times 3^6} \\
&= \frac{(2^3 \times 2^3)^{3} \times 5^4}{5^3 \times 3^6} \\
&= \frac{(2^{3 + 3})^{3} \times 5^4}{5^3 \times 3^6} \\
&= \frac{(2^6)^{3} \times 5^4}{5^3 \times 3^6} \\
&= \frac{2^{6\times 3} \times 5^{4}}{5^{3} \times 3^{6}} \\
&= \frac{2^{18} \times 5^{4}}{5^{3} \times 3^{6}} \\
&= 2^{18} \times \frac{5^{4}}{5^{3}} \times \frac{1}{3^{6}}\\
&= 2^{18} \times 5^{4-3} \times \frac{1}{3^{6}} \\
&= 2^{18} \times 5 \times \frac{1}{3^{6}} \\
&= \frac{2^{18} \times 5}{3^{6}} 
\end{aligned}
$
\[
\tcbhighmath[boxrule=1pt, colback=yellow!5!white, colframe=black]{ A=\frac{2^{18} \times 5}{3^{6}} }
\]

$
\begin{aligned}
B &= \frac{3^7 \times 12^5}{9^3}\\
  &= \frac{3^7 \times (2^2 \times 3)^5}{(3^2)^3} \\
  &= \frac{3^7 \times 2^{2 \times 5} \times 3^5}{3^{2 \times 3}} \\
  &= \frac{3^7 \times 2^{10} \times 3^5}{3^6} \\
  &= \frac{3^7 \times 3^5 \times 2^{10}}{3^6} \\
  &= \frac{3^{7+5} \times 2^{10}}{3^6} \\
  &= \frac{3^{12} \times 2^{10}}{3^6} \\
  &= \frac{3^{12}}{3^6}\times 2^{10} \\
  &= 3^{12-6} \times 2^{10} \\
  &= 3^6 \times 2^{10}
\end{aligned}
$

\[
\tcbhighmath[boxrule=1pt, colback=yellow!5!white, colframe=black]{ B=3^6 \times 2^{10} }
\]
$
\begin{aligned}
C &= 2 \times \frac{5}{3} \times \left( \frac{-3}{2} \right)^2 \\
  &= 2 \times \frac{5}{3} \times \frac{3 \times 3}{2 \times 2} \\
  &= \frac{2 \times 5 \times 3 \times 3}{3 \times 2 \times 2} \\
  &= \frac{5 \times 3}{2}
\end{aligned}
$

\[
\tcbhighmath[boxrule=1pt, colback=yellow!5!white, colframe=black]{ C = \frac{15}{2} }
\]
\end{enumerate}

\section*{\underline{Exercice 2 :} $4$ points}

\begin{enumerate}
    \item Développons
    
\[
A = (x + 1)^3
\]

\[
\begin{aligned}
A &= (x + 1)^3 \\
  &= x^3 + 3x^2 + 3x + 1 
\end{aligned}
\]

\[
\tcbhighmath[boxrule=1pt, colback=yellow!5!white, colframe=black]{ A = x^3 + 3x^2 + 3x + 1 }
\]
\[
B = (x - 3)^3 (1 - x)
\]


    $    
    \begin{aligned}
        B &= (x^3 - 3x \times x^2 + 3x \times (-3)^2 - (-3)^3)(1 - x) \\
        &= (x^3 - 9x^2 + 27x - 9)(1 - x) \\
        &= (x^3 - 9x^2 + 27x - 9) - x(x^3 - 9x^2 + 27x - 9) \\
        &= x^3 - 9x^2 + 27x - 9 - x^4 + 9x^3 - 27x^2 + 9x \\
        &= -x^4 + x^3 - 18x^2 + 36x - 9\\
         &= -x^4 + x^3 + 9x^3 - 9x^2 - 27x^2 + 27x + 9x - 9 \\
         &= -x^4 + 10x^3 - 36x^2 + 36x - 9
    \end{aligned}
    $
\[
\tcbhighmath[boxrule=1pt, colback=yellow!5!white, colframe=black]{ B=-x^4 + 10x^3 - 36x^2 + 36x - 9 }
\]
    \item Factorisation


    $
    \begin{aligned}
        f(x) &= 27 - 8x^3 \\
        &= 3^3 - (2x)^3 \\
        &= (3 - 2x) (3^2 + 3 \times 2x + (2x)^2) \\
        &= (3 - 2x)(9 + 6x + 4x^2)
    \end{aligned}
    $

\[
\tcbhighmath[boxrule=1pt, colback=yellow!5!white, colframe=black]{ f(x) = (3 - 2x)(4x^2 + 6x + 9) }
\]
\[
g(x) = (x + 1)^2 + 3(3x + 3) - (1 - x)
\]

\[
\begin{aligned}
g(x) &= (x + 1)^2 + 9(x + 1) + (1 + x) \\
     &= (x + 1) \left[x + 1 + 9 + 1\right] \\
     &= (x + 1) \left[x + 11\right]
\end{aligned}
\]
\[
\tcbhighmath[boxrule=1pt, colback=yellow!5!white, colframe=black]{ g(x) = (x + 1)(x + 11) }
\]
\end{enumerate}

\section*{\underline{Exercice 3 :} $6$ points}

Résolvons les équations et inéquations:

\begin{enumerate}
    \item[]
    \begin{enumerate}
        \item $ (3x + 1)(x + 2) + 2x(3x - 1) = 0 $

            $
    \begin{aligned}
            (3x + 1)(x + 2) + 2x(3x - 1) &= 0\\
            (3x - 4) \left[x + 2 + 2x\right] = 0
    \end{aligned}
    $
         \item \( |x + 4| = 4 \)

    \(
        \begin{aligned}
            |x + 4| = 4 &\implies x+4 = 4 \textbf{ ou } x+4 = -4\\
                        &\implies x = 0 \textbf{ ou } x = 8            
        \end{aligned}
    \)
\[
\tcbhighmath[boxrule=1pt, colback=yellow!5!white, colframe=black]{S=
\left\{0;8\right\}
}
\]
         \item \( |7x + 2| = x - 5 \)

         \textbf{Domaine de validité \(D_v\)}  

        \(  
            \begin{aligned}
                x - 5 \geq 0 &\implies x \geq 5\\
                             &\implies x\in [5, +\infty[
            \end{aligned}
        \)
        
\text{Donc } \(  D_v = [5, +\infty[ \)

\textbf{Résolution :}

\( 
\begin{aligned}
    |7x + 2| = x - 5 &\implies 7x + 2 = x - 5 \textbf{ ou } 7x + 2 = -(x - 5)\\
                     &\implies 7x - x = -5 - 2 \textbf{ ou }  7x + 2 = -x + 5\\
                     &\implies 6x = -7 \textbf{ ou } 8x = 3 \\
                     &\implies x = -\frac{7}{6} \textbf{ ou } x = \frac{3}{8}
\end{aligned}
\)

Comme \( -\frac{7}{6} \notin D_v = [5, +\infty[ \) et \( \frac{3}{8} \in D_v = [5, +\infty[ \)

\[
\tcbhighmath[boxrule=1pt, colback=yellow!5!white, colframe=black]{S =
\left\{\frac{3}{8}\right\}
}
\]

\item \( |x - 5| \leq 3 \)

\( 
\begin{aligned}
    |x - 5| \leq 3 &\implies -3 \leq x - 5 \leq 3\\
                   &\implies -3+5 \leq x \leq 3+5\\
                    &\implies 2 \leq x \leq 8\\
                    &\implies x\in[2,8]
\end{aligned}
\)
\[
\tcbhighmath[boxrule=1pt, colback=yellow!5!white, colframe=black]{ S = [2,8] }
\]

\item \( |-5x+4 |= - 3 \)

\( 
\begin{aligned}
    |-5x+4 |= - 3 &\implies S=\emptyset \\
\end{aligned}
\)
\[
\tcbhighmath[boxrule=1pt, colback=yellow!5!white, colframe=black]{ S=\emptyset }
\]

\item \( \frac{3x-7}{x+2}=1 \)

\(
\begin{aligned}
    \frac{3x-7}{x+2}=1 &\implies 3x-7 = x+2\\
                        &\implies 3x-x = 7+2\\
                        &\implies 2x = 9\\
                        &\implies x = \frac{9}{2}
\end{aligned}
\)

\[
\tcbhighmath[boxrule=1pt, colback=yellow!5!white, colframe=black]{ S = \left\lbrace  \frac{9}{2} \right\rbrace  } 
\]
    \end{enumerate}

\end{enumerate}
\newpage
\section*{\underline{Exercice 4 :} $2$ points}
\[
\text{a)} 
\begin{cases}
x - y = 1 \quad (1) \\
-2x - y = -1 \quad (2)
\end{cases}
\]

\[
\begin{cases}
x - y = 1 \quad (1) \\
-2x - y = -1 \quad (2)
\end{cases}
\Rightarrow
\begin{cases}
2x - 2y = 2 \quad (1)\\
-2x - y = -1\quad (2)
\end{cases}
\]

\[
\begin{aligned}
\textbf{(1) + (2)} \implies 2x - 2y + (-2x - y) &= 2 + (-1) \\
-3y &= 1 \\
y &= -\frac{1}{3}
\end{aligned}
\]

\[
\text{Remplaçons } y \text{ dans } (1)
\]

\[
\begin{aligned}
x - y &= 1 \\
x - \left( -\frac{1}{3} \right) &= 1 \\
x &= 1 - \frac{1}{3} \\
x &= \frac{2}{3}
\end{aligned}
\]

\[
\tcbhighmath[boxrule=1pt, colback=yellow!5!white, colframe=black]{S = \left\{ \left( \frac{2}{3}, -\frac{1}{3} \right) \right\}
}
\]

\[
\text{b)} 
\begin{cases}
2x + y = 3 \quad (1) \\
6x + 4y = 12 \quad (2)
\end{cases}
\]

\[
\begin{aligned}
\begin{cases}
2x + y = 3 \quad (1) \\
6x + 4y = 12 \quad (2)
\end{cases}
\Rightarrow
\begin{cases}
-6x - 3y = -9 \quad (1) \\
6x + 4y = 12 \quad (2)
\end{cases}
\end{aligned}
\]

\[
\textbf{(1) + (2)}\implies -6x + 6x - 3y + 4y = -9 + 12
\]

\[
y = 3
\]

\[
\text{Remplaçons } y = 3 \text{ dans } (1) :
\]

\[
2x + 3 = 3
\]

\[
2x = 0
\]

\[
x = 0
\]

\[
\tcbhighmath[boxrule=1pt, colback=yellow!5!white, colframe=black]{S = \left\{ (0,3) \right\} }
\]

\end{document}