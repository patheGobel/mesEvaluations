\documentclass[12pt]{article}
\usepackage{stmaryrd}
\usepackage{graphicx}
\usepackage[utf8]{inputenc}

\usepackage[french]{babel}
\usepackage[T1]{fontenc}
\usepackage{hyperref}
\usepackage{verbatim}

\usepackage{color, soul}

\usepackage{pgfplots}
\pgfplotsset{compat=1.15}
\usepackage{mathrsfs}

\usepackage{amsmath}
\usepackage{amsfonts}
\usepackage{amssymb}
\usepackage{tkz-tab}

\usepackage{tikz}
\usetikzlibrary{arrows, shapes.geometric, fit}


\usepackage[margin=2cm]{geometry}
\begin{document}

\begin{minipage}{0.5\textwidth}
	Ministère de l'éducation nationale  \\
	Inspection académique de Kédougou   \\
	Lycée de Dindéferlo            \\
	Cellule de mathématiques            \\
	M. BA                          \\
	Classe : $2^{nd}$S  \\
\end{minipage}
\begin{minipage}{0.5\textwidth}
	Année scolaire 2023-2024 \\
	Date : 07-06-2024 \\
	Durée : 3h 00 \\
\end{minipage}

\begin{center}
	\textbf{{\underline{Devoir N3 Du Second Semestre}}}
\end{center}

\section*{Exercice 1 (7 points) :}
\subsection*{1) Résoudre les équations suivantes}
A=$x^{2}-12x+36=0$ \textbf{1pts}

B=$3x^{2}-7|x|+4 = 0$ \textbf{2pts}

C=$(x+\frac{1}{x})^{2}-3(x+\frac{1}{x})+2=0$ \textbf{2pts}

D=$(\frac{1}{x^{2}+x+1})^{2}-(\frac{3}{x^{2}+x+1})=-2$ \textbf{2pts}
\section*{Exercice 2 (6 points) :}
1) Déterminer les valeurs du paramètre réel \( m \) pour lesquelles l'équation 

(E1):\( (m - 1)x^2 + (m - 5)x + m - 2 = 0 \) 

admet deux solutions distinctes non nulles de signes contraires. \textbf{2pts}

2) Pour quelles valeurs du paramètre réel \( m \), l'équation

(E2):\( (m - 4)x^2 + (m + 2)x - m = 0 \) 
 
admet-elle deux racines distinctes strictement positives ? \textbf{2pts}

3) Pour quelles valeurs du paramètre \( m \), l'équation
 
(E3):\( (m + 1)x^2 + (2m + 3)x + m + 2 = 0 \) 

admet-elle deux solutions distinctes strictement négatives ? \textbf{2pts}
\section*{Exercice 3 (7 points) :}
Soit $A\begin{pmatrix} 1 \\ 2\end{pmatrix}\;,\quad B\begin{pmatrix} 3 \\ 4\end{pmatrix}$ deux points du plan et $(D_{1})\ :\ 2x-3y+5=0.$

1) Déterminer un système d'équations paramétriques des droites $(AB)$ et $(D_{1}).$\textbf{2pts}

2) Soit $(D_{2})\ :\ \left\lbrace\begin{array}{rcl} x&=&2-t \\ y&=&3+4t\end{array}\right.$

a) $A\begin{pmatrix} 1 \\ 2\end{pmatrix}$ et $D\begin{pmatrix} 0 \\ 11\end{pmatrix}$ appartiennent-ils à $(D_{2})\;\ ?$ \textbf{1pts=0,5pt+0,5pt}

b) Déterminer l'équation cartésienne de $(D_{2})$. \textbf{2pts}

3) Déterminer les coordonnées de point d'intersection de $(D_{1})$ et $(D_{2})$. \textbf{2pts}
\end{document}