\documentclass[12pt]{article}
\usepackage{stmaryrd}
\usepackage{graphicx}
\usepackage[utf8]{inputenc}

\usepackage[french]{babel}
\usepackage[T1]{fontenc}
\usepackage{hyperref}
\usepackage{verbatim}

\usepackage{color, soul}

\usepackage{pgfplots}
\pgfplotsset{compat=1.15}
\usepackage{mathrsfs}

\usepackage{amsmath}
\usepackage{amsfonts}
\usepackage{amssymb}
\usepackage{tkz-tab}

\usepackage{tikz}
\usetikzlibrary{arrows, shapes.geometric, fit}


\usepackage[margin=2cm]{geometry}
\begin{document}

\begin{minipage}{0.5\textwidth}
	Ministère de l'éducation nationale  \\
	Inspection académique de Kédougou   \\
	Lycée de Dindéfelo            \\
	Cellule de mathématiques            \\
	M. BA                          \\
	Classe : Tle  \\
\end{minipage}
\begin{minipage}{0.5\textwidth}
	Année scolaire 2023-2024 \\
	Date : 08-05-2024 \\
	Durée : 3h 00 \\
\end{minipage}

\begin{center}
	\section*{\textcolor{red}{\underline{Correction Du Devoir N1 Du Second Semestre}}}
\end{center}

\section*{Exercice 1 (4 points) :}
1) Résoudre dans $\mathbb{R}$ les équations suivantes  (1 pt $\times$ 2)\\
a)$\ln(x) = \ln(5-x)$ ;\quad\quad  b)$\ln(5-x)=-5$ \\
2)Résoudre dans $\mathbb{R}$ les inéquations suivantes (1 pt $\times$ 2)\\
d)$\ln(x-2)<0$ ;\quad\quad e)$1<\ln(5-x)$
\section*{\textcolor{red}{\underline{Correction Exercice 1} (4 points) :}}
1) Résolvons dans $\mathbb{R}$ les équations suivantes  (1 pt $\times$ 2)\\
a)$\ln(x) = \ln(5-x)$\\
L'équation n'a de sens que si $x>0$ et $5-x>0$

Posons $x=0$ et $5-x=0 \Rightarrow x=5$

\definecolor{cqcqcq}{rgb}{0.7529411764705882,0.7529411764705882,0.7529411764705882}
\begin{tikzpicture}[line cap=round,line join=round,>=triangle 45,x=1cm,y=1cm]
%\draw [color=cqcqcq,, xstep=1cm,ystep=1cm] (-7,-10) grid (-22,17);
\clip(-22,3) rectangle (12,10);
\draw [line width=2pt] (-23,8)-- (-7,8); %première ligne A(-22,8)---B(-7,8)
\draw [line width=2pt] (-22,6)-- (-7,6); %deuxième ligne
\draw [line width=2pt] (-22,5)-- (-7,5); %troisième ligne
\draw [line width=2pt] (-22,4)-- (-7,4); %quatrème ligne
\draw [line width=2pt] (-22,4)-- (-22,8); %première colonne (-22,4)<----(-22,8);
\draw [line width=2pt] (-18,8)-- (-18,4); %deuxième colone  (-18,8)--->(-18,4);
\draw [line width=2pt] (-14,6)-- (-14,4); %troisième colonne(-13,6)-- (-13,4);
\draw [line width=2pt] (-12,6)-- (-12,4); %quatième colonne(-13,6)-- (-13,4);
\draw [line width=2pt] (-7,8)-- (-7,4); %cinquième colonne (-7,8)-->(-7,4);
\draw (-21,7) node[anchor=north west] {$x$};
\draw (-21,5.5) node[anchor=north west] {$x$};
\draw (-15.8,5.9) node[anchor=north west] {$-$};
\draw (-14.3,5.7) node[anchor=north west] {$O$};
\draw (-13.5,5.9) node[anchor=north west] {$+$};
\draw (-10.5,5.9) node[anchor=north west] {$+$};
\draw (-21,4.7) node[anchor=north west] {$5-x$};
\draw (-15.8,4.7) node[anchor=north west] {$+$};
\draw (-13.5,4.7) node[anchor=north west] {$+$};
\draw (-12.3,4.7) node[anchor=north west] {$O$};
\draw (-10.5,4.7) node[anchor=north west] {$-$};
\draw (-18,7) node[anchor=north west] {$-\infty$};
\draw (-14.2,7) node[anchor=north west] {$0$};
\draw (-12.2,7) node[anchor=north west] {$5$};
\draw (-8,7) node[anchor=north west] {$+\infty$};
\end{tikzpicture}

Donc le domaine de validité de l'equation est $D=\left]0; 5\right[ $

Résolution

$\ln(x) = \ln(5-x)\Rightarrow x=5-x \Rightarrow 2x=5\Rightarrow x=\frac{5}{2}$

A-t-on $\frac{5}{2} \in D ?$

Oui $\frac{5}{2} \in D$ danc S=$\left\lbrace \frac{5}{2} \right\rbrace $
\section*{Exercice 2 (11 points) :}
Soit  $f$  la fonction définie par : $f(x) =\frac{x^{2}-2x+1}{x}$ .\\
    1- a - Donner  l’ensemble de définition  Df .\textbf{(0,5pt)}\\
b - Montrer que pour tout x $\in$ Df ; $f(x) =x-2+\frac{1}{x} $.\textbf{(1pt)}\\
       2  - Etudier les limites de f aux bornes des intervalles de définition.\textbf{(2pts)}\\ 
              En déduire que f admet une asymptote verticale.\textbf{(0,5pt)}\\
       3 - Calculer  f'(x) puis étudier son signe. Dresser le tableau de variation de f \textbf{(2pt)}.\\
             En déduire les extrema  de f.\textbf{(0,5pt)}\\
        4 - Vérifier que  I (0 ; -2) est centre de symétrie de Cf.\textbf{(1pt)} \\
        5 -  Montrer la droite (D) d’équation  y = x - 2  est une asymptote à (Cf).\textbf{(1pt)} \\
             Etudier la position de (Cf) par rapport à (D).\textbf{(1pt)}\\
         6 - Construire (Cf) dans un repère orthonormal.\textbf{(1,5pt)}\\
\section*{\textcolor{red}{\underline{Correction Exercice 2} (11 points) :}}
\section*{Exercice 3 (6 points) :}
Le tableau ci-dessous  donne la quantité de matière première X en tonnes (t) et le chiffre d’affaire Y en millions francs (f) d’une entreprise. On considère  la série double(x ; y) :

\begin{tabular}{|c|c|c|c|c|c|c|c|}
\hline
X & 2 & 2,5 & 3 & 3,5 & 4 & 4,5 & 5 \\
\hline
Y & 21 & 25 & 29 & 30 & 40 & 46 & 53\\
\hline
\end{tabular}\\
\begin{itemize}
\item[1)] Représenter le nuage de points.\\
\item[2)] Calculer la variance de X et la variance de Y.\\
\item[3)]  Calculer la covariance de X et Y.\\
\item[4)] Calculer le coefficient de corrélation linéaire r.\\
\item[5)] a)  Déterminer l équation de la droite de régression de Y en X puis la représenter\\
\item[b)]  En déduire une estimation du chiffre d’affaire pour 8 tonnes de matière première.\\
\item[c)]  Pour un chiffre d’affaire 100000000 calculer la quantité de matière première.
\end{itemize}
\section*{\textcolor{red}{\underline{Correction Exercice 3} (6 points) :}}
\end{document}