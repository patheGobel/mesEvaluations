\documentclass[12pt]{article}
\usepackage{stmaryrd}
\usepackage{graphicx}
\usepackage[utf8]{inputenc}

\usepackage[french]{babel}
\usepackage[T1]{fontenc}
\usepackage{hyperref}
\usepackage{verbatim}

\usepackage{color, soul}

\usepackage{pgfplots}
\pgfplotsset{compat=1.15}
\usepackage{mathrsfs}

\usepackage{amsmath}
\usepackage{amsfonts}
\usepackage{amssymb}
\usepackage{tkz-tab}

\usepackage{tikz}
\usetikzlibrary{arrows, shapes.geometric, fit}


\usepackage[margin=2cm]{geometry}

\begin{document}

\begin{minipage}{0.5\textwidth}
	Ministère de l'éducation nationale  \\
	Inspection académique de Kédougou   \\
	Lycée de Dindéferlo            \\
	Cellule de mathématiques            \\
	M. BA                          \\
	Classe : TS  \\
\end{minipage}
\begin{minipage}{0.5\textwidth}
	Année scolaire 2023-2024 \\
	Date : 10-06-2024 \\
	Durée : 4h 00 \\
\end{minipage}

\begin{center}
	\textbf{{\underline{\textcolor{red}{Composition Du Second Semestre}}}}
\end{center}

\section*{\textcolor{red}{\underline{Exercice 1} (4 points) :}}
On considère les integrales suivantes : $I_{0}=\int_{0}^{\frac{\pi}{3}}\frac{dx}{\cos (x)}$ et $J_{n}=\int_{0}^{\frac{\pi}{3}}\frac{\sin^{n}(x)}{\cos(x)}dx,$ $n\in\mathbb{N}^{*}$

1) Calculer $I_{0}=\int_{0}^{\frac{\pi}{3}}\sin^{n}(x)\cos(x)dx$  puis en déduire $I_{n+2}-I_{n}$ en fonxtion de  n. \textbf{0,5pt+0,5pt}

2)Calculer $I_{1}$ puis en déduire $I_{3}$ et $I_{5}$. \textbf{0,5pt+0,5pt+0,5pt}

3) a) Soit $f$ la fonction qui à tout $x\in\left[0, \frac{\pi}{3}\right]$ associe $f(x)=\ln\left( \tan\left(\frac{x}{2}+\frac{\pi}{4}\right)\right) .$ 

Montrer que  $f$ est une primitive de la fonction $g$ définie par $g(x)=\frac{1}{\cos(x)}$, $x\in\left[0, \frac{\pi}{3}\right]$. \textbf{0,5pt}

b) En déduire $I_{0}$ puis $I_{2}$ \textbf{0,5pt+0,5pt}
\section*{\textcolor{red}{\underline{Exercice 2} (6 points) :}}
\subsection*{Partie A: (2pts)}
Une urne contient trois boules jaunes, cinq boules rouges et deux boules vertes.

1) On tire simultanément trois boules de l'urne.

a) Quel est le nombre de tirage unicolore ? \textbf{0,5pt}

b)Quel est le nombre de tirage comportant exactement deux boules de même couleur ? \textbf{0,5pt}

2) On tire successivement sans remise trois boules.

a) Quel est le nombre de tirage comportant des boules rouges uniquement ? \textbf{0,5pt}

b)Quel est le nombre de tirage ne comportant pas de boule verte aux deuxième tirage ?\textbf{0,5pt}
\subsection*{Partie B: (4pts)}
Un porte-monnaie contient quatre pièce de \textbf{500F} et une pièce de \textbf{200F}.

Un enfant tire au hasard 3 pièces de ce porte-monnaie.

1)Calculer la probabilité de l'évènement A : <<tiré 3 pièce de \textbf{500F}>>.\textbf{0,5pt}

2)Soit X la variable aléatoire égale au nombre de pièce de \textbf{500F} figurant parmis les 3 pièces tirées.

Déterminer la loi de probabilité de X puis représenter la fonction de répartition de X.\textbf{1pt+1pt}

3)Calculer l'espérance mathématique et l'écart-type. \textbf{0,5pt+0,5pt}

4)L'enfant rèpète 5 fois l'expérience A en remettant à chaque fois les trois pièces tirées.

Quelle est la probabilité que l'évenement A se réalise tois fois a l'issu des 5 tirages ? \textbf{0,5pt}

L'enfant efectue n fois cette épreuve. Détermine la plus petite valeur de n pour que la probabilité d'obtenir au moins une fois l'évènement A soit supérieur à 0,99. \textbf{1pt}
\section*{\textcolor{red}{\underline{Problème} (10 points) :}}
\subsection*{\underline{Partie A:}(2pts)}
On considère l'équation différentielle \textbf{(E)}: $\frac{1}{2}y'+y=3e^{-2x}+2$

1-Déterminer \textbf{a} pour que la fonction $v$ définie par $v(x)$=$\textbf{a}xe^{-2x}+2$ soit une solution de l'équation \textbf{(E')} \textbf{0,5pt.}

2-Donner les solutions de l'équation \textbf{(E')}: $\frac{1}{2}y'+y=0$.\textbf{0,5pt}

3-a) Montrer que $u$ est une solution de \textbf{(E)} si et seulement si $v-u$ est solution de \textbf{(E')}.\textbf{0,5pt}

	b) En déduire les solutions de \textbf{(E)}. \textbf{0,25pt}
	
4-Déterminer la solution de $u$ de l'équation \textbf{(E)} vérifiant $u(0)=0$. \textbf{0,25pt}
\subsection*{\underline{Partie B:}(8	pts)}
\[\text{On définit la fonction} f \text{ sur } \mathbb{R} \text{ par: } f(x) = \begin{cases} 
  2(3x-1)e^{-2x}+2, & \text{si } x \leq 0 \\
  \frac{x\ln x}{1+x} & \text{si } x > 0 
\end{cases} \]

On note $C_{f}$ sa courbe représentative dans un repère d'unité graphique 4cm.

I. On définit la fonction $g$ sur $\left]0 ; \infty\right[ $ par $g(x)=1+x+\ln(x)$

1-a) Calculer les limites de $g$ en 0 et en $+\infty$.\textbf{0,25pt+0,25pt}

b) Etudier le sens de variation de $g$.\textbf{0,5pt}

c)Dresser le tableau de variation de $g$.\textbf{0,5pt}

2-Montrer que l'équation $g(x)=0$ admet une unique solution $\alpha$ dans $\left]0 ; \infty\right[$ en déduire que   $\alpha \in \left]0,2 ; 0,3\right[$ \textbf{0,5pt}

3-Déduire le signe de $g(x)$ suivante les valeurs de x sur $\left]0 ; \infty\right[$.\textbf{0,25pt}

II.Etude de la fonction $f$

1-Etudier la continuité et la dérib=vabilité de $f$ en 0 puis interpréter graphiquement les résultats.\textbf{1pt}

2-Etudier les limites de $f$ en $+\infty$ et en $-\infty$. \textbf{0,25pt+0,25pt}

3-Etudier les branches infinies en l'inifi. \textbf{0,5pt+0,5pt}

4-Etudier le sens de variation de $f$ sur $\mathbb{R}$( on montrera que pour tout x>0, $f'(x)=\frac{g(x)}{(x+1)^{2}}$).\textbf{0,5pt}

5-Dresser le tableau de variation de $f$.\textbf{0,55pt}

6-Montrer que $f(\alpha)=-\alpha$ puis déterminer l(intersection de $Cf$ avec les axes.\textbf{0,5pt}

7-Tracer $Cf$. \textbf{0,75pt}
\section*{\underline{Partie C}:}
Soit $\beta<0$, on note $A(\beta)$ l'aire de la partie du plan délimité par les droites d'equations $x=\beta$, $x=0$, $y=0$ et la courbe $Cf$

1) On pose $F(x)=(ax+b)e^{-2x}$ détyerminer a et b pour que $F'(x)=(3x-1)e^{-2x}$. \textbf{0,5pt}

2) Calculer $A(\beta)$. \textbf{0,5pt}

3) Calculer sa limites en $-\infty$. \textbf{0,25pt}
\end{document}