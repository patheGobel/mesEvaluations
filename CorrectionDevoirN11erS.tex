\documentclass[12pt]{article}
\usepackage{stmaryrd}
\usepackage{graphicx}
\usepackage[utf8]{inputenc}

\usepackage[french]{babel}
\usepackage[T1]{fontenc}
\usepackage{hyperref}
\usepackage{verbatim}

\usepackage{color, soul}

\usepackage{pgfplots}
\pgfplotsset{compat=1.15}
\usepackage{mathrsfs}

\usepackage{amsmath}
\usepackage{amsfonts}
\usepackage{amssymb}
\usepackage{tkz-tab}

\usepackage{tikz}
\usetikzlibrary{arrows, shapes.geometric, fit}


\usepackage[margin=2cm]{geometry}
\begin{document}

\begin{minipage}{0.5\textwidth}
	Ministère de l'éducation nationale  \\
	Inspection académique de Kédougou   \\
	Lycée de Dindéfélo            \\
	Cellule de mathématiques            \\
	M. BA                          \\
	Classe : 1erS  \\
\end{minipage}
\begin{minipage}{0.5\textwidth}
	Année scolaire 2023-2024 \\
	Date : 14-05-2024 \\
	Durée : 4h 00 \\
\end{minipage}

\begin{center}
	\textbf{{\underline{Devoir N1 Du Second Semestre}}}
\end{center}
\section*{Exercice 1 (6 points) :}
\subsection*{a) Calculer les limites suivantes (1 pt $\times$ 4)}
\[ \lim_{x \to +\infty}-\frac{8}{5}x^{3}+\frac{7}{2}x+8\quad\quad \lim_{x \to -\infty}\frac{5x+3}{x^{2}-4x+1}\quad\quad \lim_{x \to 3}\frac{x^{2}-9}{x-3}\quad\quad \lim_{x \to \frac{1}{5}}\frac{x-2\sqrt{x}}{1-5x} \]
\subsection*{b) Etudier la continuité de f sur son $D_{f}$: (1 pt $\times$ 2)}
\[ f(x) = \begin{cases} 
  \frac{x^2-1}{x-1}, & \text{si } x < 1 \\
   2x,\quad\quad\quad\quad\quad \text{si } 1 \leq x \leq 2\\
  \sqrt{x+2}, & \text{si } x > 2
\end{cases} \]
\section*{\textcolor{red}{\underline{Correction Exercice 1} (6 points) :}}
\[ \lim_{x \to +\infty}-\frac{8}{5}x^{3}+\frac{7}{2}x+8=\lim_{x \to +\infty}-\frac{8}{5}x^{3}=-\infty\]
\[\text{donc, }\textcolor{red}{\boxed{\lim_{x \to +\infty}-\frac{8}{5}x^{3}+\frac{7}{2}x+8=-\infty}}\]

\[\lim_{x \to -\infty}\frac{5x+3}{x^{2}-4x+1}=\lim_{x \to -\infty}\frac{5x}{x^{2}}=\lim_{x \to -\infty}\frac{5}{x}=0\]
\[\text{donc, }\textcolor{red}{\boxed{\lim_{x \to -\infty}\frac{5x+3}{x^{2}-4x+1}=0}}\]

\[\lim_{x \to 3}\frac{x^{2}-9}{x-3}=\lim_{x \to 3}(x+3)=6\]
\[\text{donc, }\textcolor{red}{\boxed{\lim_{x \to 3}\frac{x^{2}-9}{x-3}=6}}\]

\[\lim_{x \to \frac{1}{5}}\frac{x-2\sqrt{x}}{1-5x} \begin{cases} 
  \lim_{x \to \frac{1}{5}}x-2\sqrt{x}=\frac{1}{5}-2\sqrt{\frac{1}{5}}\\
  \lim_{x \to \frac{1}{5}}=0
\end{cases}\]
Par quotient, forme infinie\\
\definecolor{cqcqcq}{rgb}{0.7529411764705882,0.7529411764705882,0.7529411764705882}
\begin{tikzpicture}[line cap=round,line join=round,>=triangle 45,x=1cm,y=1cm]
%\draw [color=cqcqcq,, xstep=1cm,ystep=1cm] (-7,-10) grid (-22,17);
\clip(-22,3) rectangle (12,10);
\draw [line width=2pt] (-23,8)-- (-7,8); %première ligne A(-22,8)---B(-7,8)
\draw [line width=2pt] (-22,6)-- (-7,6); %deuxième ligne
\draw [line width=2pt] (-22,5)-- (-7,5); %troisième ligne
\draw [line width=2pt] (-22,5)-- (-22,8); %première colonne (-22,4)<----(-22,8);
\draw [line width=2pt] (-18,8)-- (-18,5); %deuxième colone  (-18,8)--->(-18,4);
\draw [line width=2pt] (-12,6)-- (-12,5); %troisième colonne(-13,6)--->(-13,4);
\draw [line width=2pt] (-7,8)-- (-7,5); %quatrième colonne (-7,8)-->(-7,4);
\draw (-21,7) node[anchor=north west] {$x$};
\draw (-21,5.5) node[anchor=north west] {$1-5x$};
\draw (-15.8,5.9) node[anchor=north west] {$+$};
\draw (-10.5,5.9) node[anchor=north west] {$-$};
\draw (-18,7) node[anchor=north west] {$-\infty$};
\draw (-12.2,7) node[anchor=north west] {$\frac{1}{5}$};
\draw (-8,7) node[anchor=north west] {$+\infty$};
\end{tikzpicture}
\textcolor{red}{En $\frac{1}{5}^{-}$}
\[\lim_{x \to \frac{1}{5}}\frac{x-2\sqrt{x}}{1-5x}=\frac{\frac{1-2\sqrt{5}}{5}}{0^{+}}=-\infty\]
\[\text{donc, }\textcolor{red}{\boxed{\lim_{x \to \frac{1}{5}^{-}}\frac{x-2\sqrt{x}}{1-5x}=-\infty}}\]
\textcolor{red}{En $\frac{1}{5}^{+}$}
\[\lim_{x \to \frac{1}{5}^{+}}\frac{x-2\sqrt{x}}{1-5x}=\frac{\frac{1-2\sqrt{5}}{5}}{0^{-}}=+\infty\]
\[\text{donc, }\textcolor{red}{\boxed{\lim_{x \to \frac{1}{5}^{+}}\frac{x-2\sqrt{x}}{1-5x}=+\infty}}\]
\subsection*{b) Etudions la continuité de f sur son $D_{f}$: (1 pt $\times$ 2)}
\[ f(x) = \begin{cases} 
  \frac{x^2-1}{x-1}, & \text{si } x < 1 \\
   2x,\quad\quad\quad\quad\quad \text{si } 1 \leq x \leq 2\\
  \sqrt{x+2}, & \text{si } x > 2
\end{cases} \]

\[\text{Posons} f(x) = \begin{cases} 
  f_{1}(x)=\frac{x^2-1}{x-1}, & \text{si } x < 1 \\
   f_{2}(x)=2x,\quad\quad\quad\quad\quad \text{si } 1 \leq x \leq 2\\
  f_{3}(x)=\sqrt{x+2}, & \text{si } x > 2
\end{cases} \]
Le domaine de définition de f
\begin{itemize}
\item[•] $f_{1}$ existe ssi  $x-1\neq0$ et $x < 1$

donc $x\neq 1$ et $x \in \left] -\infty, 1\right[$

Comme  $1 \notin \left] -\infty, 1\right[$

Donc $Df_{1}=\left] -\infty, 1\right[$
\item[•] $f_{2}$ est polynôme définie si $1 \leq x \leq 2$

Donc $Df_{2}=\left[  1, 2\right] $
\item[•] $f_{3}$ existe ssi $x+2\geq 0$ et $x > 2$

donc $x\geq -2$ et $x \in \left]2, +\infty\right[  \Rightarrow x \in\left]-2, +\infty\right[$ et $x \in \left]2, +\infty\right[ \Rightarrow x\in (\left]-2, +\infty\right[\cap\left]2, +\infty\right[)$

Donc $Df_{3}=\left]2, +\infty\right[ $

$Df=Df_{1} \cap Df_{2} \cap Df_{3}=\left]-\infty, 1\right[\cap\left[  1, 2\right] \cap\left]2, +\infty\right[=\mathbb{R}$

Finalement $Df=\mathbb{R}$
\end{itemize}
Etudions la continuité de f en $1$ et en $2$

\textcolor{red}{\underline{Continuité de f en 1:}}
\textcolor{red}{\[\text{A-t-on, } \lim_{x \to 1^{-}}f(x)=\lim_{x \to 1^{+}}f(x) ?\]}

\[\lim_{x \to 1^{-}}f(x)=\lim_{x \to 1^{-}}\frac{x^2-1}{x-1}=\lim_{x \to 1^{-}}(x+1)=2\]
\[\lim_{x \to 1^{+}}f(x)=\lim_{x \to 1^{+}}2x=2\]

\textcolor{red}{\[\text{Oui, } \lim_{x \to 1^{-}}f(x)=\lim_{x \to 1^{+}}f(x)\text{ Donc f est continue en 1.}\]}

\textcolor{red}{\underline{Continuité de f en 2:}}
\textcolor{red}{\[\text{A-t-on, } \lim_{x \to 2^{-}}f(x)=\lim_{x \to 2^{+}}f(x) ?\]}

\[\lim_{x \to 2^{-}}f(x)=\lim_{x \to 2^{-}}2x=4\]
\[\lim_{x \to 2^{+}}f(x)=\lim_{x \to 2^{+}}\sqrt{x+2}=2\]

\textcolor{red}{\[\text{Non, } \lim_{x \to 2^{-}}f(x)\neq\lim_{x \to 2^{+}}f(x)\text{ Donc f n'est pas continue en 2. }\]}

\textcolor{green}{Finalement, f est continue en $1$ mais pas en $2$}
\section*{Exercice 2 (4 points) :}
\subsection*{a) Etudier la dérivabilté de $f$ sur domaine de définition $D_{f}$}
\[ f(x) = \begin{cases} 
  3x^{2}, & \text{si } x \leq 0 \\
  -2x^{2}, & \text{si } x > 0 
\end{cases} \]
\subsection*{b) Soit $f(x)=|x|$ }
Montrer que $f$ est continue en 0 mais n'est pas dérivable en 0.
\section*{\textcolor{red}{\underline{Correction Exercice 2} (4 points) :}}
\subsection*{a) Etudions la dérivabilté de $f$ sur domaine de définition $Df$}
$Df=\mathbb{R}$
\textcolor{red}{\[\text{A-t-on, } \lim_{x \to 0^{-}}\frac{f(x)-f(0)}{x-0}=\lim_{x \to 0^{+}}\frac{f(x)-f(0)}{x-0}?\]}
\[\lim_{x \to 0^{-}}\frac{f(x)-f(0)}{x-0}=\lim_{x \to 0^{-}}\frac{3x^{2}}{x}=\lim_{x \to 0^{-}}3x=0\]
\[\lim_{x \to 0^{+}}\frac{f(x)-f(0)}{x-0}=\lim_{x \to 0^{+}}\frac{-2x^{2}}{x}=\lim_{x \to 0^{+}}-2x=0\]

\textcolor{red}{\[\text{Oui, } \lim_{x \to 0^{-}}\frac{f(x)-f(0)}{x-0}=\lim_{x \to 0^{+}}\frac{f(x)-f(0)}{x-0}\text{ Donc f est dérivable en 0. }\]}
\subsection*{b) Soit $f(x)=|x|$ }
Montrons que $f$ est continue en 0 mais n'est pas dérivable en 0

Ecrivons $f$ sans valeur absolue d'abord.
\[ |x| = \begin{cases} 
  -x, & \text{si } x \leq 0 \\
  x, & \text{si } x > 0 
\end{cases} \]

\textcolor{red}{\underline{Continuité de f en 0:}}
\textcolor{red}{\[\text{A-t-on, } \lim_{x \to 0^{-}}f(x)=\lim_{x \to 0^{+}}f(x) ?\]}
\[\lim_{x \to 0^{-}}f(x)=\lim_{x \to 0^{-}}-x=0\]
\[\lim_{x \to 0^{+}}f(x)=\lim_{x \to 0^{+}}x=0\]
\textcolor{red}{\[\text{Oui, } \lim_{x \to 0^{-}}f(x)=\lim_{x \to 0^{+}}f(x)\text{ Donc f est continue en 0.}\]}

\textcolor{red}{\underline{Dérivabilité de f en 0:}}
\textcolor{red}{\[\text{A-t-on, } \lim_{x \to 0^{-}}\frac{f(x)-f(0)}{x-0}=\lim_{x \to 0^{+}}\frac{f(x)-f(0)}{x-0}?\]}

\[\lim_{x \to 0^{-}}\frac{f(x)-f(0)}{x-0}=\lim_{x \to 0^{-}}\frac{-x}{x}=-1\]
\[\lim_{x \to 0^{+}}\frac{f(x)-f(0)}{x-0}=\lim_{x \to 0^{+}}\frac{x}{x}=1\]

\textcolor{red}{\[\text{Non, } \lim_{x \to 0^{-}}\frac{f(x)-f(0)}{x-0}=\lim_{x \to 0^{+}}\frac{f(x)-f(0)}{x-0}\text{ Donc f n'est  pas dérivable en 0. }\]}

Conclusion, $f$ est continue sur $\mathbb{R}$ mais n'y est pas dérivable.
\section*{Exercice 3 (6 points) :}
\subsection*{ Dans chacun des cas suivants, calculer la dérivé de $f$}
\begin{itemize}
\item[a)]$f(x)=3x^{2}-5x+2$\quad\quad $b)f(x)=\frac{1}{2}x^{4}+\frac{5}{3}x^{3}-x^{2}-5x$\quad\quad $c)f(x)=(x^{2}+1)(5x-7)$
\item[d)]$d)f(x)=\frac{-2x+1}{3x+5}$\quad\quad $e)f(x)=x+3+\frac{4}{x-1}$\quad\quad $f)f(x)=\sqrt{2x+1}$
\end{itemize}
\section*{\textcolor{red}{\underline{Correction Exercice 3} (6 points) :}}
\section*{Exercice 4 (4 points) :}
\subsection*{a) Dans chaque cas donner la mesure principale de $\alpha$ (0,75 pt $\times$ 2+0,5 )}
\[\alpha =\frac{129\pi}{8}\quad\quad \alpha =\frac{108\pi}{7}\quad\quad \alpha = -26\pi \]
\subsection*{a) Compléter les formules (0,75 pt $\times$ 2+0,5 )}
\[\cos(\pi+x)=\cdots\quad\quad \sin(\pi+x)=\cdots\quad\quad \cos(\frac{\pi}{2}-x)=\cdots\]
\section*{\textcolor{red}{\underline{Correction Exercice 4} (4 points) :}}
\end{document}