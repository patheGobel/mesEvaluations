\documentclass[12pt]{article}
\usepackage{stmaryrd}
\usepackage{graphicx}
\usepackage[utf8]{inputenc}

\usepackage[french]{babel}
\usepackage[T1]{fontenc}
\usepackage{hyperref}
\usepackage{verbatim}

\usepackage{color, soul}

\usepackage{pgfplots}
\pgfplotsset{compat=1.15}
\usepackage{mathrsfs}

\usepackage{amsmath}
\usepackage{amsfonts}
\usepackage{amssymb}
\usepackage{tkz-tab}

\usepackage{tikz}
\usetikzlibrary{arrows, shapes.geometric, fit}


\usepackage[margin=2cm]{geometry}
\begin{document}

\begin{minipage}{0.5\textwidth}
	Ministère de l'éducation nationale  \\
	Inspection académique de Kédougou   \\
	Cellule de mathématiques            \\
	M. BA\\
	Classe : 1erS
\end{minipage}
\begin{minipage}{0.5\textwidth}
	Année scolaire 2023-2024 \\
	Date : 27-06-2024 \\
	Durée : 4h 00 \\
\end{minipage}

\begin{center}
	\textbf{{\underline{\textcolor{green}{Composition N2 Du Second Semestre}}}}
\end{center}
\section*{\underline{Exercice 1: }\textbf{10 pts}}
Dans chacun des cas suivants étudier les fonctions suivantes:
\[a)f(x)=\frac{7x-9}{4x-5}\quad\quad\textbf{5 pts}\]
\[b)f(x)=\frac{1-x}{2+3x}\quad\quad\textbf{5 pts}\]
\section*{\underline{\textcolor{green}{Correction Exercice 3: \textbf{10 pts}}}}
Dans chacun des cas suivants étudions les fonctions suivantes:
\begin{itemize}
\item \textcolor{green}{Pour $a)f(x)=\frac{7x-9}{4x-5}\quad\quad\textbf{5 pts}$}
\begin{itemize}
\item \textcolor{red}{Domaine de définition}

f existe ssi $4x-5\neq 0$

$4x-5\neq 0 \implies x\neq\frac{5}{4}$

Donc \textcolor{green}{\boxed{Df=\mathbb{R}\setminus\left\lbrace \frac{5}{4}\right\rbrace }}\textbf{ 0,5 pt}

\item \textcolor{red}{Limites aux bornes de Df}

Les bornes de f sont $-\infty$ ; $\frac{5}{4}$ et $+\infty$

\underline{\textcolor{green}{En $-\infty $}}:

\[\lim_{x \to -\infty}f(x)=\lim_{x \to -\infty}\frac{7x-9}{4x-5}=\lim_{x \to -\infty}\frac{7x}{4x}=\frac{7}{4}\]

Donc \textcolor{green}{\boxed{\lim_{x \to -\infty}f(x)=\frac{7}{4}}}\textbf{ 0,5 pt}

\underline{\textcolor{green}{En $+\infty $}}:

\[\lim_{x \to +\infty}f(x)=\lim_{x \to +\infty}\frac{7x-9}{4x-5}=\lim_{x \to +\infty}\frac{7x}{4x}=\frac{7}{4}\]

Donc \textcolor{green}{\boxed{\lim_{x \to +\infty}f(x)=\frac{7}{4}}}\textbf{ 0,5 pt}

\underline{\textcolor{green}{En $\frac{5}{4}^{-}$}}:

\[\lim_{x \to \frac{5}{4}^{-}}f(x)=\frac{\frac{-1}{4}}{0^{-}}=+\infty\]
Donc \textcolor{green}{\boxed{\lim_{x \to \frac{5}{4}^{-}}f(x)=+\infty}}\textbf{ 0,5 pt}

\underline{\textcolor{green}{En $\frac{5}{4}^{+}$}}:

\[\lim_{x \to \frac{5}{4}^{+}}f(x)=\frac{\frac{-1}{4}}{0^{+}}=-\infty\]
Donc \textcolor{green}{\boxed{\lim_{x \to \frac{5}{4}^{+}}f(x)=-\infty}}\textbf{ 0,5 pt}

\textbf{En résumé:}

\textcolor{green}{$y=\frac{7}{4}$ est une asymptote horizontale à (Cf)}

\textcolor{green}{$x=\frac{5}{4}$ est une asymptote verticale à (Cf)}

\item \textcolor{red}{variation de f}

\begin{align*}
f'(x)&=\frac{(7x-9)'(4x-5)-(4x-5)'(7x-9)}{(4x-5)^{2}}\\
	&=\frac{7(4x-5)-4(7x-9)}{(4x-5)^{2}}\\
	&=\frac{(28x-35)+(-28x+36)}{(4x-5)^{2}}\\
	&=\frac{1}{(4x-5)^{2}}\\
\end{align*}
Donc \textcolor{green}{\boxed{f'(x)=\frac{1}{(4x-5)^{2}}}}

\textcolor{green}{D'après l'expression de la dérivée, $\forall x\in Df, f'(x)>0 $ \text{donc f est croissante.}}\textbf{ 0,5 pt}
 
\end{itemize}

\item \textcolor{red}{Tableau de variation de f}\textbf{ 0,5 pt}
\begin{center}
\definecolor{cqcqcq}{rgb}{0.7529411764705882,0.7529411764705882,0.7529411764705882}
\begin{tikzpicture}[line cap=round,line join=round,>=triangle 45,x=1cm,y=1cm]
%\draw [color=cqcqcq,, xstep=1cm,ystep=1cm] (-7,-10) grid (-22,17);
\clip(-22,-5) rectangle (12,10);
\draw [line width=2pt] (-23,8)-- (-7,8); %première ligne A(-22,8)---B(-7,8)
\draw [line width=2pt] (-22,6)-- (-7,6); %deuxième ligne
\draw [line width=2pt] (-22,4)-- (-7,4); %troisième ligne
\draw [line width=2pt] (-22,-2)-- (-7,-2);%dernière ligne
\draw [line width=2pt] (-22,-2)-- (-22,8); %première colonne
\draw [line width=2pt] (-19,8)-- (-19,-2); %deuxième colone
\draw [line width=2pt] (-13,6)-- (-13,-2); %troisième colonne
\draw [line width=2pt] (-13.1,6)-- (-13.1,-2); %troisième colonne
\draw [line width=2pt] (-7,8)-- (-7,-2); %quatrième colonne
\draw (-21,1.5) node[anchor=north west] {$f(x)$};
\draw (-21,5.5) node[anchor=north west] {$f'(x)$};
\draw (-21,7) node[anchor=north west] {$x$};
\draw (-19,7) node[anchor=north west] {$-\infty$};
\draw (-13.4,6.9) node[anchor=north west] {$\frac{5}{4}$};
\draw (-8,7) node[anchor=north west] {$+\infty$};
%signe de la dérivé
\draw (-17,5.3) node[anchor=north west] {$+$};
%\draw (-13.39,5.3) node[anchor=north west] {$O$};
\draw (-13.39,5.3) node[anchor=north west] {\textbf{\textcolor{red}{O}}};
\draw (-10,5.3) node[anchor=north west] {$+$};

\draw [->,line width=2pt] (-18.4,-1.7) -- (-14.2,3.5);
\draw (-18.8,-1.2) node[anchor=north west] {$\frac{7}{4}$};
\draw (-14.2,3.9) node[anchor=north west] {\textbf{\textcolor{blue}{$+\infty$}}};
\draw (-12.9,-1.2) node[anchor=north west] {\textbf{\textcolor{blue}{$-\infty$}}};
\draw [->,line width=2pt] (-12.5,-1.3) -- (-8,3.5);
\draw (-8,3.9) node[anchor=north west] {$\frac{7}{4}$};
\end{tikzpicture}
\end{center}
\item \textcolor{red}{Tracons $C_f$. \textbf{1,5 pt}}

\begin{center}
\begin{figure}[h]
\centering
\includegraphics[width=1\textwidth]{c5.png}
\caption{Courbe de (Cf)}
\label{fig:monimage}
\end{figure}
\end{center}
\end{itemize}
\newpage
-----------------------------------------------------------------------------------
\begin{itemize}
\item \textcolor{green}{Pour $b)f(x)=\frac{1-x}{2+3x}\quad\quad\textbf{5 pts}$}
\begin{itemize}
\item \textcolor{red}{Domaine de définition}

f existe ssi $2+3x\neq 0$

$2+3x\neq 0 \implies x\neq-\frac{2}{3}$

Donc \textcolor{green}{\boxed{Df=\mathbb{R}\setminus\left\lbrace -\frac{2}{3}\right\rbrace }}\textbf{0,5 pt}

\item \textcolor{red}{Limites aux bornes de Df}

Les bornes de f sont $-\infty$ ; $-\frac{2}{3}$ et $+\infty$

\underline{\textcolor{green}{En $-\infty $}}:

\[\lim_{x \to -\infty}f(x)=\lim_{x \to -\infty}\frac{1-x}{2+3x}=\lim_{x \to -\infty}\frac{-x}{3x}=\frac{-1}{3}\]

Donc \textcolor{green}{\boxed{\lim_{x \to -\infty}f(x)=-\frac{1}{3}}}\textbf{0,5 pt}

\underline{\textcolor{green}{En $+\infty $}}:

\[\lim_{x \to +\infty}f(x)=\lim_{x \to +\infty}\frac{1-x}{2+3x}=\lim_{x \to +\infty}\frac{-x}{3x}=\frac{-1}{3}\]

Donc \textcolor{green}{\boxed{\lim_{x \to +\infty}f(x)=-\frac{1}{3}}}\textbf{0,5 pt}

\underline{\textcolor{green}{En $-\frac{2}{3}^{-}$}}:

\[\lim_{x \to -\frac{2}{3}^{-}}f(x)=\frac{\frac{5}{3}}{0^{-}}=-\infty\]
Donc \textcolor{green}{\boxed{\lim_{x \to -\frac{2}{3}^{-}}f(x)=-\infty}}\textbf{0,5 pt}

\underline{\textcolor{green}{En $-\frac{2}{3}^{+}$}}:

\[\lim_{x \to -\frac{2}{3}^{+}}f(x)=\frac{\frac{5}{3}}{0^{+}}=+\infty\]
\end{itemize}

Donc \textcolor{green}{\boxed{\lim_{x \to -\frac{2}{3}^{+}}f(x)=+\infty}}\textbf{0,5 pt}

En résumé:

\textcolor{green}{$y=-\frac{1}{3}$ est une asymptote horizontale à (Cf)}

\textcolor{green}{$x=-\frac{1}{3}$ est une asymptote verticale à (Cf)}

\item \textcolor{red}{variation de f}
%\frac{1-x}{2+3x}
\begin{align*}
f'(x)&=\frac{(1-x)'(2+3x)-(2+3x)'(1-x)}{(2+3x)^{2}}\\
	&=\frac{-(2+3x)-3(1-x)}{(2+3x)^{2}}\\
	&=\frac{-2-3x-3+3x}{(2+3x)^{2}}\\
	&=\frac{-7}{(2+3x)^{2}}\\
\end{align*}
Donc \textcolor{green}{\boxed{f'(x)=\frac{-5}{(2+3x)^{2}}}}\textbf{ 0,5 pt}

\textcolor{green}{D'après l'expression de la dérivée, $\forall x\in Df, f'(x)<0 $ \text{donc f est décroissante.}}
 
\item \textcolor{red}{Tableau de variation de f. }\textbf{0,5 pt}
\begin{center}
\definecolor{cqcqcq}{rgb}{0.7529411764705882,0.7529411764705882,0.7529411764705882}
\begin{tikzpicture}[line cap=round,line join=round,>=triangle 45,x=1cm,y=1cm]
%\draw [color=cqcqcq,, xstep=1cm,ystep=1cm] (-7,-10) grid (-22,17);
\clip(-22,-5) rectangle (12,10);
\draw [line width=2pt] (-23,8)-- (-7,8); %première ligne A(-22,8)---B(-7,8)
\draw [line width=2pt] (-22,6)-- (-7,6); %deuxième ligne
\draw [line width=2pt] (-22,4)-- (-7,4); %troisième ligne
\draw [line width=2pt] (-22,-2)-- (-7,-2);%dernière ligne
\draw [line width=2pt] (-22,-2)-- (-22,8); %première colonne
\draw [line width=2pt] (-19,8)-- (-19,-2); %deuxième colone
\draw [line width=2pt] (-13,6)-- (-13,-2); %troisième colonne
\draw [line width=2pt] (-13.1,6)-- (-13.1,-2); %troisième colonne
\draw [line width=2pt] (-7,8)-- (-7,-2); %quatrième colonne
\draw (-21,1.5) node[anchor=north west] {$f(x)$};
\draw (-21,5.5) node[anchor=north west] {$f'(x)$};
\draw (-21,7) node[anchor=north west] {$x$};
\draw (-19,7) node[anchor=north west] {$-\infty$};
\draw (-13.4,6.9) node[anchor=north west] {$-\frac{2}{3}$};
\draw (-8,7) node[anchor=north west] {$+\infty$};
%signe de la dérivé
\draw (-17,5.3) node[anchor=north west] {$-$};
%\draw (-13.39,5.3) node[anchor=north west] {$O$};
\draw (-13.39,5.3) node[anchor=north west] {\textbf{\textcolor{red}{O}}};
\draw (-10,5.3) node[anchor=north west] {$-$};

\draw [->,line width=2pt] (-18.4,3.5) -- (-14.2,-1.7);
\draw (-19.2,4) node[anchor=north west] {$-\frac{1}{3}$};
\draw (-14.2,-1.2) node[anchor=north west] {\textbf{\textcolor{blue}{$-\infty$}}};
\draw (-12.9,4) node[anchor=north west] {\textbf{\textcolor{blue}{$+\infty$}}};
\draw [->,line width=2pt] (-12.3,3.6) -- (-8,-1.2);
\draw (-8,-1.2) node[anchor=north west] {$-\frac{1}{3}$};
\end{tikzpicture}
\end{center}
\item \textcolor{red}{Tracons $C_f$.}\textbf{1,5 pt}
\end{itemize}

\begin{center}
\begin{figure}[h]
\centering
\includegraphics[width=1\textwidth]{c6.png}
\caption{Courbe de (Cf)}
\label{fig:monimage}
\end{figure}
\end{center}
\newpage
\section*{\underline{Exercice 2: }\textbf{5 pts}}
Une urne contient 4 boules rouges, 3 boules vertes et une boule noire.

On tire au hasard 3 boules de cette urne.

\begin{itemize}
\item[1)] Calculer le nombre de tirages possibles. \textbf{1 pt}
\item[2)] Quel est le nombre de tirages contenant :
\begin{itemize}
\item[a)] 2 boules rouges. \textbf{1 pt}
\item[b)] 3 boules unicolores. \textbf{1 pt}
\item[c)] 3 boules de couleurs différentes. \textbf{1 pt}
\item[d)] une boule noire. \textbf{1 pt}
\end{itemize}
\end{itemize}
\section*{\underline{\textcolor{green}{Correction Exercice 3: \textbf{5 pts}}}}
Une urne contient 4 boules rouges, 3 boules vertes et une boule noire.

On tire au hasard 3 boules de cette urne.

\begin{itemize}
\item[1)] Soit $\Omega$:<<Le nombre de tirages possibles>>. \textbf{1 pt}
\begin{align*}
card(\Omega)&=C_{8}^{3}\\
		&=56	
\end{align*}
Donc \textcolor{green}{\boxed{card(\Omega)=56}}
\item[2)] Le nombre de tirages contenant :
\begin{itemize}
\item[a)] Soit A:<< Tirer 2 boules rouges >>. \textbf{1 pt}

C'est-à-dire : Avoir <<RRV ou RRN>>
\begin{align*}
card(A)&=C_{4}^{2}\times C_{3}^{1}+C_{4}^{2}\times C_{1}^{1}\\
		&=24
\end{align*}
Donc \textcolor{green}{\boxed{card(A)=24}}
\item[b)] Soit B:<< Tirer 3 boules unicolores >>. \textbf{1 pt}

C'est-à-dire : Avoir <<RRR ou VVV>>
\begin{align*}
card(B)&=C_{4}^{3}+C_{3}^{3}\\
	&=4
\end{align*}
Donc \textcolor{green}{\boxed{card(B)=4}}
\item[c)] Soit C:<< Tirer 3 boules de couleurs différentes >>. \textbf{1 pt}

C'est-à-dire : Avoir <<RVN>>
\begin{align*}
card(C)&=C_{4}^{1} \times C_{3}^{1} \times C_{1}^{1}\\
		&=24
\end{align*}
Donc \textcolor{green}{\boxed{card(C)=24}}
\item[d)] Soit D:<< une boule noire >>. \textbf{1 pt}

C'est-à-dire : Avoir <<$N\overline{N}\overline{N}$>>
\begin{align*}
card(D)&=C_{1}^{1} \times C_{7}^{2}\\
		&=42
\end{align*}
Donc \textcolor{green}{\boxed{card(D)=42}}
\end{itemize}
\end{itemize}
\section*{\underline{Exercice 3: }\textbf{5 pts}}
Soit $(u_n)$ une suite numérique définie par:

\[ u_n = 3n + 1 \]

1. Démontrer que $(u_n)$ est une suite arithmétique et déterminer sa raison et son premier terme. \textbf{1,5 pts}

2. Étudier les variations de la suite $(u_n)$. \textbf{1,5 pts}

3. Calculer la somme des 1000 premiers termes de la suite $(u_n)$. \textbf{2 pts}

\section*{\underline{\textcolor{green}{Correction Exercice 3: \textbf{5 pts}}}}
Soit $(u_n)$ une suite numérique définie par:

\[ u_{n} = 3n + 1 \]

1. Démontrons que $(u_n)$ est une suite arithmétique et déterminons sa raison et son premier terme. \textbf{1,5 pts}
\[u_{n} \text{ est une suite arithmétiques ssi }  u_{n+1}-u_{n}=r \]
\[\text{Calculons } u_{n+1}\]
\begin{align*}
u_{n+1}&=3(n+1) + 1\\
&=3n+4
\end{align*}
\[\text{donc } u_{n+1}=3n+4\]
\[\text{Calculons maintenant } u_{n+1}-u_{n}\]
\[u_{n+1}-u_{n}=3n+4-3n-1\]
\[u_{n+1}-u_{n}=3\]
\[\text{Finalement, } (u_{n}) \text{ est une suite arithmétique de raisons } 3 \text{ et de premier terme  } u_{0}=1\]

2. Étudions les variations de la suite $(u_n)$. \textbf{1,5 pts}
\[\text{Pour ce faire, étudions le signe de } u_{n+1}-u_{n} \text{ D'après ce qui précède, } u_{n+1}-u_{n}>0 \text{ donc la suite est croissante}\]
3. Calculons la somme des 1000 premiers termes de la suite $(u_n)$. \textbf{2 pts}
\begin{align*}
S_{n}&=u_{0}+u_{1}+u_{2}+...+u_{999}\\
	&=\frac{(999+1)(u_{0}+u_{999})}{2}\\
	&=\frac{(1000)(1+2998)}{2}\\
	&=1499500
\end{align*}
Donc \textcolor{green}{\boxed{S_{n}=1499500}}
\end{document}